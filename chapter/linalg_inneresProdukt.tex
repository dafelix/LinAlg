\chapter{Inneres Produkt und Bilinearform}
\begin{definition}
Sei V ein K-VR. Ein \textbf{inneres Produkt} (oder \textbf{Skalarprodukt}) ist eine Abbildung $\phi$ von $V \times V$ nach $K, \phi (v_1,v_2) = (v_1,v_2) \in K,~v_1,v_2 \in V$ mit folgenden Eigenschaften:
\begin{enumerate}
	\item $(v_1+v_2,v_3) = (v_1,v_3)+(v_2,v_3)$
	\item $(\alpha v_1,v_2) = \alpha (v_1,v_2)$
	\item $(v_1,v_2) = \overline{(v_2,v_1)}$ (hermitesch)
	\item $(v,v) \geq 0$, $(v,v) = 0 \Leftrightarrow v=0$ (positiv definit)
\end{enumerate}
\end{definition}
\begin{remark}
V zusammen mit einem inneren Produkt nennt man einen \textbf{euklidischen Raum}\index{Euklidischer Raum} (falls $K=\mathbb{R}$) bzw. \textbf{unitären Raum}\index{Unitärer Raum} (falls $K=\mathbb{C}$).
\end{remark}

\begin{lemma}
Sei $A:V \rightarrow W$ linear und injektiv. Sei $(~,~):W \times W \rightarrow K$ ein inneres Produkt auf $W$. Dann wird durch $p_A(v_1,v_2):=(Av_1,Av_2),~v_1,v_2 \in V$ ein inneres Produkt definiert.
\end{lemma}

\begin{theorem}Cauchy-Schwarz-Ungleichung\index{Cauchy-Schwarz-Ungl.}
\begin{align*}
|(v_1,v_2)| \leq \norm{v_1} \norm{v_2}
\end{align*}
\end{theorem}

\isection{Erhaltung innerer Produkte}
\begin{theorem}
\leavevmode
\begin{compactitem}
\item Eine lineare Abbildung $f: V \to W$ erhält innere Produkte, falls gilt: $(f(v_1), f(v_2))_W = (v_1, v_2)_V$
\item f erhält innere Produkte $\Rightarrow$ $||f(v)||_W = \sqrt{(f(v), f(v))_W} = \sqrt{(v,v)_V} = ||v||_V$
\item $f$ erhält innere Produkte und ist bijektiv $\Rightarrow$ $f^{-1}$ erhält ebenfalls innere Produkte, denn: $(f^{-1}(w_1), f^{-1}(w_2))_V = (w_1, w_2)_W$
\item Sei dim(V) = n < $\infty$, sei $f: V \to W$ linear. TFE:
\begin{enumerate}
\item f erhält das innere Produkt
\item f ist ein Isom. zwischen VR mit innerem Produkt
\item f bildet jede ON-Basis von V auf eine ON-Basis von W ab
\item f bildet eine ON-Basis von V auf eine ON-Basis von W ab
\end{enumerate}
Eine unitäre Abbildung ist ein Isom. $u: V \to V$, der innere Produkte erhält.
Die unitären Abbildungen bilden eine Gruppe U(V). Für $u \in U(V)$ gilt:
\begin{enumerate}
\item Sei dim(V) = n < $\infty$. Dann gilt: $u \in U(V)$ $\Leftrightarrow$ $u$(ON-Basis) = ON-Basis
\item $u \in U(V)$ $\Leftrightarrow$ $\exists u^*$ mit $u^*u=uu^*=id$
\end{enumerate}
\end{compactitem}
\end{theorem}

\isection{Bilinearform}
\begin{definition}
Seien V, W K-VR. Eine Abbildung $\beta: V \times W \to K$ heißt Bilinearform, falls sie linear in jedem Argument ist, d.h.:
\begin{enumerate}
\item $\beta(\alpha_1 v_1 + \alpha_2 v_2, w) = \alpha_1 \beta(v_1, w) + \alpha_2 \beta(v_1, w)$
\item $\beta(v, \alpha_1 w_1 + \alpha_2 w_2) = \alpha_1 \beta(v, w_1) + \alpha_2 \beta(v, w_2)$
\end{enumerate}
\end{definition}
\isection{Bilinearform ausgeartet}
\begin{definition}
Sei $\beta: V \times W \to K$ eine Bilinearform. Dann heißt $\beta$ \textbf{ausgeartet im 1.Argument}, falls es $0 \neq v \in V$ gibt, sodass $\beta(v,w) = 0$ $\forall w \in W$.\\
(ausgeartet im 2.Argument analog)
\end{definition}

\begin{remark}
\leavevmode
\begin{compactitem}
\item Innere Produkte sind stets nicht ausgeartet
\item Bilinearform nicht ausgeartet $\Leftrightarrow$ die darstellende Matrix hat vollen Rang
\item $V = \{ f: \mathbb{R} \to \mathbb{R} | f~stetig \}$, $\beta(f,g) = \displaystyle\int_0^1 f(x) g(x) dx$ ist ausgeartet
\end{compactitem}
\end{remark}

\section{Strukturmatrix}
\begin{definition}
Sei $\beta: V \times W \to K$ eine Bilinearform und dim(V) = n < $\infty$, dim(W) = m < $\infty$. Seien $v_1, …, v_n$ bzw. $w_1, …, w_m$ Basen von V bzw. W.\\
Dann heißt B = ($\beta(v_i, w_j))_{ij}$ $\in K^{n \times m}$ die \textbf{Strukturmatrix}\index{Strukturmatrix} von $\beta$ bez. der gewählten Basen.
\end{definition}

\begin{example}
\leavevmode
\begin{compactitem}
\item Sei $v_1, …, v_n$ Basis von V und $f_1, …, f_n$ die duale Basis. Dann: B = ($f_i(v_j)$) = E
\item Sei $v_1, …, v_n$ ON-Basis von V. Dann: B = (($v_i, v_j))_{ij}$ = E
\item Bez. der Standardbasis ist B = E
\item Bez. der Standardbasen hat $\beta(x,y) := x^tBy$ die Strukturmatrix B
\end{compactitem}
\end{example}

\begin{theorem}
Seien $v_1, …, v_n$ bzw. $w_1, …, w_m$ Basen von V bzw. W. Sei $\beta: V \times W \to K$ eine Bilinearform und B die Strukturmatrix.\\
Sei $V \ni v = x_1v_1 + … + x_nv_n$, $x_i \in K$ und\\
\hspace*{4mm} $W \ni w = y_1w_1 + … + y_mw_m$, $y_j \in K$.\\
Dann gilt: $\beta(v, w) = x^tBy$
\textbf{Folgerung}:
Sei dim(V) = dim(W) = n < $\infty$. Sei $\beta: V \times W \to K$ eine Bilinearform mit Strukturmatrix B. TFE:
\begin{enumerate}
\item $\beta$ ist ausgeartet im 1.Argument
\item $\beta$ ist ausgeartet im 2.Argument
\item $\det(B) = \det(B^t) = 0$
\item $\rg(B) = \rg(B^t) < n$
\item $\ker(B) = \{0\}$
\item $\ker(B^t) = \{0\}$
\end{enumerate}
\end{theorem}
