\chapter{Inneres Produkt (Skalarprodukt)}
\begin{definition}
Sei V ein K-VR. Ein \textbf{inneres Produkt} (oder \textbf{Skalarprodukt}) ist eine Abbildung $\phi$ von $V \times V$ nach $K, \phi (v_1,v_2) = (v_1,v_2) \in K,~v_1,v_2 \in V$ mit folgenden Eigenschaften:
\begin{enumerate}
	\item $(v_1+v_2,v_3) = (v_1,v_3)+(v_2,v_3)$
	\item $(\alpha v_1,v_2) = \alpha (v_1,v_2)$
	\item $(v_1,v_2) = \overline{(v_2,v_1)}$ (hermitesch)
	\item $(v,v) \geq 0$, $(v,v) = 0 \Leftrightarrow v=0$ (positiv definit)
\end{enumerate}
\end{definition}
\begin{remark}
V zusammen mit einem inneren Produkt nennt man einen \textbf{euklidischen Raum}\index{Euklidischer Raum} (falls $K=\mathbb{R}$) bzw. \textbf{unitären Raum}\index{Unitärer Raum} (falls $K=\mathbb{C}$).
\end{remark}

\begin{lemma}
Sei $A:V \rightarrow W$ linear und injektiv. Sei $(~,~):W \times W \rightarrow K$ ein inneres Produkt auf $W$. Dann wird durch $p_A(v_1,v_2):=(Av_1,Av_2),~v_1,v_2 \in V$ ein inneres Produkt definiert.
\end{lemma}

\section{Cauchy-Schwarz-Ungleichung}
\begin{theorem}
\begin{align*}
|(v_1,v_2)| \leq \norm{v_1} \norm{v_2}
\end{align*}
\end{theorem}

\section{Erhaltung innerer Produkte}
\begin{theorem}
\leavevmode
\begin{compactitem}
\item Eine lineare Abbildung $f: V \to W$ erhält innere Produkte, falls gilt: $(f(v_1), f(v_2))_W = (v_1, v_2)_V$
\item f erhält innere Produkte $\Rightarrow$ $||f(v)||_W = \sqrt{(f(v), f(v))_W} = \sqrt{(v,v)_V} = ||v||_V$
\item $f$ erhält innere Produkte und ist bijektiv $\Rightarrow$ $f^{-1}$ erhält ebenfalls innere Produkte, denn: $(f^{-1}(w_1), f^{-1}(w_2))_V = (w_1, w_2)_W$
\item Sei dim(V) = n < $\infty$, sei $f: V \to W$ linear. TFE:
\begin{enumerate}
\item f erhält das innere Produkt
\item f ist ein Isom. zwischen VR mit innerem Produkt
\item f bildet jede ON-Basis von V auf eine ON-Basis von W ab
\item f bildet eine ON-Basis von V auf eine ON-Basis von W ab
\end{enumerate}
Eine unitäre Abbildung ist ein Isom. $u: V \to V$, der innere Produkte erhält.
Die unitären Abbildungen bilden eine Gruppe U(V). Für $u \in U(V)$ gilt:
\begin{enumerate}
\item Sei dim(V) = n < $\infty$. Dann gilt: $u \in U(V)$ $\Leftrightarrow$ $u$(ON-Basis) = ON-Basis
\item $u \in U(V)$ $\Leftrightarrow$ $\exists u^*$ mit $u^*u=uu^*=id$
\end{enumerate}
\end{compactitem}
\end{theorem}

\section{Bilinearform ausgeartet}
\begin{definition}
Sei $\beta: V \times W \to K$ eine Bilinearform. Dann heißt $\beta$ \textbf{ausgeartet im 1.Argument}, falls es $0 \neq v \in V$ gibt, sodass $\beta(v,w) = 0$ $\forall w \in W$.\\
(ausgeartet im 2.Argument analog)
\end{definition}
\begin{remark}
\leavevmode
\begin{compactitem}
\item Innere Produkte sind stets nicht-ausgeartet
\item $V = \{ f: \mathbb{R} \to \mathbb{R} | f~stetig \}$, $\beta(f,g) = \displaystyle\int_0^1 f(x) g(x) dx$ ist ausgeartet.
\end{compactitem}
\end{remark}