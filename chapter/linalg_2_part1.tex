\section{Diagonalisierbarkeit}
\begin{enumerate}
	\item f heißt diagonalisierbar, falls eine der Eigenschaften gilt:
	\begin{enumerate}
		\item $\exists$ Basis von V aus Eigenvektoren (auf der Diagonalen stehen Eigenwerte)
		\item $\exists$ Matrix $S \in Gl_n(K)$, sodass $S^{-1}AS$ Diagnoalgestalt hat
	\end{enumerate}
	\item f ist diagonalisierbar $\Leftrightarrow$ $\chi_f(x) = \prod \limits_{i=1}^{r} (x- \alpha_i)^{mi}$. Es gilt dann: $V=\bigoplus \limits_{i=1}^r V_f(\alpha_i)$
	\item f ist diagonalisierbar $\Leftrightarrow \mu_f(x) = \prod \limits_{i=1}^{r}(x-\alpha_i)$ (mit paarweise verschiedenen $\alpha_i$) 
\end{enumerate}

\section{Ideal}
Sei R ein Ring. Eine Teilmenge $I \subseteq R, I \neq \emptyset$ heißt Ideal, falls
\begin{enumerate}
	\item $a_1,a_2 \in I \Rightarrow a_1+a_2 \in I$
	\item $a \in I \Rightarrow r_1 a r_2 \in I, \forall r_1,r_2 \in R$
\end{enumerate}
Bemerkungen:
\begin{compactitem}
	\item ker(f) ist stets Ideal
	\item falls $1 \in I \Rightarrow I = R$
	\item seien $I_1,I_2 \subseteq R$ Ideale, dann sind auch $I_1 \cap I_2, I_1+I_2, I_1 \cdot I_2$ wieder Ideale
	\item In $\mathbb{Z}$ sind alle Ideale von der Form $I=a\mathbb{Z}, a \in \mathbb{Z}$
	\item falls $I_1+I_2 = R$, so nennt man $I_1$ und $I_2$ teilerfremd.
	\item $I_1+I_2 = R \Rightarrow I_1 \cap I_2 = I_1 \cdot I_2$
	\item R HIR: $(p)$ ist maximal $ \Leftrightarrow $ $p$ ist irreduzibel $\Leftrightarrow$ $p$ ist Primideal
	\item Sei R komm. Ring. Dann gilt:
	\begin{enumerate}
		\item P Primideal $\Leftrightarrow$ $\QR{R}{P}$ Integritätsbereich
		\item M max. Ideal $\Leftrightarrow$ $\QR{R}{M}$ ist Körper
	\end{enumerate}
	\item Sei R Ring mit 1. Sei $I \subsetneq R $ Ideal. Dann gibt es ein maximales Ideal M mit $ I \subseteq M$. Insbesondere existiern max. Ideale.
\end{compactitem}

\section{Ringe}
\begin{enumerate}
	\item R ist ein euklidischer Ring, falls es eine funktion $\phi : R\\ \{0\} \rightarrow \mathbb{N}_0$ gibt, sodass gilt:
	$\forall a,b \in R, b \neq 0$, gibt es $q,r \in R$ mit $a=qb+r$, $r=0$ oder $\phi(r) < \phi (b)$
	\item Integritätsbereich: R ist kommutativ und Nullteilerfrei
	\item R heißt faktoriell, falls jedes $0 \neq x \in R$ eine eindeutige Primzerlegung hat
	\item Euklidisch $\Rightarrow$ HIR $\Rightarrow$ faktoriell
\end{enumerate}
Bemerkungen:
\begin{compactitem}
	\item $\mathbb{Z}$ mit $\phi(a) = |a|$ ist euklidisch
	\item R euklidisch $\Rightarrow$ R ist Hauptidealring
	\item Sei R kommutativ. Dann gilt $R^n \simeq R^m \Leftrightarrow n=m$
\end{compactitem}
%Todo: Hauptidealring$

\section{Moduln}
Eine Menge $M \neq \emptyset$ ist ein R-(Links-) Modul, falls es eine Verknüpfung $+:M \times M \rightarrow M, (m_1,m_2) \mapsto m_1+m_2$ und eine weitere Verknüpfung $\cdot R \times M \rightarrow M, (r,m) \mapsto rm$ gibt, sodass:
\begin{itemize}
	\item $(M,+)$ ist abelsche Gruppe
	\item $(r_1,r_2)m = r_1 m + r_2 m$\\$r(m_1+_m2) = rm_1+rm_2$
	\item $1_R m = m$
\end{itemize}

\subsection{freier Modul}
\subsection{Torsionselement}

\subsection{R-Modulhomomorphismus}
Sei R ein Ring, M,N zwei R-Moduln. Dann ist $f:M\mapsto N$ ein R-Modulhom., falls:
\begin{itemize}
	\item $f$ ist ein Gruppenhom., d.h. $f(m_1+m_2) = f(m_1)+f(m_2)$
	\item $f(rm) = rf(m), \forall r \in R, n\in \mathbb{N}$
\end{itemize}

\subsection{Untermoduln}
Sei M ein R-Modul. Eine Teilmenge $N \subseteq M$ heißt Untermodul, falls:
\begin{enumerate}
	\item $(N,+) ist Untergruppe$
	\item $rn \in N, \forall r \in R, n\in N$
\end{enumerate}
Es gilt der Hom. Satz, insbesondere $\QR{M}{\ker(f)} \simeq \im(f)$
Bemerkung: $(\ker(f),\im(f))$ sind Untermoduln

