\section{Diagonalisierbarkeit}
\begin{theorem}
\leavevmode
\begin{enumerate}
	\item f heißt diagonalisierbar, falls eine der Eigenschaften gilt:
	\begin{enumerate}
		\item $\exists$ Basis von V aus Eigenvektoren (auf der Diagonalen stehen Eigenwerte)
		\item $\exists$ Matrix $S \in Gl_n(K)$, sodass $S^{-1}AS$ Diagnoalgestalt hat
	\end{enumerate}
	\item f ist diagonalisierbar $\Leftrightarrow$ $\chi_f(x) = \prod \limits_{i=1}^{r} (x- \alpha_i)^{m_i}$. Es gilt dann: $V=\bigoplus \limits_{i=1}^r V_f(\alpha_i)$
	\item f ist diagonalisierbar $\Leftrightarrow \mu_f(x) = \prod \limits_{i=1}^{r}(x-\alpha_i)$ (mit paarweise verschiedenen $\alpha_i$) 
\end{enumerate}
\end{theorem}

\section{Ideal}
\begin{definition}
Sei R ein Ring. Eine Teilmenge $I \subseteq R, I \neq \emptyset$ heißt Ideal, falls
\begin{enumerate}
	\item $a_1,a_2 \in I \Rightarrow a_1+a_2 \in I$
	\item $a \in I \Rightarrow r_1 a r_2 \in I, \forall r_1,r_2 \in R$
\end{enumerate}
\end{definition}
\begin{remark}
\begin{itemize}
	\item ker(f) ist stets Ideal
	\item falls $1 \in I \Rightarrow I = R$
	\item seien $I_1,I_2 \subseteq R$ Ideale, dann sind auch $I_1 \cap I_2, I_1+I_2, I_1 \cdot I_2$ wieder Ideale
	\item In $\mathbb{Z}$ sind alle Ideale von der Form $I=a\mathbb{Z}, a \in \mathbb{Z}$
	\item falls $I_1+I_2 = R$, so nennt man $I_1$ und $I_2$ teilerfremd.
	\item $I_1+I_2 = R \Rightarrow I_1 \cap I_2 = I_1 \cdot I_2$
	\item R HIR: $(p)$ ist maximal $ \Leftrightarrow $ $p$ ist irreduzibel $\Leftrightarrow$ $p$ ist Primideal
	\item Sei R komm. Ring. Dann gilt:
	\begin{enumerate}
		\item P Primideal $\Leftrightarrow$ $\QR{R}{P}$ Integritätsbereich
		\item M max. Ideal $\Leftrightarrow$ $\QR{R}{M}$ ist Körper
	\end{enumerate}
	\item Sei R Ring mit 1. Sei $I \subsetneq R $ Ideal. Dann gibt es ein maximales Ideal M mit $ I \subseteq M$. Insbesondere existieren max. Ideale.
\end{itemize}
\end{remark}
\section{Ringe}
\begin{definition}
\begin{enumerate}
	\item R ist ein \textbf{euklidischer} Ring, falls es eine Funktion $\phi : R\backslash \{0\} \rightarrow \mathbb{N}_0$ gibt, sodass gilt:
	$\forall a,b \in R, b \neq 0$, gibt es $q,r \in R$ mit $a=qb+r$, $r=0$ oder $\phi(r) < \phi (b)$
	\item \textbf{Integritätsbereich}: R ist kommutativ und Nullteilerfrei
	\item R heißt \textbf{faktoriell}, falls jedes $0 \neq x \in R$ eine eindeutige Primzerlegung hat
\end{enumerate}
\end{definition}
\begin{remark}
\begin{itemize}
	\item Euklidisch $\Rightarrow$ HIR $\Rightarrow$ faktoriell
	\item $\mathbb{Z}$ mit $\phi(a) = |a|$ ist euklidisch
	\item R euklidisch $\Rightarrow$ R ist Hauptidealring
	\item Sei R kommutativ. Dann gilt $R^n \simeq R^m \Leftrightarrow n=m$
\end{itemize}
\end{remark}
%Todo: Hauptidealring$

\section{Moduln}
\begin{definition}
Eine Menge $M \neq \emptyset$ ist ein R-(Links-) Modul, falls es eine Verknüpfung $+:M \times M \rightarrow M, (m_1,m_2) \mapsto m_1+m_2$ und eine weitere Verknüpfung $\cdot:R \times M \rightarrow M, (r,m) \mapsto rm$ gibt, sodass:
\begin{itemize}
	\item $(M,+)$ ist abelsche Gruppe
	\item $(r_1,r_2)m = r_1 m + r_2 m$\\$r(m_1+m_2) = rm_1+rm_2$
	\item $1_R m = m$
\end{itemize}
\end{definition}
\begin{remark}
Sei R HIR
\begin{itemize}
	\item Sei $ M \subseteq F = R^n$ ein R-Untermodul. Dann ist $M \simeq R^k$ mit $k \leq n$
	\item Sei $M=<m_1,...,m_n>$ e-e. Modul. Dann gilt: M frei $\Leftrightarrow$ M torsionsfrei
	\item Ist M e-e. R-Modul, so ist $M \simeq T(M) \oplus F, F \simeq R^k$
\end{itemize}
\end{remark}

\subsection{freier Modul}
\begin{theorem}
\leavevmode
\begin{enumerate}
	\item Sei $F = \bigoplus \limits_{i \in I} R e_i $ (F ist freier Modul und die $e_i$ sind eine \enquote{Basis} von F, d.h. $r e_i = 0 \Leftrightarrow r=0$).
	Für $i\in I$ sei ein Element $m_i \in M$ gegeben, wobei M ein R-Modul ist.
	Dann gibt es genau einen Modulhom. $f:F \rightarrow M $ mit $f(e_i) = m_i$
	\item Sei $f:M \twoheadrightarrow F$ ein surjektiver Modulhom., wobei $F = \bigoplus \limits_{i \in I}R e_i$ frei ist. Dann gibt es einen Teilmodul $N \subseteq M$ mit
	\begin{itemize}
		\item $M = N \oplus \ker(f)$
		\item $N \simeq F$
	\end{itemize}
\end{enumerate}
\end{theorem}
\subsection{Torsionselement}
\begin{definition}
Sei R komm. Integritätsbereich. Sei M ein R-Modul. Ein Element $m\in M$ heißt Torsionselement, falls es ein $r\in R \backslash \{0\}$ gibt mit $rm=0$.

Sei $T(M) := \{m \in M | m~\text{ist torsion} \} \ni 0$. $M$ heißt Torsionsfrei, falls $T(M) = 0$.
\end{definition}

\begin{remark}
$~~$
\begin{itemize}
	\item $T(M) \subseteq M$ ist Teilmodul
	\item $\QR{M}{T(M)}$ ist torsionsfrei
\end{itemize}
\end{remark}

\subsection{$\pi$-Torsionselement}
\begin{definition}
Sei $\pi \in R$ irreduzibel und M ein R-Modul. Dann heißt $T_\pi (M):= \{m \in M | \exists e \in \mathbb{N}: \pi^em = 0 \} \subseteq T(M)$ $\pi$-Torsionsteilmodul von M.
%TODO: inhaltlich prüfen$
\end{definition}

\subsection{R-Modulhomomorphismus}
\begin{definition}
Sei R ein Ring, M,N zwei R-Moduln. Dann ist $f:M\mapsto N$ ein R-Modulhom., falls:
\begin{itemize}
	\item $f$ ist ein Gruppenhom., d.h. $f(m_1+m_2) = f(m_1)+f(m_2)$
	\item $f(rm) = rf(m), \forall r \in R, n\in \mathbb{N}$
\end{itemize}
\end{definition}

\subsection{Untermoduln}
\begin{definition}
Sei M ein R-Modul. Eine Teilmenge $N \subseteq M$ heißt Untermodul, falls:
\begin{enumerate}
	\item $(N,+)$ ist Untergruppe
	\item $rn \in N, \forall r \in R, n\in N$
\end{enumerate}
\end{definition}
\begin{remark}
$~~$
\begin{itemize}
	\item Es gilt der Hom. Satz, insbesondere $\QR{M}{\ker(f)} \simeq \im(f)$
	\item $\ker(f),\im(f)$ sind Untermoduln.
\end{itemize}
\end{remark}

\subsection{K[x]-Modul}
\begin{theorem}
Sei $A \in End_K(V)$. Dann wird V zu einem K[x]-Modul vermöge\footnote{vermöge = "kann das"} $f(x)v := f(A)v, f\in K[x], v \in V.~(V_A=V)$\\
Seien $A,B \in M_n(K), V=K^n$. TFE\footnote{TFE = The following are equivalent}:
\begin{enumerate}
	\item $A \approx B$
	\item $V_A \simeq V_B$ als K[x]-Modul
	\item $\QR{K[x]^n}{<M_A(x)>} \simeq \QR{K[x]^n}{<M_B(x)>}$ als K[x]-Moduln
	\item $M_A(x) \sim M_B(x)$
	\item $M_A(x)$ und $M_B(x)$ haben die gleichen Elementarteiler
	\item $M_A(x)$ und $M_B(x)$ haben die gleichen Invariantenteiler
\end{enumerate}
\end{theorem}

\section{Elementarteiler}
Sei R euklidischer Ring. Sei M ein e-e. R-Modul. Dann gibt es $k \in \mathbb{N}_0$ (Rang von M) und $\epsilon_1,...,\epsilon_m \in R, \epsilon_i \notin R^\times$ (Elementarteiler) mit
\begin{enumerate}
	\item $M \simeq R^k \oplus \bigoplus \limits_{i=1}^m \QR{R}{\epsilon_i R}$
	\item $\epsilon_1 | \epsilon_2 | ... | \epsilon_m$
\end{enumerate}
\textbf{Zusatz}: $k,\epsilon_i$ sind eindeutig durch den Isomorphietyp von M bestimmt\\
\textbf{Anwendung}: Seien $M_1,M_2$ zwei e-e. Moduln, $\epsilon_1,...,\epsilon_m$ die Elementarteiler von $M_1$, $\mu_1,...,\mu_n$, die Elementarteiler von $M_2$. Dann gilt:
\begin{itemize}
	\item $M_1 \simeq M_2 \Leftrightarrow rg(M_1) = rg(M_2)$
	\item $\epsilon_1 = \mu_1,...,\epsilon_m = \mu_m$ (insbes. $l=m$) %TODO: inhaltlich prüfen$
\end{itemize}
\begin{remark}
Sei R komm. Ring. Dann heißen $a,b \in R$ \textbf{assoziiert}, in Zeichen $a \sim b$, falls es $u \in R^\times$ gibt mit $a=ub$.
\end{remark}

\section{Invariantenteiler}
\begin{definition}
Die Elemente $\pi_1^{e_{1,1}},...,\pi_1^{e_{s,1}},\pi_2^{e_{1,2}},...,\pi_2^{e_{s,2}},...,\pi_t^{e_{1,t}},...,\pi_t^{e_{s,t}}$ mit 
\begin{align*}
\begin{array}{ll}
e_{1,1} \geq e_{2,1} \geq ... \geq e_{s,1} \geq & 0 \\ 
~~\vdots & \\ 
e_{1,t} \geq e_{2,t} \geq ... \geq e_{s,t} \geq & 0
\end{array} 
\end{align*}
heißen die Invariantenteiler von M.
\end{definition}
\section{Äquivalent und Ähnlich}
\begin{definition}
Sei R ein komm. Ring.
\begin{enumerate}
	\item Seien $A,B \in K^{n \times m}$. Dann heißen $A$ und $B$ äquivalent, falls es $P \in Gl_n(R), Q \in Gl_m(R)$ gibt mit $B=P^{-1}AQ$. In Zeichen $A \sim B$.
	\item Seien $A,B \in M_n(R) = K^{n \times n}$. Dann heißen $A$ und $B$ ähnlich oder konjugiert, falls es $S \in Gl_n(R)$ gibt mit $B=S^{-1}AS$. In Zeichen $A \approx B$.
\end{enumerate}
\end{definition}

\section{Begleitmatrix}
\begin{definition}
Sei $g(x) = x^n + a_{n-1}x^{n-1}+\ldots+a_1x + a_0 \in K[x]$ mit $n \geq 1$. Dann heißt die folgende Matrix Begleitmatrix zu g:
\begin{align*}
B_j =
\begin{pmatrix}
0 &        &		&  & -a_0 \\
1 & \ddots &		&  & -a_1 \\
  & \ddots & \ddots	&  & \vdots \\
  &        & \ddots	& 0 & \vdots \\
  &        &		& 1 & -a_{n-1}
\end{pmatrix}
\in M_n(K)
\end{align*} 
\end{definition}
\begin{remark}
\begin{enumerate}
	\item $\chi_{B_g}(x) = g(x)$
	\item $\mu_{B_g}(x) \sim 
	\begin{pmatrix}
		1 &        & & \\
		  & \ddots & & \\
		  &        &1& \\
		  &        & & g
	\end{pmatrix}$
\end{enumerate}
\end{remark}
\section{Jordanmatrix}
\begin{definition}
Sei $K$ bel. und $h(x)=(x-\alpha)^e, e \geq 1$. Dann heißt $J(\alpha,e)$ Jordanmatrix:
\begin{align*}
B_n \approx J(\alpha,e) :=
\begin{pmatrix}
\alpha &        &   &\\
1      & \ddots &   &\\
       & \ddots & \ddots &\\
       &        & 1 & \alpha
\end{pmatrix}
\end{align*}
\end{definition}
\section{Normalformen}
\subsection{Frobeniussche Normalform}
Sei $A \in M_n(K)$. Dann ist $A$ zu genau einer Matrix $B_{g_1,...,g_r}$ ähnlich ($\approx$) mit Polynomen $g_1(x) | g_2(x)| ... | g_r(x)$. Die $g_i$ sind dabei die Elementarteiler von $M_A(x)$.

\subsection{Weierstraß'sche Normalform}
Sei $A \in M_n(K)$. Dann gibt es ein bis auf Reihenfolge eindeutig bestimmtes System von Potenzen $h_1,...,h_m$ von normierten, irreduziblen Polynomen, sodass $A$ zur folgenden Matrix ähnlich ist.
\begin{align*}
\begin{pmatrix}
B_{n_1} &        & \\
        & \ddots & \\
        &		  & B_{n_m}
\end{pmatrix}
\end{align*}
Die Polynome $h_1,...,h_m$ sind genau die Invariantenteiler von $M_A(x)$

\subsection{Jordan Normalform}
Sei $A \in M_n(K)$ und $\chi_A$ zerfalle vollständig in Linearfaktoren. Dann gibt es bis auf Reihenfolge ein eindeutig bestimmtes System von Jordanmatrizen $J_1,...,J_m$:
\begin{align*}
A \simeq 
\begin{pmatrix}
J_1 & & 0\\
    & \ddots & \\
0   &        & J_m
\end{pmatrix}
\end{align*}
\begin{remark}
Aus der JNF lässt sich das Minimalpolynom $\mu_A$ direkt ablesen. Sortiere Jordankästchen nach EW und nach Größe der Jordankästchen.
\begin{align*}
\begin{pmatrix}
J(\alpha_1,e_{1,1}) & & & & & & 0\\
& \ddots & & & & & \\
 & & J(\alpha_1,e_{1,n_1}) & & & & \\
 & & & \ddots & & & \\
 & & & & J(\alpha_s,e_{s,1}) & &\\
 & & & & & \ddots & \\
0 & & & & & & J(\alpha_s,e_{s,n_s}) \\
\end{pmatrix}
\end{align*}
mit $e_{i,1} \leq ... \leq e_{i,n_i}$. Dann gilt $\mu_A(x)=\prod \limits_{i=1}^s (x-\alpha_i)^{e_{i,n_i}}$. TFE:
\begin{enumerate}
	\item A diagonalisierbar
	\item Die JNF ist eine Diagonalmatrix
	\item Jedes Jordankästchen hat die Größe 1
\end{enumerate}
\end{remark}
\section{Inneres Produkt (Skalarprodukt)}
\begin{definition}
Sei V ein K-VR. Ein inneres Produkt (oder Skalarprodukt) ist eine Abbildung $\phi$ von $V \times V$ nach $K, \phi (v_1,v_2) = (v_1,v_2) \in K,~v_1,v_2 \in V$ mit folgenden Eigenschaften:
\begin{enumerate}
	\item $(v_1+v_2,v_3) = (v_1,v_3)+(v_2,v_3)$
	\item $(\alpha v_1,v_2) = \alpha (v_1,v_2)$
	\item $(v_1,v_2) = \overline{(v_2,v_1)}$ (hermitesch)
	\item $(v,v) \geq 0$, $(v,v) = 0 \Leftrightarrow v=0$ (positiv definit)
\end{enumerate}
\end{definition}
\begin{remark}
V zusammen mit einem inneren Produkt nennt man einen \textbf{euklidischen Raum} (falls $K=\mathbb{R}$) bzw. \textbf{unitären Raum} (falls $K=\mathbb{C}$).
\end{remark}
\textbf{Lemma}:
Sei $A:V \rightarrow W$ linear und injektiv. Sei $(~,~):W \times W \rightarrow K$ ein inneres Produkt auf $W$. Dann wird durch $p_A(v_1,v_2):=(Av_1,Av_2),~v_1,v_2 \in V$ ein inneres Produkt definiert.

\section{Cauchy-Schwarz-Ungleichung}
\begin{align*}
|(v_1,v_2)| \leq \norm{v_1} \norm{v_2}
\end{align*}

\section{Gram-Schmidt-Verfahren}
Seien $w_1,...,w_n$ linear unabhängige Vektoren. Die Vektoren $v_1,...,v_n$ des Orthogonalsystems werden rekursiv berechnet:
\begin{align*}
v_1 &= w_1\\
v_2 &= w_2 - \frac{(v_1,w_2)}{(v_1,v_1)}v_1\\
v_3 &= w_3 - \frac{(v_1,w_3)}{(v_1,v_1)}v_1 - \frac{(v_2,w_3)}{(v_2,v_2)}v_2\\
v_n &= w_n - \sum_{i=1}^{n-1}\frac{(v_i,w_n)}{v_i,v_i}v_i
\end{align*}

\section{Lotraum}
Sei $M \subseteq V$ eine Menge. $M^\perp = \{v\in V | (v,m)=0~\forall m \in M\}$. $M^\perp \subseteq V$ ist ein Unterrraum.

Sei $U \subseteq V$ ein Unterraum mit $\dim(U) = n < \infty$. Dann gilt:
\begin{enumerate}
	\item $U \oplus U^\perp = V$
	\item $(U^\perp)^\perp = U$
\end{enumerate}