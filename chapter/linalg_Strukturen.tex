\section{Strukturen}
%%%%%%%%%%%%%%%%%%%%%%%%%%%%%%%%%%%%%%%%%%%%%
\subsection{Gruppe\index{Gruppe}}
\begin{definition}
Eine Menge G, zusammen mit einer Verknüpfung $\cdot$ ist eine \textbf{Gruppe}\index{Gruppe}, falls gilt:
\begin{enumerate}
\item $\cdot$ ist assoziativ: a$\cdot$(b$\cdot$c)=(a$\cdot$b)$\cdot$c für alle a,b,c $\in$ G
\item es gibt ein (links)-neutrales Element e $\in$ G mit e $\cdot$ a = a für alle a $\in$ G 
\item zu jedem a $\in$ G gibt es ein (links)-inverses Element, d.h. ein b $\in$ G mit b $\cdot$ a = e
\end{enumerate}
\hspace*{3mm} Die Gruppe G ist kommutativ oder \textbf{abelsch}\index{abelsch}, falls zusätzlich gilt:
\begin{enumerate}
\setcounter{enumi}{3}
\item a $\cdot$ b = b $\cdot$ a für a, b $\in$ G
\end{enumerate}
\end{definition}

\begin{remark}
\leavevmode
\begin{enumerate}
\item Das neutrale Element ist eindeutig bestimmt. Ebenso ist zu jedem a $\in$ G das zugehörige inverse Element eindeutig bestimmt.
\item Ist (G, $\cdot$) eine Gruppe, so schreibe e = 1 (Einselement) und b = a$^{-1}$ für das zu a inverse Element. Sind $a_1, a_2, . . . , a_n \in G$, so schreibe $\prod\nolimits_{i=1}^{n} a_i = a_1 \cdots a_n$; nach Definition gilt $\prod\nolimits_{i=1}^{0}a_i = 1$.
\item Ist (G,+) eine abelsche Gruppe, so setze e = 0 (Nullelement) und bezeichne das zu a inverse Element mit -a. In diesem Fall bezeichnet $\sum\nolimits_{i=1}^{n} a_i$ die Summe der endlich vielen Elemente $a_1, . . . , a_n$; nach Definition $\sum\nolimits_{i=1}^{0} a_i = 0$.
\end{enumerate}
\end{remark}

\begin{definition}
Sei G eine Gruppe. Eine Teilmenge H $\subseteq$ G ist eine \textbf{Untergruppe}\index{Untergruppe} von G, H $\le$ G, falls gilt:
\begin{enumerate}
\item 1 $\in$ H
\item a,b $\in$ H $\Rightarrow$ ab $\in$ H
\item a $\in$ H $\Rightarrow$ a$^{-1}$ $\in$ H
\end{enumerate}
\end{definition}

\begin{remark} 
Ist $\emptyset \neq H \subseteq G$ eine nichtleere Teilmenge, so lassen sich die Kriterien (a)-(c) der obigen Definition zu einer Bedingung vereinfachen: 
$\emptyset \neq H \subseteq G$ ist genau dann eine Untergruppe, falls gilt: a,b $\in$ H $\Rightarrow$ ab$^{-1}$ $\in$ H.
\end{remark}

\begin{definition}
Sei G eine Gruppe, U $\le$ G eine Untergruppe, und $\sim$ die durch U definierte Äquivalenzrelation auf G (a $\sim$ b $\Leftrightarrow$ ab$^{-1}$ $\in$ U). Ist a $\in$ G, so ist die entsprechende Äquivalenzklasse die Menge 
$[a]=\{ b \in G | a \sim b \} = \{ b \in G | ab^{-1} \in U \} = \{ b \in G | b = Ua \} = Ua$; 
diese Mengen sind die \textbf{Rechtsnebenklassen}\index{Rechtsnebenklasse} von U. Sind die $Ua_j$ für j $\in$ J, die verschiedenen Rechtsnebenklassen, so bilden diese eine Partition G=$\bigcup\limits_{j \in J} Ua_j$.
\end{definition}

Ist |J| endlich, so nennt man |J| den Index von U in G und schreibt |J| = |G : U|.
\begin{enumerate}
\item Genauso definiert a $\sim$ b $\Leftrightarrow$ a$^{-1}$b $\in$ eine Äquivalenzrelation auf G. Die Äquivalenzklasse von a $\in$ G ist die Linksnebenklasse $[a] = \{b \in G | a \sim b \} = \{ b \in G | a^{-1}b \in U \} = aU$. 
Ist G abelsch, so gilt aU = Ua; für eine nicht-abelsche Gruppe gilt dies im allgemeinen nicht.
\item Der Versuch analog zur Definition der Addition auf $\mathbb{Z}/m\mathbb{Z}$ mittels der Addition auf $\mathbb{Z}$ eine Verknüpfung auf der Menge der Nebenklassen $G/U = \{Ua | a \in G\}$ durch Ua $\cdot$ Ub = Uab zu definieren funktioniert für abelsche Gruppen, aber nicht für allgemeine Gruppen.
\end{enumerate}

\begin{definition}
Seien G und H (multiplikativ geschriebene) Gruppen.
\begin{compactenum}
\item Ein \textbf{Homomorphismus}\index{Homomorphismus} (oder \textbf{Gruppenhomomorphismus}\index{Gruppenhomomorphismus}) ist eine Abbildung f: G $\to$ H, die mit den Gruppenstrukturen verträglich ist, d.h. für $g_1, g_2 \in G$ gilt:\\
$f(g_1g_2) = f(g_1) f(g_2)$.
\item Ist f: G $\to$ H ein Homomorphismus, so setze\\
$im(f) = \{f(g) | g \in G\}$,\\
$ker(f) = \{g \in G | f(g) = 1\}$.
\end{compactenum}
\end{definition}

\begin{definition}
Sei G eine Gruppe. Eine Untergruppe U $\le$ G ist ein \textbf{Normalteiler}\index{Normalteiler} (oder eine normale Untergruppe), U $\vartriangleleft$ G, falls gilt
\begin{center}
$u \in U, g \in G \Rightarrow g^{-1}ug \in U$.
\end{center}
Ist U < G (d.h. U $\neq$ G), so schreibe U $\vartriangleleft$ G.\\
Ist G abelsch, so folgt aus $g^{-1}ug = g^{-1}gu = 1u = u \in U$, dass jede Untergruppe ein Normalteiler ist.
\end{definition}

\begin{lemma}
Sei f: G $\to$ H ein Homomorphismus. Dann gilt:
\begin{enumerate}
\item im(f) $\subseteq$ H ist eine Untergruppe
\item ker(f) $\subseteq$ G ist ein Normalteiler
\end{enumerate}
\end{lemma}

\begin{lemma}
Sei N $\vartriangleleft$ G ein Normalteiler und G/N = \{gN | g $\in$ G\}.
\begin{enumerate}
\item Die Menge G/N ist mittels der Verknüpfung $g_1N \cdot g_2N = g_1g_2N, g_1,g_2 \in G$ eine Gruppe mit neutralem Element N
\item Die Abbildung $\pi$: G $\to$ G/N, g $\mapsto$ gN, ist ein Epimorphismus mit ker($\pi$) = N
\end{enumerate}
\end{lemma}

\begin{theorem}
(\textbf{Homomorphiesatz} für Gruppen\index{Homomorphiesatz für Gruppen})
Seien G, H Gruppen und f: G $\to$ H ein Homomorphismus. Dann gibt es einen Epimorphismus $\pi$: G $\to$ G/ ker(f) und einen Monomorphismus h: G/ ker(f) $\to$ H mit f = h $\circ$ $\pi$ und im(f) = im(h).
\end{theorem}

Die symmetrische Gruppe S$_n$ ist die Gruppe der bijektiven Abbildungen der Menge \{1, …, n\}. Die Gruppenoperation auf S$_n$ ist die Verknüpfung von Abbildungen, S$_n$ ist eine endliche Gruppe mit |S$_n$| = n!. Für n $\ge$ 3 ist S$_n$ nicht abelsch.

\begin{lemma}
\leavevmode
\begin{compactenum}
\item Jede Permutation $\tau \in S_n$ hat eine Darstellung als ein Produkt von disjunkten Zyklen (nicht eindeutig).
\item Es gilt $(a_1, a_2, …, a_k) = (a_1, a_k)(a_1, a_{k-1}) \cdots (a_1, a_2)$; insbesondere lässt sich jede Permutation als ein Produkt von Transpositionen schreiben.
\end{compactenum}
\end{lemma}

\begin{theorem}
Sei n > 1 und sei \{-1, 1\} die multiplikative Gruppe.
\begin{compactenum}
\item Es gibt einen Epimorphismus $sgn: S_n \to \{-1,+1\}$ mit $sgn(\tau) = 1$ für alle Transpositonen $\tau \in S_n$.
\item Sei K ein Körper und $f: S_n \to K^\times$ ein Homomorphismus. Dann ist entweder f($\tau$) = 1 für alle $\tau \in S_n$, oder es ist char(k) $\neq$ 2 und f = sgn.
\end{compactenum}
\end{theorem}

\begin{remark}
\begin{itemize}
\item Ist $\pi \in S_n$ und $\pi = \tau_1 \cdots \tau_k$ eine Zerlegung in Transpositionen, so ist nach 1) sgn ein Homomorphismus mit sgn($\pi$) = -1, also ist sgn($\pi$) = (-1)$^k$.

\item Die Zerlegung von $\pi$ in Transpositionen ist nicht eindeutig, aber für jede solche Zerlegung gilt, dass die Parität (gerade oder ungerade) der Anzahl der Faktoren eindeutig ist.\\
Ist $\pi \in S_n$ und $\pi = \zeta_1 \cdots \zeta_l$ eine Zerlegung in disjunkte Zyklen der Länge $k_i, i = 1, …, l$. Es sei m die Anzahl der bewegten Ziffern, also $m = \sum\nolimits_{i=1}^{l} k_i$. Dann folgt $sgn(\pi) = (-1)^{m-l}$.
\end{itemize}
\end{remark}

\begin{definition}
Für n $\ge$ 2 ist $A_n := ker\{sgn : S_n \to \{-1,1\}\}$ die alternierende Gruppe auf n Ziffern.
\begin{enumerate}
\item Es ist $A_n \vartriangleleft S_n$ und $|S_n : A_n| = 2$.
\item Für jedes $\pi \in S_n$ mit sgn($\pi$)= -1 ist $S_n = A_n \cup \pi A_n = A_n \cup A_n \pi$, wobei die Vereinigung jeweils disjunkt ist.
\end{enumerate}
\end{definition}

\begin{example}
Sei U < S$_n$ eine Untergruppe mit |S$_n$ : U| = 2. Dann ist U = A$_n$.
\end{example}

\begin{remark}
Für n = 3 und n $\ge$ 5 sind \{1\}, A$_n$ und S$_n$ die einzigen Normalteiler von S$_n$, und A$_n$ besitzt nur die trivialen Normalteiler \{1\} und A$_n$ (man sagt A$_n$ ist eine einfache Gruppe). Für n = 4 ist A$_4$ nicht-einfach.
\end{remark}

%%%%%%%%%%%%%%%%%%%%%%%%%%%%%%%%%%%%%%%%%%%%%%
\subsection{Ring\index{Ring}}
%Definition fehlt!
\begin{definition}
\leavevmode
\begin{enumerate}
	\item R ist ein \textbf{euklidischer}\index{euklidischer Ring} Ring, falls es eine Funktion $\phi : R\backslash \{0\} \rightarrow \mathbb{N}_0$ gibt, sodass gilt:
	$\forall a,b \in R, b \neq 0$, gibt es $q,r \in R$ mit $a=qb+r$, $r=0$ oder $\phi(r) < \phi (b)$
	\item \textbf{Integritätsbereich}\index{Integritätsbereich}: R ist kommutativ und Nullteilerfrei
	\item R heißt \textbf{faktoriell}\index{faktorieller Ring}, falls jedes $0 \neq x \in R$ eine eindeutige Primzerlegung hat
\end{enumerate}
\end{definition}

\begin{remark}
\leavevmode
\begin{itemize}
	\item Euklidisch $\Rightarrow$ HIR $\Rightarrow$ faktoriell
	\item $\mathbb{Z}$ mit $\phi(a) = |a|$ ist euklidisch
	\item R euklidisch $\Rightarrow$ R ist Hauptidealring
	\item Sei R kommutativ. Dann gilt $R^n \simeq R^m \Leftrightarrow n=m$
\end{itemize}
\end{remark}

\subsubsection{Hauptidealring\index{Hauptidealring}}
%Todo: Hauptidealring$

\subsubsection{Ideal\index{Ideal}}
\begin{definition}
Sei R ein Ring. Eine Teilmenge $I \subseteq R, I \neq \emptyset$ heißt \textbf{Ideal}\index{Ideal}, falls
\begin{enumerate}
	\item $a_1,a_2 \in I \Rightarrow a_1+a_2 \in I$
	\item $a \in I \Rightarrow r_1 a r_2 \in I, \forall r_1,r_2 \in R$
\end{enumerate}
\end{definition}

\subsubsection{Integritätsbereich\index{Integritätsbereich}}
%Todo

\begin{remark}
\leavevmode
\begin{itemize}
	\item ker(f) ist stets Ideal
	\item falls $1 \in I \Rightarrow I = R$
	\item seien $I_1,I_2 \subseteq R$ Ideale, dann sind auch $I_1 \cap I_2, I_1+I_2, I_1 \cdot I_2$ wieder Ideale
	\item In $\mathbb{Z}$ sind alle Ideale von der Form $I=a\mathbb{Z}, a \in \mathbb{Z}$
	\item falls $I_1+I_2 = R$, so nennt man $I_1$ und $I_2$ teilerfremd.
	\item $I_1+I_2 = R \Rightarrow I_1 \cap I_2 = I_1 \cdot I_2$
	\item R HIR: $(p)$ ist maximal $ \Leftrightarrow $ $p$ ist irreduzibel $\Leftrightarrow$ $p$ ist Primideal
	\item Sei R komm. Ring. Dann gilt:
	\begin{enumerate}
		\item P Primideal $\Leftrightarrow$ $\QR{R}{P}$ Integritätsbereich
		\item M max. Ideal $\Leftrightarrow$ $\QR{R}{M}$ ist Körper
	\end{enumerate}
	\item Sei R Ring mit 1. Sei $I \subsetneq R $ Ideal. Dann gibt es ein maximales Ideal M mit $ I \subseteq M$. Insbesondere existieren max. Ideale.
\end{itemize}
\end{remark}

\subsubsection{Polynomring\index{Polynomring}}
\begin{definition}
Sei R ein Ring. Der Polynomring R[x] über R ist R[x] = \{($a_0,a_1,...) | a_j \in R$, nur endlich viele $a_j \neq 0$\},
mit Addition und Multiplikation definiert durch $(a_j) + (b_j) = (a_j + b_j)$ und $(a_j)(b_j) = (c_j)$ mit $c_k = \sum\nolimits_{j=0}^{k} a_j b_{k-j}$.
\end{definition}

\subsubsection{Ringhomomorphismus\index{Ringhomomorphismus}}
\begin{definition}
Seien R, S Ringe.
\begin{enumerate}
\item Eine Abbildung f: R $\to$ S ist ein \textbf{Ringhomomorphimus}\index{Ringhomomorphimus}, falls
\begin{enumerate}
\item $f(r_1 +r_2) = f(r_1) + f(r_2), r_1,r_2 \in R$
\item $f(r_1r_2) = f(r_1)f(r_2), r_1,r_2 \in R$
\item $f(1_R) = 1_S$
\end{enumerate}
\item Ein Monomorphismus (bzw. Epimorphismus, Isomorphismus) ist ein injektiver (bzw. surjektiver, bijektiver) Ringhomomorphimus.
\end{enumerate}
\end{definition}

%%%%%%%%%%%%%%%%%%%%%%%%%%%%%%%%%%%%%%%%%%%%
\subsection{Moduln\index{Modul}}
\begin{definition}
Eine Menge $M \neq \emptyset$ ist ein R-(Links-) \textbf{Modul}\index{Modul}, falls es eine Verknüpfung $+:M \times M \rightarrow M, (m_1,m_2) \mapsto m_1+m_2$ und eine weitere Verknüpfung $\cdot:R \times M \rightarrow M, (r,m) \mapsto rm$ gibt, sodass:
\begin{itemize}
	\item $(M,+)$ ist abelsche Gruppe
	\item $(r_1,r_2)m = r_1 m + r_2 m$\\$r(m_1+m_2) = rm_1+rm_2$
	\item $1_R m = m$
\end{itemize}
\end{definition}
\begin{remark}
Sei R HIR
\begin{itemize}
	\item Sei $ M \subseteq F = R^n$ ein R-Untermodul. Dann ist $M \simeq R^k$ mit $k \leq n$
	\item Sei $M=<m_1,...,m_n>$ e-e. Modul. Dann gilt: M frei $\Leftrightarrow$ M torsionsfrei
	\item Ist M e-e. R-Modul, so ist $M \simeq T(M) \oplus F, F \simeq R^k$
\end{itemize}
\end{remark}

\subsubsection{Untermoduln\index{Untermodul}}
\begin{definition}
Sei M ein R-Modul. Eine Teilmenge $N \subseteq M$ heißt Untermodul, falls:
\begin{enumerate}
	\item $(N,+)$ ist Untergruppe
	\item $rn \in N, \forall r \in R, n\in N$
\end{enumerate}
\end{definition}

\begin{remark}
\leavevmode
\begin{itemize}
	\item Es gilt der Hom. Satz, insbesondere $\QR{M}{\ker(f)} \simeq \im(f)$
	\item $\ker(f),\im(f)$ sind Untermoduln.
\end{itemize}
\end{remark}

\subsubsection{Freier Modul\index{freier Modul}}
\begin{theorem}
\leavevmode
\begin{enumerate}
	\item Sei $F = \bigoplus \limits_{i \in I} R e_i $ (F ist freier Modul und die $e_i$ sind eine \enquote{Basis} von F, d.h. $r e_i = 0 \Leftrightarrow r=0$).
	Für $i\in I$ sei ein Element $m_i \in M$ gegeben, wobei M ein R-Modul ist.
	Dann gibt es genau einen Modulhom. $f:F \rightarrow M $ mit $f(e_i) = m_i$
	\item Sei $f:M \twoheadrightarrow F$ ein surjektiver Modulhom., wobei $F = \bigoplus \limits_{i \in I}R e_i$ frei ist. Dann gibt es einen Teilmodul $N \subseteq M$ mit
	\begin{itemize}
		\item $M = N \oplus \ker(f)$
		\item $N \simeq F$
	\end{itemize}
\end{enumerate}
\end{theorem}

\subsubsection{Torsionselement\index{Torsionselement}}
\begin{definition}
Sei R komm. Integritätsbereich. Sei M ein R-Modul. Ein Element $m\in M$ heißt Torsionselement, falls es ein $r\in R \backslash \{0\}$ gibt mit $rm=0$.

Sei $T(M) := \{m \in M | m~\text{ist torsion} \} \ni 0$. $M$ heißt Torsionsfrei, falls $T(M) = 0$.
\end{definition}

\begin{remark}
\leavevmode
\begin{itemize}
	\item $T(M) \subseteq M$ ist Teilmodul
	\item $\QR{M}{T(M)}$ ist torsionsfrei
\end{itemize}
\end{remark}

\subsubsection{$\pi$-Torsionselement\index{$\pi$-Torsionselement}}
\begin{definition}
Sei $\pi \in R$ irreduzibel und M ein R-Modul.\\
Dann heißt $T_\pi (M):= \{m \in M | \exists e \in \mathbb{N}: \pi^em = 0 \} \subseteq T(M)$ $\pi$-Torsionsteilmodul von M.
%TODO: inhaltlich prüfen$
\end{definition}

\subsubsection{R-Modulhomomorphismus\index{R-Modulhomomorphismus}}
\begin{definition}
Sei R ein Ring, M,N zwei R-Moduln. Dann ist $f:M\mapsto N$ ein R-Modulhom., falls:
\begin{itemize}
	\item $f$ ist ein Gruppenhom., d.h. $f(m_1+m_2) = f(m_1)+f(m_2)$
	\item $f(rm) = rf(m), \forall r \in R, n\in \mathbb{N}$
\end{itemize}
\end{definition}

\subsubsection{K[x]-Modul\index{K[x]-Modul}}
\begin{theorem}
Sei $A \in End_K(V)$. Dann wird V zu einem K[x]-Modul vermöge\footnote{vermöge = "kann das"} $f(x)v := f(A)v, f\in K[x], v \in V.~(V_A=V)$\\
Seien $A,B \in M_n(K), V=K^n$. TFE\footnote{TFE = The following are equivalent}:
\begin{enumerate}
	\item $A \approx B$
	\item $V_A \simeq V_B$ als K[x]-Modul
	\item $\QR{K[x]^n}{<M_A(x)>} \simeq \QR{K[x]^n}{<M_B(x)>}$ als K[x]-Moduln
	\item $M_A(x) \sim M_B(x)$
	\item $M_A(x)$ und $M_B(x)$ haben die gleichen Elementarteiler
	\item $M_A(x)$ und $M_B(x)$ haben die gleichen Invariantenteiler
\end{enumerate}
\end{theorem}

%%%%%%%%%%%%%%%%%%%%%%%%%%%%%%%%%%%%%%%%%%%%%
\subsection{Körper\index{Körper}}
\begin{definition}
Ein \textbf{Körper}\index{Körper} K ist eine Menge mit zwei Verknüpfungen + und $\cdot$, für die gilt:
\begin{enumerate}
\item (K, +) ist eine abelsche Gruppe mit Nullelement 0,
\item (K \textbackslash \{0\}, $\cdot$) ist eine abelsche Gruppen mit Einselement 1$\neq$ 0,
\item a$\cdot$(b+c)=a$\cdot$b + a$\cdot$c und (a+b)$\cdot$c = a$\cdot$c + a$\cdot$b.
\end{enumerate}
\end{definition}

In Körpern gelten viele der ‘üblichen’ \textbf{Rechenregeln}. Für a, b $\in$ K ist:
\begin{enumerate} 
\item 0a = a0 = 0 
\item (-1)a = -a
\item (-a)b = a(-b) = -ab
\item ab = 0 $\Rightarrow$ a = 0 oder b = 0
\end{enumerate}

\begin{example}
\leavevmode
\begin{enumerate}
\item $\mathbb{Q}$ und $\mathbb{R}$ sind Körper.
\item Sei p eine Primzahl und $\mathbb{Z}/p\mathbb{Z}$ die Menge der Restklassen modulo p. Dann bildet $\mathbb{Z}/p\mathbb{Z}$ \textbackslash\{[0]\} bzgl. der evidenten Multiplikation [a]$\cdot$[b] = [ab] eine abelsche Gruppe, d.h. $\mathbb{Z}/p\mathbb{Z}$ ist ein Körper mit p Elementen. In $\mathbb{Z}/p\mathbb{Z}$ gilt pa = 0 für alle a $\in$ $\mathbb{Z}/p\mathbb{Z}$.
\end{enumerate}
\end{example}

\begin{definition}
Sei K ein Körper. Ein \textbf{Teilkörper}\index{Teilkörper}  L $\subseteq$ K ist eine Teilmenge, so dass gilt:
\begin{enumerate}
\item a,b $\in$ L $\Rightarrow$ a+b, a$\cdot$b $\in$ L
\item 0, 1 $\in$ L
\item a $\in$ L $\Rightarrow$ -a $\in$ L
\item 0 $\neq$ a $\in$ L $\Rightarrow$ a$^{-1}$ $\in$ L
\end{enumerate}
\end{definition}

%%%%%%%%%%%%%%%%%%%%%%%%%%%%%%%%%%%%%%%%%%%%%
\subsection{Vektorraum\index{Vektorraum}}
\begin{definition}
Sei K ein Körper. Ein \textbf{K-Vektorraum}\index{K-Vektorraum} ist eine Menge V , zusammen mit einer Verknüpfung V $\times$ V $\to$ V, (a, b) $\mapsto$ a + b und einer Verknüpfung K $\times$ V $\to$ V, ($\alpha$, a) $\mapsto$ $\alpha$ $\cdot$ a (einer ‘Skalarmultiplikation’), so dass gilt:
\begin{enumerate}
\item V ist bzgl. + eine abelsche Gruppe,
\item ($\alpha$ + $\beta$) $\cdot$ a = $\alpha$ $\cdot$ a + $\beta$ $\cdot$ a und $\alpha$ $\cdot$ (a + b) = $\alpha$ $\cdot$ a + $\alpha$ $\cdot$ b,
\item ($\alpha$ $\cdot$ $\beta$) $\cdot$ a = $\alpha$ $\cdot$ ($\beta$ $\cdot$ a) für $\alpha$, $\beta$ $\in$ K und a $\in$ V,
\item 1 $\cdot$ a = a für 1 $\in$ K und a $\in$ V.
\end{enumerate}
\end{definition}

Für K-Vektorräume gelten die folgenden \textbf{Rechenregeln}:
\begin{enumerate}
\item $\alpha \cdot 0_V = 0_V$ für alle $\alpha$ $\in$ K,
\item $0_K \cdot a = 0_V$  für alle a $\in$ V,
\item $(-\alpha) \cdot a = \alpha \cdot (-a) = \alpha(-a)$ für $\alpha$ $\in$ K und a $\in$ V,
\item $\alpha \cdot a = 0$ für $\alpha$ $\in$ K und a $\in$ V impliziert $\alpha = 0_K$ oder $a = 0_V$,
\item $\alpha \cdot (\sum\nolimits_{i=1}^{n} a_i) = \sum\nolimits_{i=0}^{n} \alpha a_i$ und $(\sum\nolimits_{i=0}^{n}\alpha_i) \cdot a = \sum\nolimits_{i=0}^{n} \alpha_i a$,
\item $\sum\nolimits_{i=0}^{n} \alpha_i a_i + \sum\nolimits_{i=0}^{n} \beta_i a_i = \sum\nolimits_{i=0}^{n} (\alpha_i + \beta_i) a_i$
\end{enumerate}

\begin{definition}
Sei V ein K-Vektorraum. Eine Teilmenge U $\subseteq$ V ist ein \textbf{K-Untervektorraum}\index{K-Untervektorraum} oder \textbf{K-linearer Unterraum} von V, falls gilt:
\begin{enumerate}
\item $\emptyset$ $\neq$ U,
\item a, b $\in$ U $\Rightarrow$ a + b $\in$ U,
\item $\alpha$ $\in$ K, a $\in$ U $\Rightarrow$ $\alpha$ $\cdot$ a $\in$ U (insbesondere: a $\in$ U $\Rightarrow$ -a $\in$ U ).
\end{enumerate}
\end{definition}

\begin{lemma}
\leavevmode
\begin{itemize}
\item Sei V ein K -Vektorraum und sei $(U_i )_{i \in I}$ eine Familie von linearen Unterräumen von V. Dann ist U =$\cap_i U_i \subseteq V$ ebenfalls ein linearer Unterraum.
\item Sei V ein K-Vektorraum und A $\subseteq$ V eine Teilmenge. Dann ist die Menge $\langle A \rangle := \big\{ \sum\nolimits_{i=0}^{n} \alpha_i a_i | n \in \mathbb{N}_0, \alpha_i \in K, a_i \in A \big\} \subseteq V$ ein linearer Unterraum (der von A erzeugte lineare Unterraum). Weiter ist $\langle A \rangle = \cap U$, wobei der Schnitt über alle linearen Unterräume U von V mit A $\subseteq$ U zu erstrecken ist. Also ist $\langle A \rangle$ der kleinste lineare Unterraum, der die Teilmenge A enthält.
\end{itemize}
\end{lemma}

\begin{definition}
Eine Menge $A = \{a_i\}_{i \in I} \subseteq V$ von Elementen eines K-Vektorraums V ist ein \textbf{Erzeugendensystem}\index{Erzeugendensystem} von V, falls $\langle A \rangle$ = V gilt, d.h. falls jeder Vektor a $\in$ V eine Darstellung als endliche Summe
\begin{center}
$a = \sum\nolimits_{i=1}^{n}\alpha_i a_i, \alpha_i \in K, a_i \in A$
\end{center}
besitzt. Der Vektorraum K ist endlich erzeugt (über K), falls V eine endliches Erzeugendensystem A = $\{a_1, …, a_n\}$ besitzt.
\end{definition}

\begin{remark} 
Ist $(a_i)_{i \in I}$ eine Familie von Elementen von V , so definiert man analog den von den Elementen $a_i$ erzeugten linearen Unterraum $\langle a_i | i \in I \rangle \subseteq V$ als den von der Menge A = $\{a_i | i \in I\}$ erzeugten Unterraum. Klar ist 
damit:
\begin{enumerate}
\item $\langle \emptyset \rangle = \{0\}$,
\item $A \subseteq \langle A \rangle$ für jede Teilmenge A $\subseteq$ V, 
\item $U = \langle U \rangle$ für jeden linearen Unterraum U $\subseteq$ V, 
\item Sind A, B $\subseteq$ V Teilmengen, so gilt\\ 
A $\subseteq$ B $\Rightarrow$ $\langle A \rangle \subset \langle B \rangle$, \hspace*{3mm}
A $\subseteq$ $\langle B \rangle$ $\Rightarrow$ $\langle A \rangle \subseteq \langle B \rangle$.
\end{enumerate} 
\end{remark}

\begin{definition}
Sei V ein K -Vektorraum und seien $\{a_i \}_{i \in I}$ Vektoren in V . Die Menge $\{a_i\}_{i \in I}$ ist \textbf{linear unabhängig}\index{linear unabhängig} , falls für jede endliche Teilmenge J $\subseteq$ I gilt:
\begin{center}
$\sum\nolimits_{j \in J} \alpha_j a_j = 0 \Rightarrow \alpha_j = 0$ für alle j $\in$ J.
\end{center}
Sind die $a_i$ nicht linear unabhängig, so sind sie linear abhängig.
\end{definition}

\begin{theorem}
\textbf{Basisergänzungssatz}\index{Basisergänzungssatz} : Sei V ein endlich erzeugter K-Vektorraum, V = $\langle A \rangle$ mit A = $\{a_1,... ,a_n\}$. Sei C $\subseteq$ A eine linear unabhängige Menge von Vektoren. Dann gibt es eine Basis B von V mit C $\subseteq$ B $\subseteq$ A. Insbesondere besitzt jeder endlich erzeugte Vektorraum V eine Basis.
\end{theorem}

\begin{lemma}
\textbf{Austauschlemma}\index{Austauschlemma}: Sei V ein K-Vektorraum und sei B = $\{b_1,... ,b_n\} \subseteq V$ eine Basis von V. Ist $b= \sum\nolimits_{i=1}^{n} \alpha_i b_i$ mit $\alpha_i \in K$ und $\alpha_1 \neq 0$, so ist auch B' = $\{ b, b_1, …, b_n \} \subseteq V$ eine Basis.
\end{lemma}

\begin{theorem}
\textbf{Austauschsatz von Steinitz}\index{Austauschsatz von Steinitz}: Sei V ein K-Vektorraum und $\{b_1,... ,b_n\} \subseteq$ V eine Basis von V. Ist $\{a_1,... ,a_m\} \subseteq$ V eine linear unabhängige Teilmenge, so ist m $\le$ n und bei geeigneter Nummerierung der $b_i$ ist $\{a_1,... ,a_m, b_{m+1},... ,b_n\}$ ebenfalls eine Basis von V.
\end{theorem}

