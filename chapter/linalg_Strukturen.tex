\chapter{Strukturen}
%%%%%%%%%%%%%%%%%%%%%%%%%%%%%%%%%%%%%%%%%%%%%
\section{Gruppe}
\begin{definition}
Eine Menge G, zusammen mit einer Verknüpfung $\cdot$ ist eine \textbf{Gruppe}\index{Gruppe}, falls gilt:
\begin{enumerate}
\item $\cdot$ ist assoziativ: a$\cdot$(b$\cdot$c)=(a$\cdot$b)$\cdot$c für alle a,b,c $\in$ G
\item es gibt ein (links)-neutrales Element e $\in$ G mit e $\cdot$ a = a für alle a $\in$ G 
\item zu jedem a $\in$ G gibt es ein (links)-inverses Element, d.h. ein b $\in$ G mit b $\cdot$ a = e
\end{enumerate}
\hspace*{3mm} Die Gruppe G ist kommutativ oder \textbf{abelsch}\index{Abelsche Gruppe}, falls zusätzlich gilt:
\begin{enumerate}
\setcounter{enumi}{3}
\item a $\cdot$ b = b $\cdot$ a für a, b $\in$ G
\end{enumerate}
\end{definition}

\begin{remark}
\leavevmode
\begin{enumerate}
\item Das neutrale Element ist eindeutig bestimmt. Ebenso ist zu jedem a $\in$ G das zugehörige inverse Element eindeutig bestimmt.
\item Ist (G, $\cdot$) eine Gruppe, so schreibe e = 1 (Einselement) und b = a$^{-1}$ für das zu a inverse Element. Sind $a_1, a_2, . . . , a_n \in G$, so schreibe $\prod\nolimits_{i=1}^{n} a_i = a_1 \cdots a_n$; nach Definition gilt $\prod\nolimits_{i=1}^{0}a_i = 1$.
\item Ist (G,+) eine abelsche Gruppe, so setze e = 0 (Nullelement) und bezeichne das zu a inverse Element mit -a. In diesem Fall bezeichnet $\sum\nolimits_{i=1}^{n} a_i$ die Summe der endlich vielen Elemente $a_1, . . . , a_n$; nach Definition $\sum\nolimits_{i=1}^{0} a_i = 0$.
\end{enumerate}
\end{remark}

\begin{definition}
Sei G eine Gruppe. Eine Teilmenge H $\subseteq$ G ist eine \textbf{Untergruppe}\index{Untergruppe} von G, H $\le$ G, falls gilt:
\begin{enumerate}
\item 1 $\in$ H
\item a,b $\in$ H $\Rightarrow$ ab $\in$ H
\item a $\in$ H $\Rightarrow$ a$^{-1}$ $\in$ H
\end{enumerate}
\end{definition}

\begin{remark} 
Ist $\emptyset \neq H \subseteq G$ eine nichtleere Teilmenge, so lassen sich die Kriterien (a)-(c) der obigen Definition zu einer Bedingung vereinfachen: 
$\emptyset \neq H \subseteq G$ ist genau dann eine Untergruppe, falls gilt: a,b $\in$ H $\Rightarrow$ ab$^{-1}$ $\in$ H.
\end{remark}

\begin{definition}
Sei G eine Gruppe, U $\le$ G eine Untergruppe, und $\sim$ die durch U definierte Äquivalenzrelation auf G (a $\sim$ b $\Leftrightarrow$ ab$^{-1}$ $\in$ U). Ist a $\in$ G, so ist die entsprechende Äquivalenzklasse die Menge 
$[a]=\{ b \in G | a \sim b \} = \{ b \in G | ab^{-1} \in U \} = \{ b \in G | b = Ua \} = Ua$; 
diese Mengen sind die \textbf{Rechtsnebenklassen}\index{Rechtsnebenklasse} von U. Sind die $Ua_j$ für j $\in$ J, die verschiedenen Rechtsnebenklassen, so bilden diese eine Partition G=$\bigcup\limits_{j \in J} Ua_j$.
\end{definition}

\begin{remark}
Ist |J| endlich, so nennt man |J| den Index von U in G und schreibt |J| = |G : U|.
\begin{enumerate}
\item Genauso definiert a $\sim$ b $\Leftrightarrow$ a$^{-1}$b $\in$ eine Äquivalenzrelation auf G. Die Äquivalenzklasse von a $\in$ G ist die Linksnebenklasse $[a] = \{b \in G | a \sim b \} = \{ b \in G | a^{-1}b \in U \} = aU$. 
Ist G abelsch, so gilt aU = Ua; für eine nicht-abelsche Gruppe gilt dies im allgemeinen nicht.
\item Der Versuch analog zur Definition der Addition auf $\mathbb{Z}/m\mathbb{Z}$ mittels der Addition auf $\mathbb{Z}$ eine Verknüpfung auf der Menge der Nebenklassen $G/U = \{Ua | a \in G\}$ durch Ua $\cdot$ Ub = Uab zu definieren funktioniert für abelsche Gruppen, aber nicht für allgemeine Gruppen.
\end{enumerate}
\end{remark}

\begin{definition}
Seien G und H (multiplikativ geschriebene) Gruppen.
\begin{compactenum}
\item Ein \textbf{Homomorphismus}\index{Homomorphismus} (oder \textbf{Gruppenhomomorphismus}\index{Gruppenhomomorphismus}) ist eine Abbildung f: G $\to$ H, die mit den Gruppenstrukturen verträglich ist, d.h. für $g_1, g_2 \in G$ gilt:\\
$f(g_1g_2) = f(g_1) f(g_2)$.
\item Ist f: G $\to$ H ein Homomorphismus, so setze\\
$\im(f) = \{f(g) | g \in G\}$,\\
$\ker(f) = \{g \in G | f(g) = 1\}$.
\end{compactenum}
\end{definition}

\begin{definition}
Sei G eine Gruppe. Eine Untergruppe U $\le$ G ist ein \textbf{Normalteiler}\index{Normalteiler} (oder eine normale Untergruppe), U $\vartriangleleft$ G, falls gilt
\begin{center}
$u \in U, g \in G \Rightarrow g^{-1}ug \in U$.
\end{center}
Ist U < G (d.h. U $\neq$ G), so schreibe U $\vartriangleleft$ G.\\
Ist G abelsch, so folgt aus $g^{-1}ug = g^{-1}gu = 1u = u \in U$, dass jede Untergruppe ein Normalteiler ist.
\end{definition}

\begin{lemma}
Sei f: G $\to$ H ein Homomorphismus. Dann gilt:
\begin{enumerate}
\item im(f) $\subseteq$ H ist eine Untergruppe
\item ker(f) $\subseteq$ G ist ein Normalteiler
\end{enumerate}
\end{lemma}

\begin{lemma}
Sei N $\vartriangleleft$ G ein Normalteiler und G/N = \{gN | g $\in$ G\}.
\begin{enumerate}
\item Die Menge G/N ist mittels der Verknüpfung $g_1N \cdot g_2N = g_1g_2N, g_1,g_2 \in G$ eine Gruppe mit neutralem Element N
\item Die Abbildung $\pi$: G $\to$ G/N, g $\mapsto$ gN, ist ein Epimorphismus mit ker($\pi$) = N
\end{enumerate}
\end{lemma}

\begin{theorem}
(\textbf{Homomorphiesatz} für Gruppen\index{Hom. -satz für Gruppen})
Seien G, H Gruppen und f: G $\to$ H ein Homomorphismus. Dann gibt es einen Epimorphismus $\pi$: G $\to$ G/ ker(f) und einen Monomorphismus h: G/ ker(f) $\to$ H mit f = h $\circ$ $\pi$ und im(f) = im(h).
\end{theorem}

\begin{remark}
Die \textbf{symmetrische Gruppe}\index{Symmetrische Gruppe} S$_n$ ist die Gruppe der bijektiven Abbildungen der Menge \{1, …, n\}. Die Gruppenoperation auf S$_n$ ist die Verknüpfung von Abbildungen, S$_n$ ist eine endliche Gruppe mit |S$_n$| = n!. Für n $\ge$ 3 ist S$_n$ nicht abelsch.
\end{remark}

\begin{lemma}
\leavevmode
\begin{compactenum}
\item Jede Permutation $\tau \in S_n$ hat eine Darstellung als ein Produkt von disjunkten Zyklen (nicht eindeutig).
\item Es gilt $(a_1, a_2, …, a_k) = (a_1, a_k)(a_1, a_{k-1}) \cdots (a_1, a_2)$; insbesondere lässt sich jede Permutation als ein Produkt von Transpositionen schreiben.
\end{compactenum}
\end{lemma}

\begin{theorem}
Sei n > 1 und sei \{-1, 1\} die multiplikative Gruppe.
\begin{compactenum}
\item Es gibt einen Epimorphismus $sgn: S_n \to \{-1,+1\}$ mit $sgn(\tau) = 1$ für alle Transpositonen $\tau \in S_n$.
\item Sei K ein Körper und $f: S_n \to K^\times$ ein Homomorphismus. Dann ist entweder f($\tau$) = 1 für alle $\tau \in S_n$, oder es ist char(k) $\neq$ 2 und f = sgn.
\end{compactenum}
\end{theorem}

\begin{remark}
\leavevmode
\begin{itemize}
\item Ist $\pi \in S_n$ und $\pi = \tau_1 \cdots \tau_k$ eine Zerlegung in Transpositionen, so ist nach 1) sgn ein Homomorphismus mit sgn($\pi$) = -1, also ist sgn($\pi$) = (-1)$^k$.

\item Die Zerlegung von $\pi$ in Transpositionen ist nicht eindeutig, aber für jede solche Zerlegung gilt, dass die Parität (gerade oder ungerade) der Anzahl der Faktoren eindeutig ist.\\
Ist $\pi \in S_n$ und $\pi = \zeta_1 \cdots \zeta_l$ eine Zerlegung in disjunkte Zyklen der Länge $k_i, i = 1, …, l$. Es sei m die Anzahl der bewegten Ziffern, also $m = \sum\nolimits_{i=1}^{l} k_i$. Dann folgt $sgn(\pi) = (-1)^{m-l}$.
\end{itemize}
\end{remark}

\begin{definition}
Für n $\ge$ 2 ist $A_n := ker\{sgn : S_n \to \{-1,1\}\}$ die \textbf{alternierende Gruppe}\index{Alternierende Gruppe}\index{Alternierende Gruppe} auf n Ziffern.
\begin{enumerate}
\item Es ist $A_n \vartriangleleft S_n$ und $|S_n : A_n| = 2$.
\item Für jedes $\pi \in S_n$ mit sgn($\pi$)= -1 ist $S_n = A_n \cup \pi A_n = A_n \cup A_n \pi$, wobei die Vereinigung jeweils disjunkt ist.
\end{enumerate}
\end{definition}

\begin{example}
Sei U < S$_n$ eine Untergruppe mit |S$_n$ : U| = 2. Dann ist U = A$_n$.
\end{example}

\begin{remark}
Für n = 3 und n $\ge$ 5 sind \{1\}, A$_n$ und S$_n$ die einzigen Normalteiler von S$_n$, und A$_n$ besitzt nur die trivialen Normalteiler \{1\} und A$_n$ (man sagt A$_n$ ist eine einfache Gruppe). Für n = 4 ist A$_4$ nicht-einfach.
\end{remark}

%%%%%%%%%%%%%%%%%%%%%%%%%%%%%%%%%%%%%%%%%%%%%%
\isection{Ring}
\begin{definition}
Ein \textbf{Ring} R ist eine Menge, zusammen mit zwei Verknüpfungen + und $\cdot$, so dass gilt:
\begin{compactenum}
\item R ist bzgl. + eine abelsche Gruppe (mit neutralem Element 0).
\item Es gibt ein 1 $\in$ R mit 1r = r = r1 für r $\in$ R, und es gilt das Assoziativgesetz $r_1(r_2r_3) = (r_1r_2)r_3$ für $r_1, r_2, r_3 \in R$.
\item Es gelten die Distributivgesetze, d.h. für $r_1, r_2, r_3 \in R$ ist $r_1(r_2 + r_3) = r_1r_2 + r_1r_3$ und $(r_1 + r_2)r_3 = r_1r_3 + r_2r_3$.
\end{compactenum}
Ein Ring R ist kommutativ, falls zusätzlich gilt
\begin{compactenum}
\setcounter{enumi}{3}
\item $r_1r_2 = r_2r_1$ für $r_1,r_2 \in R$.
\end{compactenum}
\end{definition}

\begin{example}
\leavevmode
\begin{compactenum}
\item Jeder Körper ist insbesondere ein kommutativer Ring.
\item Die ganzen Zahlen $\mathbb{Z}$ formen bzgl. der üblichen Addition und Multiplikation einen kommutativen Ring.
\item Ist R ein (kommutativer) Ring, so bildet die Menge der n-Tupel $R^n = \{(r_1,..., r_n) | r_i \in R\}$ bzgl. der komponentenweisen Addition und Multiplikation wieder einen (kommutativen) Ring. Achtung: Selbst wenn R = K ein Körper ist, so ist $R^n$ für n > 1 kein Körper.
\item Die Menge der Polynome K[x] (bzw. R[x]) über einem Körper (bzw. einem kommutativen Ring R) bilden bzgl. der Addition und Mutiplikation von Polynomen einen kommutativen Ring.
\item Sei $R^{n \times n}$ die Menge der quadratischen Matrizen A = ($\alpha_{ij}$) vom Typ (n,n) mit Einträgen $\alpha_{ij}$ aus einem kommutativen Ring R. Dann ist $R^{n \times n}$ bzgl. der Addition und Multiplikation von Matrizen ein Ring. Der Ring $R^{n \times n}$ ist für n $\ge$ 2 in der Regel (z.B.falls 0 $\neq$1 in R) nicht kommutativ.
\end{compactenum}
\end{example}

\begin{definition}
\leavevmode
\begin{enumerate}
	\item R ist ein \textbf{euklidischer}\index{Euklidischer Ring} Ring, falls es eine Funktion $\phi : R\backslash \{0\} \rightarrow \mathbb{N}_0$ gibt, sodass gilt:
	$\forall a,b \in R, b \neq 0$, gibt es $q,r \in R$ mit $a=qb+r$, $r=0$ oder $\phi(r) < \phi (b)$
	\item \textbf{Integritätsbereich}\index{Integritätsbereich}: R ist kommutativ und Nullteilerfrei
	\item R heißt \textbf{faktoriell}\index{Faktorieller Ring}, falls jedes $0 \neq x \in R$ eine eindeutige Primzerlegung hat
\end{enumerate}
\end{definition}

\begin{remark}
\leavevmode
\begin{itemize}
	\item Euklidisch $\Rightarrow$ HIR $\Rightarrow$ faktoriell
	\item $\mathbb{Z}$ mit $\phi(a) = |a|$ ist euklidisch
	\item R euklidisch $\Rightarrow$ R ist Hauptidealring
	\item Sei R kommutativ. Dann gilt $R^n \simeq R^m \Leftrightarrow n=m$
\end{itemize}
\end{remark}

\isubsection{Hauptidealring}
Ein \textbf{Integritätsbereich} $R$ (nullteilerfreier kommutativer Ring mit $1 \neq 0$) heißt HIR, wenn jedes Ideal $I \subseteq R$ ein \textbf{Hauptideal} ist. Es muss folglich gelten:
$I = R \cdot x = \{ r \cdot x | r \in R\}$

\isubsection{Ideal}
\begin{definition}
Sei R ein Ring. Eine Teilmenge $I \subseteq R, I \neq \emptyset$ heißt \textbf{Ideal}, falls
\begin{enumerate}
	\item $a_1,a_2 \in I \Rightarrow a_1+a_2 \in I$
	\item $a \in I \Rightarrow r_1 a r_2 \in I, \forall r_1,r_2 \in R$
\end{enumerate}
\end{definition}

\begin{remark}
\leavevmode
\begin{itemize}
	\item ker(f) ist stets Ideal
	\item falls $1 \in I \Rightarrow I = R$
	\item seien $I_1,I_2 \subseteq R$ Ideale, dann sind auch $I_1 \cap I_2, I_1+I_2, I_1 \cdot I_2$ wieder Ideale
	\item In $\mathbb{Z}$ sind alle Ideale von der Form $I=a\mathbb{Z}, a \in \mathbb{Z}$
	\item falls $I_1+I_2 = R$, so nennt man $I_1$ und $I_2$ teilerfremd.
	\item $I_1+I_2 = R \Rightarrow I_1 \cap I_2 = I_1 \cdot I_2$
	\item R HIR: $(p)$ ist maximal $ \Leftrightarrow $ $p$ ist irreduzibel $\Leftrightarrow$ $p$ ist Primideal
	\item Sei R komm. Ring. Dann gilt:
	\begin{enumerate}
		\item P Primideal $\Leftrightarrow$ $\QR{R}{P}$ Integritätsbereich
		\item M max. Ideal $\Leftrightarrow$ $\QR{R}{M}$ ist Körper
	\end{enumerate}
	\item Sei R Ring mit 1. Sei $I \subsetneq R $ Ideal. Dann gibt es ein maximales Ideal M mit $ I \subseteq M$. Insbesondere existieren max. Ideale.
\end{itemize}
\end{remark}

\subsection{Integritätsbereich / Integritätsring}\index{Integritätsbereich}\index{Integritätsring}
Ein vom Nullring verschiedener nullteilerfreier kommutativer Ring mit dem neutralen Element.

\isubsection{Polynomring}
\begin{definition}
Sei R ein Ring. Der Polynomring R[x] über R ist R[x] = \{($a_0,a_1,...) | a_j \in R$, nur endlich viele $a_j \neq 0$\},
mit Addition und Multiplikation definiert durch $(a_j) + (b_j) = (a_j + b_j)$ und $(a_j)(b_j) = (c_j)$ mit $c_k = \sum\nolimits_{j=0}^{k} a_j b_{k-j}$.
\end{definition}

\begin{lemma}
Sei R ein Ring mit Einselement 1.
\begin{compactenum}
\item R[x] ist ein Ring mit Einselement 1 = (1,0,0,...); der Ring R[x] ist genau dann kommutativ, wenn R kommutativ ist.
\item Die Abbildung R $\to$ R[x], a $\mapsto$ (a,0,0,...) ist ein Monomorphismus von Ringen.
\item Ist K ein Körper, so ist K[x] eine kommutative K-Algebra. Ist x = (0,1,0,...), so ist \{$x^j | j = 0,1,2,...$\} eine K-Basis von K[x].
\end{compactenum}
\end{lemma}


\subsection{Ringhomomorphismus}
\begin{definition}
Seien R, S Ringe.
\begin{enumerate}
\item Eine Abbildung f: R $\to$ S ist ein \textbf{Ringhomomorphismus}\index{Ringhomomorphismus}, falls
\begin{enumerate}
\item $f(r_1 +r_2) = f(r_1) + f(r_2), r_1,r_2 \in R$
\item $f(r_1r_2) = f(r_1)f(r_2), r_1,r_2 \in R$
\item $f(1_R) = 1_S$
\end{enumerate}
\item Ein Monomorphismus (bzw. Epimorphismus, Isomorphismus) ist ein injektiver (bzw. surjektiver, bijektiver) Ringhomomorphimus.
\end{enumerate}
\end{definition}

%%%%%%%%%%%%%%%%%%%%%%%%%%%%%%%%%%%%%%%%%%%%
\isection{Moduln}
\begin{definition}
Eine Menge $M \neq \emptyset$ ist ein R-(Links-) \textbf{Modul}\index{Modul}, falls es eine Verknüpfung $+:M \times M \rightarrow M, (m_1,m_2) \mapsto m_1+m_2$ und eine weitere Verknüpfung $\cdot:R \times M \rightarrow M, (r,m) \mapsto rm$ gibt, sodass:
\begin{itemize}
	\item $(M,+)$ ist abelsche Gruppe
	\item $(r_1+r_2)m = r_1 m + r_2 m$\\$r(m_1+m_2) = rm_1+rm_2$
	\item $1_R m = m$
\end{itemize}
\end{definition}
\begin{remark}
Sei R HIR
\begin{itemize}
	\item Sei $ M \subseteq F = R^n$ ein R-Untermodul. Dann ist $M \simeq R^k$ mit $k \leq n$
	\item Sei $M=<m_1,...,m_n>$ e-e. Modul. Dann gilt: M frei $\Leftrightarrow$ M torsionsfrei
	\item Ist M e-e. R-Modul, so ist $M \simeq T(M) \oplus F, F \simeq R^k$
\end{itemize}
\end{remark}

\isubsection{Untermoduln}
\begin{definition}
Sei M ein R-Modul. Eine Teilmenge $N \subseteq M$ heißt \textbf{Untermodul}, falls:
\begin{enumerate}
	\item $(N,+)$ ist Untergruppe
	\item $rn \in N, \forall r \in R, n\in N$
\end{enumerate}
\end{definition}

\begin{remark}
\leavevmode
\begin{itemize}
	\item Es gilt der Hom. Satz, insbesondere $\QR{M}{\ker(f)} \simeq \im(f)$
	\item $\ker(f),\im(f)$ sind Untermoduln.
\end{itemize}
\end{remark}

\isubsection{Freier Modul}
\begin{theorem}
\leavevmode
\begin{enumerate}
	\item Sei $F = \bigoplus \limits_{i \in I} R e_i $ (F ist freier Modul und die $e_i$ sind eine \enquote{Basis} von F, d.h. $r e_i = 0 \Leftrightarrow r=0$).
	Für $i\in I$ sei ein Element $m_i \in M$ gegeben, wobei M ein R-Modul ist.
	Dann gibt es genau einen Modulhom. $f:F \rightarrow M $ mit $f(e_i) = m_i$
	\item Sei $f:M \twoheadrightarrow F$ ein surjektiver Modulhom., wobei $F = \bigoplus \limits_{i \in I}R e_i$ frei ist. Dann gibt es einen Teilmodul $N \subseteq M$ mit
	\begin{itemize}
		\item $M = N \oplus \ker(f)$
		\item $N \simeq F$
	\end{itemize}
\end{enumerate}
\end{theorem}

\isubsection{Torsionselement}
\begin{definition}
Sei R komm. Integritätsbereich. Sei M ein R-Modul. Ein Element $m\in M$ heißt \textbf{Torsionselement}, falls es ein $r\in R \backslash \{0\}$ gibt mit $rm=0$.

Sei $T(M) := \{m \in M | m~\text{ist torsion} \} \ni 0$. $M$ heißt Torsionsfrei, falls $T(M) = 0$.
\end{definition}

\begin{remark}
\leavevmode
\begin{itemize}
	\item $T(M) \subseteq M$ ist Teilmodul
	\item $\QR{M}{T(M)}$ ist torsionsfrei
\end{itemize}
\end{remark}

\isubsection{$\pi$-Torsionselement}
\begin{definition}
Sei $\pi \in R$ irreduzibel und M ein R-Modul.\\
Dann heißt $T_\pi (M):= \{m \in M | \exists e \in \mathbb{N}: \pi^em = 0 \} \subseteq T(M)$ $\pi$-Torsionsteilmodul von M.
%TODO: inhaltlich prüfen$
\end{definition}

\isubsection{R-Modulhomomorphismus}
\begin{definition}
Sei R ein Ring, M,N zwei R-Moduln. Dann ist $f:M\mapsto N$ ein R-Modulhom., falls:
\begin{itemize}
	\item $f$ ist ein Gruppenhom., d.h. $f(m_1+m_2) = f(m_1)+f(m_2)$
	\item $f(rm) = rf(m), \forall r \in R, n\in \mathbb{N}$
\end{itemize}
\end{definition}

\isubsection{K[x]-Modul}
\begin{theorem}
Sei $A \in End_K(V)$. Dann wird V zu einem K[x]-Modul vermöge\footnote{vermöge = "kann das"} $f(x)v := f(A)v, f\in K[x], v \in V.~(V_A=V)$\\
Seien $A,B \in M_n(K), V=K^n$. TFE\footnote{TFE = The following are equivalent}:
\begin{enumerate}
	\item $A \approx B$
	\item $V_A \simeq V_B$ als K[x]-Modul
	\item $\QR{K[x]^n}{<M_A(x)>} \simeq \QR{K[x]^n}{<M_B(x)>}$ als K[x]-Moduln
	\item $M_A(x) \sim M_B(x)$
	\item $M_A(x)$ und $M_B(x)$ haben die gleichen Elementarteiler
	\item $M_A(x)$ und $M_B(x)$ haben die gleichen Invariantenteiler
\end{enumerate}
\end{theorem}

\isubsection{Elementarteiler}
\begin{theorem}
Sei R euklidischer Ring. Sei M ein e-e. R-Modul. Dann gibt es $k \in \mathbb{N}_0$ (Rang von M) und $\epsilon_1,...,\epsilon_m \in R, \epsilon_i \notin R^\times$ (Elementarteiler) mit
\begin{enumerate}
	\item $M \simeq R^k \oplus \bigoplus \limits_{i=1}^m \QR{R}{\epsilon_i R}$
	\item $\epsilon_1 | \epsilon_2 | ... | \epsilon_m$
\end{enumerate}
\textbf{Zusatz}: $k,\epsilon_i$ sind eindeutig durch den Isomorphietyp von M bestimmt
\end{theorem}

\begin{remark}
Sei R komm. Ring. Dann heißen $a,b \in R$ \textbf{assoziiert}\index{Assoziierte Ringelemente}, in Zeichen $a \sim b$, falls es $u \in R^\times$ gibt mit $a=ub$.

\textbf{Anwendung}: Seien $M_1,M_2$ zwei e-e. Moduln, $\epsilon_1,...,\epsilon_m$ die Elementarteiler von $M_1$, $\mu_1,...,\mu_n$, die Elementarteiler von $M_2$. Dann gilt:
\begin{itemize}
	\item $M_1 \simeq M_2 \Leftrightarrow rg(M_1) = rg(M_2)$
	\item $\epsilon_1 = \mu_1,...,\epsilon_m = \mu_m$ (insbes. $l=m$) %TODO: inhaltlich prüfen$
\end{itemize}
\end{remark}

\isubsection{Invariantenteiler}
\begin{definition}
Die Elemente $\pi_1^{e_{1,1}},...,\pi_1^{e_{s,1}},\pi_2^{e_{1,2}},...,\pi_2^{e_{s,2}},...,\pi_t^{e_{1,t}},...,\pi_t^{e_{s,t}}$ mit 
\begin{align*}
\begin{array}{ll}
e_{1,1} \geq e_{2,1} \geq ... \geq e_{s,1} \geq & 0 \\ 
~~\vdots & \\ 
e_{1,t} \geq e_{2,t} \geq ... \geq e_{s,t} \geq & 0
\end{array} 
\end{align*}
heißen die Invariantenteiler von M.
\end{definition}

\section{Faktorräume}
\begin{definition}
Sei $V$ ein K-Vektorraum und $U \subseteq V$ ein linearer Unterraum. Für $a \in V$ sei $a + U = \{a + u | u \in U\} \subseteq V$. Der \textbf{Quotienten-}\index{Quotientenraum} oder \textbf{Faktorraum}\index{Faktorraum} von V nach U ist die Menge 
\begin{center}
$\QR{V}{U} = \{a + U | a \in V\}$
\end{center}
\end{definition}

\begin{lemma}
Sei V ein K-Vektorraum und U $\subseteq$ V ein linearer Unterraum. Dann ist der Faktorraum V/U ein K-Vektorraum mittels
\begin{center}
$(a_1 + U)+(a_2 + U)=(a_1 +a_2)+U$ und $\alpha(a+U)=\alpha a+U$.
\end{center} 
Die Abbildung q : V $\to$ V /U, a $\mapsto$ a + U ist ein Epimorphismus.
\end{lemma}

\begin{lemma}
Sei V ein K-Vektorraum und seien U $\subseteq$ W $\subseteq$ V linearere Unterräume. Dann gilt:
\begin{enumerate}
\item Ist $\{w_i + U | i \in I\}$ eine Basis von W/U und $\{v_j + W | j \in J\}$ eine Basis von V/W, so ist $\{w_i +U, v_j +U | i \in I, j \in J\}$ eine Basis von V/U.
\item Ist dimV/U = n < $\infty$, so ist dimV/U = dimV/W + dimW/U.
\item Ist dim V = n < $\infty$, so ist dim V/W = dim V - dim W.
\end{enumerate}
\end{lemma}

%%%%%%%%%%%%%%%%%%%%%%%%%%%%%%%%%%%%%%%%%%%%%
\section{K-Algebra}
Eine \textbf{K-Algebra}\index{K-Algebra} ist ”fast ein Köper”, aber
\begin{enumerate}
\item die Multiplikation ist im Allgemeinen nicht kommutativ
\item nicht jedes Element ungleich Null ist invertierbar bezüglich der Multiplikation\\
\end{enumerate}

\begin{definition}
Sei K ein Körper, $\mathcal{A}$ eine K-Algebra und c $\in$ A. Ist f(x) = $\sum\nolimits_{j=0}^{n} a_j x^j \in K[x]$, so setze f(c) = $\sum\nolimits_{j=0}^{n} a_j c^j \in \mathcal{A}$. Die Abbildung
\begin{center}
$\alpha = \alpha_c: K[x] \to A, f \mapsto f(c)$
\end{center}
ist der \textbf{Einsetzungshomomorphismus}\index{Einsetzungshom.} (bzgl. c); $\alpha_c$ ist ein Homomorphismus von K-Algebren, d.h. $\alpha_c$ ist K-linear und es gilt $\alpha_c(f)\alpha_c(g) = \alpha_c(fg)$ für alle f,g $\in$ K[x].
\end{definition}

\begin{example}
\leavevmode
\begin{compactenum}
\item Seien K $\subseteq$ L Körper. Dann ist L eine K-Algebra und für f $\in$ K[x] und c $\in$ L ist f(c) $\in$ L definiert.
\item Ist K endlich mit |K| = q = p$^n$, so gilt c$^q$ = c für alle c $\in$ K. Ist f(x) = x$^q$ - x $\in$ K[x], so ist f $\neq$ 0, aber f(c) = 0 für alle c $\in$ K, d.h. das Polynom f $\in$ K[x] ist von der durch f induzierten Abbildung K $\to$ K, c $\mapsto$ f(c), zu unterscheiden. Ist $\phi: K^2 \to K^2, (x_1,x_2) \mapsto (0,x_1)$ so folgt $f(\phi) = \phi^q - \phi = -\phi \neq 0$, da $\phi^2$ = 0 ist.
\end{compactenum}
\end{example}

\isection{Körper}
\begin{definition}
Ein \textbf{Körper} K ist eine Menge mit zwei Verknüpfungen + und $\cdot$, für die gilt:
\begin{enumerate}
\item (K, +) ist eine abelsche Gruppe mit Nullelement 0,
\item (K \textbackslash \{0\}, $\cdot$) ist eine abelsche Gruppen mit Einselement 1$\neq$ 0,
\item a$\cdot$(b+c)=a$\cdot$b + a$\cdot$c und (a+b)$\cdot$c = a$\cdot$c + a$\cdot$b.
\end{enumerate}
\end{definition}

\begin{remark}
In Körpern gelten viele der ‘üblichen’ \textbf{Rechenregeln}. Für a, b $\in$ K ist:
\begin{enumerate} 
\item 0a = a0 = 0 
\item (-1)a = -a
\item (-a)b = a(-b) = -ab
\item ab = 0 $\Rightarrow$ a = 0 oder b = 0
\end{enumerate}
\end{remark}

\begin{example}
\leavevmode
\begin{enumerate}
\item $\mathbb{Q}$ und $\mathbb{R}$ sind Körper.
\item Sei p eine Primzahl und $\mathbb{Z}/p\mathbb{Z}$ die Menge der Restklassen modulo p. Dann bildet $\mathbb{Z}/p\mathbb{Z}$ \textbackslash\{[0]\} bzgl. der evidenten Multiplikation [a]$\cdot$[b] = [ab] eine abelsche Gruppe, d.h. $\mathbb{Z}/p\mathbb{Z}$ ist ein Körper mit p Elementen. In $\mathbb{Z}/p\mathbb{Z}$ gilt pa = 0 für alle a $\in$ $\mathbb{Z}/p\mathbb{Z}$.
\end{enumerate}
\end{example}

\subsection{Algebraische Abgeschlossenheit}\index{Alg. Abgeschlossenheit}
\begin{definition}
Seien K $\subseteq$ L Körper und sei f $\in$ K[x]. Dann zerfällt f über L, falls es $a, c_1, …, c_n \in L$ gibt, so dass in L[x] gilt
\begin{center}
$f = a \prod\limits_{j=1}^{n}(x - c_j)$.
\end{center}
Ein Körper K ist \textbf{algebraisch abgeschlossen}, falls jedes f $\in$ K[x] mit Grad(f) $\ge$ 1 in K eine Nullstelle hat (also über K zerfällt).
\end{definition}

\begin{remark}
\leavevmode
\begin{compactenum}
\item Der \textbf{Fundamentalsatz der Algebra}\index{Fundamentalsatz d. Alg.} besagt, dass jedes f $\in$ $\mathbb{C}$[x] mit Grad(f) $\ge$ 1 in $\mathbb{C}$ einen Nullstelle besitzt. Also ist $\mathbb{C}$ algebraisch abgeschlossen. Insbesondere gilt: Sind K $\subseteq$ $\mathbb{C}$ Körper und ist f $\in$ K[x], so liegen alle Nullstellen von f in $\mathbb{C}$.
\item Da $x^2 + 1 \in \mathbb{R}[x]$ keine reelle Nullstelle hat, ist $\mathbb{R}$ nicht algebraisch abgeschlossen.
\item Sei K ein endlicher Körper mit |K| = q = p$^n$. Dann gilt c$^q$ = c für alle c $\in$ K. Also hat $f = x^q - x+1 \in K[x]$ keine Nullstelle in K und K ist nicht algebraisch abgeschlossen. Endliche Körper sind also nie algebraisch abgeschlossen.
\item Ein Satz der Algebra besagt, dass es zu jedem Körper K einen algebraisch abgeschlossenen Körper L mit K $\subseteq$ L gibt.
\end{compactenum}
\end{remark}

\subsection{Unterkörper}
\begin{definition}
Sei K ein Körper. Ein \textbf{Unterkörper}\index{Unterkörper} oder \textbf{Teilkörper}\index{Teilkörper} L $\subseteq$ K ist eine Teilmenge, so dass gilt:
\begin{enumerate}
\item a,b $\in$ L $\Rightarrow$ a+b, a$\cdot$b $\in$ L
\item 0, 1 $\in$ L
\item a $\in$ L $\Rightarrow$ -a $\in$ L
\item 0 $\neq$ a $\in$ L $\Rightarrow$ a$^{-1}$ $\in$ L
\end{enumerate}
\end{definition}

\isubsection{Polynome über Körpern}
\begin{proposition}
(\textbf{Division mit Rest}\index{Division mit Rest}): Seien f,g $\in$ K[x] mit g $\neq$ 0. Dann gibt es eindeutig bestimmte h, r $\in$ K[x], so dass gilt f = gh + r mit Grad(r) < Grad(g).
\end{proposition}

\begin{lemma}
Seien K $\subseteq$ L Körper, f $\in$ K[x] und c $\in$ L.
\begin{compactenum}
\item Ist f(c) = 0, so ist f = (x - c)h für ein geeignetes h $\in$ L[x].
\item Ist f $\neq$ 0 und f(c) = 0, so gibt es ein eindeutiges bestimmtes m $\in$ $\mathbb{N}$ und ein eindeutig bestimmtes Polynom h $\in$ L[x] mit f = (x - c)$^m$h und h(c) $\neq$ 0.
\end{compactenum}
\end{lemma}

\begin{definition}
Seien K $\subseteq$ L Körper, f $\in$ K[x] und c $\in$ L.
\begin{compactenum}
\item Ist f $\in$ K[x] und f(c) = 0, so ist c eine Nullstelle von f.
\item Die Zahl m aus Lemma 11.8 (b) nennt man die Vielfachheit der Nullstelle c von f.
\end{compactenum}
\end{definition}

\begin{lemma}
Seien K $\subseteq$ L Körper und sei 0 $\neq$ f $\in$ K[x]. Seien $c_1, …, c_r$ die paarweise verschiedenen Nullstellen von f in L mit Vielfachheiten $m_1 , …, m_r$ . Dann gibt es ein g $\in$ L[x], so dass gilt:
\begin{center}
$f = \prod\limits_{j=1}^{r} (x - c_j)^{mj} g$ und $g(c_j) \neq 0$ für $j = 1, …, r$.
\end{center}
Weiter ist $r \le \sum\nolimits_{j=1}^{r}m_j \le Grad(f)$.\\
Insbesondere hat f höchstens Grad(f) viele verschiedene Nullstellen.
\end{lemma}

Zur Ableitung: Ist Grad(f) = n, so ist Grad(f') = n - 1, falls char(K) $\nmid$ n, und Grad(f') $\le$ n - 1, falls char(K) $\mid$ n.

\begin{lemma}
Sei f $\in$ K[x] mit Grad(f) $\ge$ 1, und sei c $\in$ K. Ist char(K) = 0 oder char(K) > m, so ist c eine m-fache Nullstelle von f genau dann, wenn $f(c) = f'(c) = \cdots = f^{(m-1)}(c) = 0 \neq f^{(m)}(c)$ ist.
\end{lemma}

%%%%%%%%%%%%%%%%%%%%%%%%%%%%%%%%%%%%%%%%%%%%%
\section{Vektorraum}
\begin{definition}
Sei K ein Körper. Ein \textbf{K-Vektorraum}\index{Vektorraum} (oder Vektorraum) ist eine Menge V , zusammen mit einer Verknüpfung V $\times$ V $\to$ V, (a, b) $\mapsto$ a + b und einer Verknüpfung K $\times$ V $\to$ V, ($\alpha$, a) $\mapsto$ $\alpha$ $\cdot$ a (einer ‘Skalarmultiplikation’), so dass gilt:
\begin{enumerate}
\item V ist bzgl. + eine abelsche Gruppe,
\item ($\alpha$ + $\beta$) $\cdot$ a = $\alpha$ $\cdot$ a + $\beta$ $\cdot$ a und $\alpha$ $\cdot$ (a + b) = $\alpha$ $\cdot$ a + $\alpha$ $\cdot$ b,
\item ($\alpha$ $\cdot$ $\beta$) $\cdot$ a = $\alpha$ $\cdot$ ($\beta$ $\cdot$ a) für $\alpha$, $\beta$ $\in$ K und a $\in$ V,
\item 1 $\cdot$ a = a für 1 $\in$ K und a $\in$ V.
\end{enumerate}
\end{definition}

\begin{remark}
Für K-Vektorräume gelten die folgenden \textbf{Rechenregeln}:
\begin{enumerate}
\item $\alpha \cdot 0_V = 0_V$ für alle $\alpha$ $\in$ K,
\item $0_K \cdot a = 0_V$  für alle a $\in$ V,
\item $(-\alpha) \cdot a = \alpha \cdot (-a) = \alpha(-a)$ für $\alpha$ $\in$ K und a $\in$ V,
\item $\alpha \cdot a = 0$ für $\alpha$ $\in$ K und a $\in$ V impliziert $\alpha = 0_K$ oder $a = 0_V$,
\item $\alpha \cdot (\sum\nolimits_{i=1}^{n} a_i) = \sum\nolimits_{i=0}^{n} \alpha a_i$ und $(\sum\nolimits_{i=0}^{n}\alpha_i) \cdot a = \sum\nolimits_{i=0}^{n} \alpha_i a$,
\item $\sum\nolimits_{i=0}^{n} \alpha_i a_i + \sum\nolimits_{i=0}^{n} \beta_i a_i = \sum\nolimits_{i=0}^{n} (\alpha_i + \beta_i) a_i$
\end{enumerate}
\end{remark}

\begin{definition}
Sei V ein K-Vektorraum. Eine Teilmenge U $\subseteq$ V ist ein \textbf{K-Untervektorraum}\index{Untervektorraum} oder \textbf{K-linearer Unterraum} von V, falls gilt:
\begin{enumerate}
\item $\emptyset$ $\neq$ U,
\item a, b $\in$ U $\Rightarrow$ a + b $\in$ U,
\item $\alpha$ $\in$ K, a $\in$ U $\Rightarrow$ $\alpha$ $\cdot$ a $\in$ U (insbesondere: a $\in$ U $\Rightarrow$ -a $\in$ U ).
\end{enumerate}
\end{definition}

\begin{lemma}
\leavevmode
\begin{itemize}
\item Sei V ein K -Vektorraum und sei $(U_i )_{i \in I}$ eine Familie von linearen Unterräumen von V. Dann ist U =$\cap_i U_i \subseteq V$ ebenfalls ein linearer Unterraum.
\item Sei V ein K-Vektorraum und A $\subseteq$ V eine Teilmenge. Dann ist die Menge $\langle A \rangle := \big\{ \sum\nolimits_{i=0}^{n} \alpha_i a_i | n \in \mathbb{N}_0, \alpha_i \in K, a_i \in A \big\} \subseteq V$ ein linearer Unterraum (der von A erzeugte lineare Unterraum). Weiter ist $\langle A \rangle = \cap U$, wobei der Schnitt über alle linearen Unterräume U von V mit A $\subseteq$ U zu erstrecken ist. Also ist $\langle A \rangle$ der kleinste lineare Unterraum, der die Teilmenge A enthält.
\end{itemize}
\end{lemma}

\begin{definition}
Eine Menge $A = \{a_i\}_{i \in I} \subseteq V$ von Elementen eines K-Vektorraums V ist ein \textbf{Erzeugendensystem}\index{Erzeugendensystem} von V, falls $\langle A \rangle$ = V gilt, d.h. falls jeder Vektor a $\in$ V eine Darstellung als endliche Summe
\begin{center}
$a = \sum\nolimits_{i=1}^{n}\alpha_i a_i, \alpha_i \in K, a_i \in A$
\end{center}
besitzt. Der Vektorraum K ist endlich erzeugt (über K), falls V eine endliches Erzeugendensystem A = $\{a_1, …, a_n\}$ besitzt.
\end{definition}

\begin{remark} 
Ist $(a_i)_{i \in I}$ eine Familie von Elementen von V , so definiert man analog den von den Elementen $a_i$ erzeugten linearen Unterraum $\langle a_i | i \in I \rangle \subseteq V$ als den von der Menge A = $\{a_i | i \in I\}$ erzeugten Unterraum. Klar ist 
damit:
\begin{enumerate}
\item $\langle \emptyset \rangle = \{0\}$,
\item $A \subseteq \langle A \rangle$ für jede Teilmenge A $\subseteq$ V, 
\item $U = \langle U \rangle$ für jeden linearen Unterraum U $\subseteq$ V, 
\item Sind A, B $\subseteq$ V Teilmengen, so gilt\\ 
A $\subseteq$ B $\Rightarrow$ $\langle A \rangle \subset \langle B \rangle$, \hspace*{3mm}
A $\subseteq$ $\langle B \rangle$ $\Rightarrow$ $\langle A \rangle \subseteq \langle B \rangle$.
\end{enumerate} 
\end{remark}

\begin{definition}
Sei V ein K -Vektorraum und seien $\{a_i \}_{i \in I}$ Vektoren in V . Die Menge $\{a_i\}_{i \in I}$ ist \textbf{linear unabhängig}\index{Lineare Unabhängigkeit} , falls für jede endliche Teilmenge J $\subseteq$ I gilt:
\begin{center}
$\sum\nolimits_{j \in J} \alpha_j a_j = 0 \Rightarrow \alpha_j = 0$ für alle j $\in$ J.
\end{center}
Sind die $a_i$ nicht linear unabhängig, so sind sie linear abhängig.
\end{definition}

\begin{theorem}
\textbf{Basisergänzungssatz}\index{Basisergänzungssatz} : Sei V ein endlich erzeugter K-Vektorraum, V = $\langle A \rangle$ mit A = $\{a_1,... ,a_n\}$. Sei C $\subseteq$ A eine linear unabhängige Menge von Vektoren. Dann gibt es eine Basis B von V mit C $\subseteq$ B $\subseteq$ A. Insbesondere besitzt jeder endlich erzeugte Vektorraum V eine Basis.
\end{theorem}

\begin{lemma}
\textbf{Austauschlemma}\index{Austauschlemma}: Sei V ein K-Vektorraum und sei B = $\{b_1,... ,b_n\} \subseteq V$ eine Basis von V. Ist $b= \sum\nolimits_{i=1}^{n} \alpha_i b_i$ mit $\alpha_i \in K$ und $\alpha_1 \neq 0$, so ist auch B' = $\{ b, b_1, …, b_n \} \subseteq V$ eine Basis.
\end{lemma}

\begin{theorem}
\textbf{Austauschsatz von Steinitz}\index{Austauschsatz von Steinitz}: Sei V ein K-Vektorraum und $\{b_1,... ,b_n\} \subseteq$ V eine Basis von V. Ist $\{a_1,... ,a_m\} \subseteq$ V eine linear unabhängige Teilmenge, so ist m $\le$ n und bei geeigneter Nummerierung der $b_i$ ist $\{a_1,... ,a_m, b_{m+1},... ,b_n\}$ ebenfalls eine Basis von V.
\end{theorem}

\section{Affine Hyperebene}
\begin{definition}
Falls dim(V) = n < $\infty$ und dim(W) = n-1, so heißt H = a + W \textbf{affine Hyperebene}\index{Affine Hyperebene}.
\end{definition}

\begin{remark}
Erinnerung affiner Unterraum: Ein m-dim. affiner Unterraum von V ist eine Menge M = v + W mit  $v \in V$ und $W \subseteq V$ ein m-dim. linearer Unterraum.\\
Dazu: Sei dim(V) = n < $\infty$. Dann ist jeder m-dim. affiner Unterraum Schnitt von n-m affinen Hyperebenen.
\end{remark}

\section{Invariante Unterräume}
\begin{theorem}
Sei $f \in End(V)$ beliebig. Sei $W \subseteq V$ ein $f$ \textbf{invarianter Unterraum}\index{Invarianter Unterraum}, d.h. $f(W) \subseteq W$. Dann ist $W^\perp$ $f^*$ invariant, d.h. $f^*(w') \subseteq W^\perp$ $\forall w' \in W^\perp$.
\end{theorem}

\section{Orthogonales Komplement / Lotraum}
\begin{theorem}
Sei $M \subseteq V$ eine Menge. $M^\perp = \{v\in V | (v,m)=0~\forall m \in M\}$ definiert das \textbf{orthogonale Komplement}\index{Orthogonales Komplement} / den \textbf{Lotraum}\index{Lotraum} von M. $M^\perp \subseteq V$ ist ein Unterraum.

Sei $U \subseteq V$ ein Unterraum mit $\dim(U) = n < \infty$. Dann gilt:
\begin{enumerate}
	\item $U \oplus U^\perp = V$
	\item $(U^\perp)^\perp = U$
\end{enumerate}
\end{theorem}
