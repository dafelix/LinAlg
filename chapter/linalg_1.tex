%
%\pagenumbering{arabic}
%\addtokomafont{section}{\center}

%\maketitle
%\setcounter{section}{1}

\section{Gruppen}
\begin{definition}
Eine Menge G, zusammen mit einer Verknüpfung $\cdot$ ist eine \textbf{Gruppe}, falls gilt:
\begin{enumerate}
\item $\cdot$ ist assoziativ: a$\cdot$(b$\cdot$c)=(a$\cdot$b)$\cdot$c für alle a,b,c $\in$ G
\item es gibt ein (links)-neutrales Element e $\in$ G mit e $\cdot$ a = a für alle a $\in$ G 
\item zu jedem a $\in$ G gibt es ein (links)-inverses Element, d.h. ein b $\in$ G mit b $\cdot$ a = e
\end{enumerate}
\hspace*{3mm} Die Gruppe G ist kommutativ oder \textbf{abelsch}, falls zusätzlich gilt:
\begin{enumerate}
\setcounter{enumi}{3}
\item a $\cdot$ b = b $\cdot$ a für a, b $\in$ G
\end{enumerate}
\end{definition}

\begin{remark}
\begin{enumerate}
\item Das neutrale Element ist eindeutig bestimmt. Ebenso ist zu jedem a $\in$ G das zugehörige inverse Element eindeutig bestimmt.
\item Ist (G, $\cdot$) eine Gruppe, so schreibe e = 1 (Einselement) und b = a$^{-1}$ für das zu a inverse Element. Sind $a_1, a_2, . . . , a_n \in G$, so schreibe $\prod\nolimits_{i=1}^{n} a_i = a_1 \cdots a_n$; nach Definition gilt $\prod\nolimits_{i=1}^{0}a_i = 1$.
\item Ist (G,+) eine abelsche Gruppe, so setze e = 0 (Nullelement) und bezeichne das zu a inverse Element mit -a. In diesem Fall bezeichnet $\sum\nolimits_{i=1}^{n} a_i$ die Summe der endlich vielen Elemente $a_1, . . . , a_n$; nach Definition $\sum\nolimits_{i=1}^{0} a_i = 0$.
\end{enumerate}
\end{remark}

\begin{definition}
Sei G eine Gruppe. Eine Teilmenge H $\subseteq$ G ist eine \textbf{Untergruppe} von G, H $\le$ G, falls gilt:
\begin{enumerate}
\item 1 $\in$ H
\item a,b $\in$ H $\Rightarrow$ ab $\in$ H
\item a $\in$ H $\Rightarrow$ a$^{-1}$ $\in$ H
\end{enumerate}
\end{definition}

\begin{remark} 
Ist $\emptyset \neq H \subseteq G$ eine nichtleere Teilmenge, so lassen sich die Kriterien (a)-(c) der obigen Definition zu einer Bedingung vereinfachen: 
$\emptyset \neq H \subseteq G$ ist genau dann eine Untergruppe, falls gilt: a,b $\in$ H $\Rightarrow$ ab$^{-1}$ $\in$ H.
\end{remark}

\begin{definition}
Sei G eine Gruppe, U $\le$ G eine Untergruppe, und $\sim$ die durch U definierte Äquivalenzrelation auf G (a $\sim$ b $\Leftrightarrow$ ab$^{-1}$ $\in$ U). Ist a $\in$ G, so ist die entsprechende Äquivalenzklasse die Menge 
$[a]=\{ b \in G | a \sim b \} = \{ b \in G | ab^{-1} \in U \} = \{ b \in G | b = Ua \} = Ua$; 
diese Mengen sind die \textbf{Rechtsnebenklassen} von U. Sind die $Ua_j$ für j $\in$ J, die verschiedenen Rechtsnebenklassen, so bilden diese eine Partition G=$\bigcup\limits_{j \in J} Ua_j$.
\end{definition}

Ist |J| endlich, so nennt man |J| den Index von U in G und schreibt |J| = |G : U|.
\begin{enumerate}
\item Genauso definiert a $\sim$ b $\Leftrightarrow$ a$^{-1}$b $\in$ eine Äquivalenzrelation auf G. Die Äquivalenzklasse von a $\in$ G ist die Linksnebenklasse $[a] = \{b \in G | a \sim b \} = \{ b \in G | a^{-1}b \in U \} = aU$. 
Ist G abelsch, so gilt aU = Ua; für eine nicht-abelsche Gruppe gilt dies im allgemeinen nicht.
\item Der Versuch analog zur Definition der Addition auf $\mathbb{Z}/m\mathbb{Z}$ mittels der Addition auf $\mathbb{Z}$ eine Verknüpfung auf der Menge der Nebenklassen $G/U = \{Ua | a \in G\}$ durch Ua $\cdot$ Ub = Uab zu definieren funktioniert für abelsche Gruppen, aber nicht für allgemeine Gruppen.
\end{enumerate}

\begin{definition}
Seien G und H (multiplikativ geschriebene) Gruppen.
\begin{compactenum}
\item Ein \textbf{Homomorphismus} (oder \textbf{Gruppenhomomorphismus}) ist eine Abbildung f: G $\to$ H, die mit den Gruppenstrukturen verträglich ist, d.h. für $g_1, g_2 \in G$ gilt:\\
$f(g_1g_2) = f(g_1) f(g_2)$.
\item Ist f: G $\to$ H ein Homomorphismus, so setze\\
$im(f) = \{f(g) | g \in G\}$,\\
$ker(f) = \{g \in G | f(g) = 1\}$.
\end{compactenum}
\end{definition}

\begin{definition}
Sei G eine Gruppe. Eine Untergruppe U $\le$ G ist ein \textbf{Normalteiler} (oder eine normale Untergruppe), U $\vartriangleleft$ G, falls gilt
\begin{center}
$u \in U, g \in G \Rightarrow g^{-1}ug \in U$.
\end{center}
Ist U < G (d.h. U $\neq$ G), so schreibe U $\vartriangleleft$ G.\\
Ist G abelsch, so folgt aus $g^{-1}ug = g^{-1}gu = 1u = u \in U$, dass jede Untergruppe ein Normalteiler ist.
\end{definition}

\begin{lemma}
Sei f: G $\to$ H ein Homomorphismus. Dann gilt:
\begin{enumerate}
\item im(f) $\subseteq$ H ist eine Untergruppe
\item ker(f) $\subseteq$ G ist ein Normalteiler
\end{enumerate}
\end{lemma}

\begin{lemma}
Sei N $\vartriangleleft$ G ein Normalteiler und G/N = \{gN | g $\in$ G\}.
\begin{enumerate}
\item Die Menge G/N ist mittels der Verknüpfung $g_1N \cdot g_2N = g_1g_2N, g_1,g_2 \in G$ eine Gruppe mit neutralem Element N
\item Die Abbildung $\pi$: G $\to$ G/N, g $\mapsto$ gN, ist ein Epimorphismus mit ker($\pi$) = N
\end{enumerate}
\end{lemma}

\textbf{Homomorphiesatz} für Gruppen: Seien G, H Gruppen und f: G $\to$ H ein Homomorphismus. Dann gibt es einen Epimorphismus $\pi$: G $\to$ G/ ker(f) und einen Monomorphismus h: G/ ker(f) $\to$ H mit f = h $\circ$ $\pi$ und im(f) = im(h).\\

Die symmetrische Gruppe S$_n$ ist die Gruppe der bijektiven Abbildungen der Menge \{1, …, n\}. Die Gruppenoperation auf S$_n$ ist die Verknüpfung von Abbildungen, S$_n$ ist eine endliche Gruppe mit |S$_n$| = n!. Für n $\ge$ 3 ist S$_n$ nicht abelsch.

\begin{lemma}
\begin{compactenum}
\item Jede Permutation $\tau \in S_n$ hat eine Darstellung als ein Produkt von disjunkten Zyklen (nicht eindeutig).
\item Es gilt $(a_1, a_2, …, a_k) = (a_1, a_k)(a_1, a_{k-1}) \cdots (a_1, a_2)$; insbesondere lässt sich jede Permutation als ein Produkt von Transpositionen schreiben.
\end{compactenum}
\end{lemma}

\begin{theorem}
Sei n > 1 und sei \{-1, 1\} die multiplikative Gruppe.
\begin{compactenum}
\item Es gibt einen Epimorphismus $sgn: S_n \to \{-1,+1\}$ mit $sgn(\tau) = 1$ für alle Transpositonen $\tau \in S_n$.
\item Sei K ein Körper und $f: S_n \to K^\times$ ein Homomorphismus. Dann ist entweder f($\tau$) = 1 für alle $\tau \in S_n$, oder es ist char(k) $\neq$ 2 und f = sgn.
\end{compactenum}
\end{theorem}

\begin{remark}
\begin{itemize}
\item Ist $\pi \in S_n$ und $\pi = \tau_1 \cdots \tau_k$ eine Zerlegung in Transpositionen, so ist nach 1) sgn ein Homomorphismus mit sgn($\pi$) = -1, also ist sgn($\pi$) = (-1)$^k$.

\item Die Zerlegung von $\pi$ in Transpositionen ist nicht eindeutig, aber für jede solche Zerlegung gilt, dass die Parität (gerade oder ungerade) der Anzahl der Faktoren eindeutig ist.\\
Ist $\pi \in S_n$ und $\pi = \zeta_1 \cdots \zeta_l$ eine Zerlegung in disjunkte Zyklen der Länge $k_i, i = 1, …, l$. Es sei m die Anzahl der bewegten Ziffern, also $m = \sum\nolimits_{i=1}^{l} k_i$. Dann folgt $sgn(\pi) = (-1)^{m-l}$.
\end{itemize}
\end{remark}

\begin{definition}
Für n $\ge$ 2 ist $A_n := ker\{sgn : S_n \to \{-1,1\}\}$ die alternierende Gruppe auf n Ziffern.
\begin{enumerate}
\item Es ist $A_n \vartriangleleft S_n$ und $|S_n : A_n| = 2$.
\item Für jedes $\pi \in S_n$ mit sgn($\pi$)= -1 ist $S_n = A_n \cup \pi A_n = A_n \cup A_n \pi$, wobei die Vereinigung jeweils disjunkt ist.
\end{enumerate}
\end{definition}

\begin{example}
Sei U < S$_n$ eine Untergruppe mit |S$_n$ : U| = 2. Dann ist U = A$_n$.
\end{example}

\begin{remark}
Für n = 3 und n $\ge$ 5 sind \{1\}, A$_n$ und S$_n$ die einzigen Normalteiler von S$_n$, und A$_n$ besitzt nur die trivialen Normalteiler \{1\} und A$_n$ (man sagt A$_n$ ist eine einfache Gruppe). Für n = 4 ist A$_4$ nicht-einfach.
\end{remark}


\section{Körper}
\begin{definition}
Ein \textbf{Körper} K ist eine Menge mit zwei Verknüpfungen + und $\cdot$, für die gilt:
\begin{enumerate}
\item (K, +) ist eine abelsche Gruppe mit Nullelement 0,
\item (K \textbackslash \{0\}, $\cdot$) ist eine abelsche Gruppen mit Einselement 1$\neq$ 0,
\item a$\cdot$(b+c)=a$\cdot$b + a$\cdot$c und (a+b)$\cdot$c = a$\cdot$c + a$\cdot$b.
\end{enumerate}
\end{definition}

In Körpern gelten viele der ‘üblichen’ \textbf{Rechenregeln}. Für a, b $\in$ K ist:
\begin{enumerate} 
\item 0a = a0 = 0 
\item (-1)a = -a
\item (-a)b = a(-b) = -ab
\item ab = 0 $\Rightarrow$ a = 0 oder b = 0
\end{enumerate}

\begin{example}
\begin{enumerate}
\item $\mathbb{Q}$ und $\mathbb{R}$ sind Körper.
\item Sei p eine Primzahl und $\mathbb{Z}/p\mathbb{Z}$ die Menge der Restklassen modulo p. Dann bildet $\mathbb{Z}/p\mathbb{Z}$ \textbackslash\{[0]\} bzgl. der evidenten Multiplikation [a]$\cdot$[b] = [ab] eine abelsche Gruppe, d.h. $\mathbb{Z}/p\mathbb{Z}$ ist ein Körper mit p Elementen. In $\mathbb{Z}/p\mathbb{Z}$ gilt pa = 0 für alle a $\in$ $\mathbb{Z}/p\mathbb{Z}$.
\end{enumerate}
\end{example}

\begin{definition}
Sei K ein Körper. Ein \textbf{Teilkörper} L $\subseteq$ K ist eine Teilmenge, so dass gilt:
\begin{enumerate}
\item a,b $\in$ L $\Rightarrow$ a+b, a$\cdot$b $\in$ L
\item 0, 1 $\in$ L
\item a $\in$ L $\Rightarrow$ -a $\in$ L
\item 0 $\neq$ a $\in$ L $\Rightarrow$ a$^{-1}$ $\in$ L
\end{enumerate}
\end{definition}


\section{Vektorräume}
\begin{definition}
Sei K ein Körper. Ein \textbf{K-Vektorraum} ist eine Menge V , zusammen mit einer Verknüpfung V $\times$ V $\to$ V, (a, b) $\mapsto$ a + b und einer Verknüpfung K $\times$ V $\to$ V, ($\alpha$, a) $\mapsto$ $\alpha$ $\cdot$ a (einer ‘Skalarmultiplikation’), so dass gilt:
\begin{enumerate}
\item V ist bzgl. + eine abelsche Gruppe,
\item ($\alpha$ + $\beta$) $\cdot$ a = $\alpha$ $\cdot$ a + $\beta$ $\cdot$ a und $\alpha$ $\cdot$ (a + b) = $\alpha$ $\cdot$ a + $\alpha$ $\cdot$ b,
\item ($\alpha$ $\cdot$ $\beta$) $\cdot$ a = $\alpha$ $\cdot$ ($\beta$ $\cdot$ a) für $\alpha$, $\beta$ $\in$ K und a $\in$ V,
\item 1 $\cdot$ a = a für 1 $\in$ K und a $\in$ V.
\end{enumerate}
\end{definition}

Für K-Vektorräume gelten die folgenden \textbf{Rechenregeln}:
\begin{enumerate}
\item $\alpha \cdot 0_V = 0_V$ für alle $\alpha$ $\in$ K,
\item $0_K \cdot a = 0_V$  für alle a $\in$ V,
\item $(-\alpha) \cdot a = \alpha \cdot (-a) = \alpha(-a)$ für $\alpha$ $\in$ K und a $\in$ V,
\item $\alpha \cdot a = 0$ für $\alpha$ $\in$ K und a $\in$ V impliziert $\alpha = 0_K$ oder $a = 0_V$,
\item $\alpha \cdot (\sum\nolimits_{i=1}^{n} a_i) = \sum\nolimits_{i=0}^{n} \alpha a_i$ und $(\sum\nolimits_{i=0}^{n}\alpha_i) \cdot a = \sum\nolimits_{i=0}^{n} \alpha_i a$,
\item $\sum\nolimits_{i=0}^{n} \alpha_i a_i + \sum\nolimits_{i=0}^{n} \beta_i a_i = \sum\nolimits_{i=0}^{n} (\alpha_i + \beta_i) a_i$
\end{enumerate}

\begin{definition}
Sei V ein K-Vektorraum. Eine Teilmenge U $\subseteq$ V ist ein \textbf{K-Untervektorraum} oder K-linearer Unterraum von V, falls gilt:
\begin{enumerate}
\item $\emptyset$ $\neq$ U,
\item a, b $\in$ U $\Rightarrow$ a + b $\in$ U,
\item $\alpha$ $\in$ K, a $\in$ U $\Rightarrow$ $\alpha$ $\cdot$ a $\in$ U (insbesondere: a $\in$ U $\Rightarrow$ -a $\in$ U ).
\end{enumerate}
\end{definition}

\begin{lemma}
\begin{itemize}
\item Sei V ein K -Vektorraum und sei $(U_i )_{i \in I}$ eine Familie von linearen Unterräumen von V. Dann ist U =$\cap_i U_i \subseteq V$ ebenfalls ein linearer Unterraum.
\item Sei V ein K-Vektorraum und A $\subseteq$ V eine Teilmenge. Dann ist die Menge $\langle A \rangle := \big\{ \sum\nolimits_{i=0}^{n} \alpha_i a_i | n \in \mathbb{N}_0, \alpha_i \in K, a_i \in A \big\} \subseteq V$ ein linearer Unterraum (der von A erzeugte lineare Unterraum). Weiter ist $\langle A \rangle = \cap U$, wobei der Schnitt über alle linearen Unterräume U von V mit A $\subseteq$ U zu erstrecken ist. Also ist $\langle A \rangle$ der kleinste lineare Unterraum, der die Teilmenge A enthält.
\end{itemize}
\end{lemma}

\begin{definition}
Eine Menge $A = \{a_i\}_{i \in I} \subseteq V$ von Elementen eines K-Vektorraums V ist ein \textbf{Erzeugendensystem} von V, falls $\langle A \rangle$ = V gilt, d.h. falls jeder Vektor a $\in$ V eine Darstellung als endliche Summe
\begin{center}
$a = \sum\nolimits_{i=1}^{n}\alpha_i a_i, \alpha_i \in K, a_i \in A$
\end{center}
besitzt. Der Vektorraum K ist endlich erzeugt (über K), falls V eine endliches Erzeugendensystem A = $\{a_1, …, a_n\}$ besitzt.
\end{definition}

\begin{remark} 
Ist $(a_i)_{i \in I}$ eine Familie von Elementen von V , so definiert man analog den von den Elementen $a_i$ erzeugten linearen Unterraum $\langle a_i | i \in I \rangle \subseteq V$ als den von der Menge A = $\{a_i | i \in I\}$ erzeugten Unterraum. Klar ist 
damit:
\begin{enumerate}
\item $\langle \emptyset \rangle = \{0\}$,
\item $A \subseteq \langle A \rangle$ für jede Teilmenge A $\subseteq$ V, 
\item $U = \langle U \rangle$ für jeden linearen Unterraum U $\subseteq$ V, 
\item Sind A, B $\subseteq$ V Teilmengen, so gilt\\ 
A $\subseteq$ B $\Rightarrow$ $\langle A \rangle \subset \langle B \rangle$, \hspace*{3mm}
A $\subseteq$ $\langle B \rangle$ $\Rightarrow$ $\langle A \rangle \subseteq \langle B \rangle$.
\end{enumerate} 
\end{remark}

\begin{definition}
Sei V ein K -Vektorraum und seien $\{a_i \}_{i \in I}$ Vektoren in V . Die Menge $\{a_i\}_{i \in I}$ ist \textbf{linear unabhängig}, falls für jede endliche Teilmenge J $\subseteq$ I gilt:
\begin{center}
$\sum\nolimits_{j \in J} \alpha_j a_j = 0 \Rightarrow \alpha_j = 0$ für alle j $\in$ J.
\end{center}
Sind die $a_i$ nicht linear unabhängig, so sind sie linear abhängig.
\end{definition}

\begin{theorem}
\textbf{Basisergänzungssatz}: Sei V ein endlich erzeugter K-Vektorraum, V = $\langle A \rangle$ mit A = $\{a_1,... ,a_n\}$. Sei C $\subseteq$ A eine linear unabhängige Menge von Vektoren. Dann gibt es eine Basis B von V mit C $\subseteq$ B $\subseteq$ A. Insbesondere besitzt jeder endlich erzeugte Vektorraum V eine Basis.
\end{theorem}

\begin{lemma}
\textbf{Austauschlemma}: Sei V ein K-Vektorraum und sei B = $\{b_1,... ,b_n\} \subseteq V$ eine Basis von V. Ist $b= \sum\nolimits_{i=1}^{n} \alpha_i b_i$ mit $\alpha_i \in K$ und $\alpha_1 \neq 0$, so ist auch B' = $\{ b, b_1, …, b_n \} \subseteq V$ eine Basis.
\end{lemma}

\begin{theorem}
\textbf{Austauschsatz von Steinitz}: Sei V ein K-Vektorraum und $\{b_1,... ,b_n\} \subseteq$ V eine Basis von V. Ist $\{a_1,... ,a_m\} \subseteq$ V eine linear unabhängige Teilmenge, so ist m $\le$ n und bei geeigneter Nummerierung der $b_i$ ist $\{a_1,... ,a_m, b_{m+1},... ,b_n\}$ ebenfalls eine Basis von V.
\end{theorem}


\section{Lineare Abbildungen}
\begin{definition}
Sei K ein Körper und seien V,W K-Vektorräume.
\begin{compactenum}
\item Eine Abbildung f : V $\to$ W heißt \textbf{linear} (oder \textbf{Homomorphismus}), falls für alle $a_1, a_2, a \in V$ und $\alpha \in K$ gilt:
$f (a_1 + a_2) = f (a_1) + f (a_2)$ und $f (\alpha \cdot a) = \alpha \cdot f (a)$.\\
Sei Hom(V,W) = Hom$_K$(V,W) die Menge aller linearen Abbildungen von V nach W; ist V = W, so schreibe $End_K(V )$ = $Hom_K(V,V)$, die Elemente von $End_K(V)$ sind die \textbf{Endomorphismen} von V.
\item Ist f $\in$ Hom$_K$(V,W), so definiere \textbf{Kern} und \textbf{Bild} von f als
\begin{center}
Bild(f) = $\{ f (a) | a \in V\}$,\\
Kern(f) = $\{ a \in V | f(a) = 0\}$.
\end{center}
Oft schreiben wir auch ker(f) und im(f) anstelle von Kern(f) und Bild(f).
\item Sei f $\in$ HomK(V,W). Ist f mengentheoretisch surjektiv (bzw. injektiv), so heißt f \textbf{Epimorphismus} (bzw. \textbf{Monomorphismus}). Ist f bijektiv, so ist f ein \textbf{Isomorphismus}. Gibt es einen Isomorphismus f: V $\to$ W, so sind V und W \textbf{isomorph},V $\cong$ W.
\end{compactenum}
Ist f $\in$ Hom$_K$(V,W), so ist f(0) = 0 und f(a) = f(a).
\end{definition}

\begin{remark}
Seien $\{a_1, …, a_n\}$ und $\{b_1, …, b_m\}$ Basen von V und W. Eine lineare Abbildung $f: V \to W$ ist durch die Bilder der Basisvektoren eindeutig bestimmt. Jedes der Bilder $f(a_j)$ besitzt eine eindeutige Darstellung als Linearkombination der $b_i$, d.h. für j = 1, …, n gibt es eindeutig bestimmte Skalare $\alpha_{ij} \in K$, so dass $f(a_j) = \sum\limits_{i=1}^m \alpha_{ij}b_i$.\\
Wir ordnen diese Vektoren $(\alpha_{1j}, …, \alpha_{mj})$, j= 1, …, n als Spalten einer Matrix an:\\
$\begin{pmatrix} \alpha_{11} & \cdots & \alpha_{1n} \\ \vdots & & \vdots \\ \alpha_{m1} & \cdots & \alpha_{mn} \end{pmatrix}$\\
Damit lässt sich die lineare Abbildung f bez. dieser Basen durch eine solche Abbildungsmatrix darstellen.
\end{remark}

\begin{lemma}
Seien V,W K-Vektorräume und sei f $\in$ Hom$_K$(V,W).
\begin{enumerate}
\item Bild(f) $\subseteq$ W und Kern(f) $\subseteq$ V sind lineare Unterräume.
\item f ist ein Monomorphismus $\Leftrightarrow$ Kern(f) = \{0\}.
\end{enumerate}
\end{lemma}

\begin{lemma}
Seien V und W K-Vektorräume, $\{a_j | j \in J\}$ eine Basis von V, und $\{b_i | i \in I\}$ eine Basis von W.
\begin{compactenum}
\item Seien $c_j \in W$, $j \in J$ beliebig vorgegeben. Dann gibt es genau eine lineare Abbildung f :V $\to$ W mit $f(a_j)=c_j$ für $j \in J$.
\item Seien $\alpha_{ij} \in$ K, i $\in$ I, j $\in$ J, so dass für j $\in$ J nur endlich viele $\alpha_{ij} \neq 0$ sind. Dann gibt es genau ein f $\in$ Hom$_K$(V, W) mit: $f(a_j) = \sum\nolimits_{i \in I} \alpha_{ij} b_i, j \in J$.
\end{compactenum}
\end{lemma}

\begin{theorem}
Seien V und W K-Vektorräume, und sei dim$_K$V = n < $\infty$. TFE:
\begin{enumerate}
\item dim$_K$W = n,
\item Es gibt einen Isomorphismus f : V $\to$ W, d.h. V $\cong$ W.
\end{enumerate}
\end{theorem}

\begin{remark}
Das Theorem besagt, dass jeder n-dimensionale K-Vektorraum isomorph zu K$^n$ ist. Der Beweis zeigt: jeder Isomorphismus bildet eine Basis wieder auf eine Basis ab; damit gilt für isomorphe K-Vektorräume V $\cong$ W (beliebiger Dimension) stets dim$_K$V = dim$_K$W.
\end{remark}

Eine lineare Abbildung f: V $\to$ W ist durch die linearen Unterräume Bild(f) $\subseteq$ W und Kern(f) $\subseteq$ V charakterisiert; das Bild im(f) sind die in W ‘sichtbaren’ Elemente, der Kern ker(f) die Elemente in V, die in W ‘verlorengehen’ (d.h. kein nicht-triviales Bild haben). Um diese linearen Räume studieren zu können führen wir Faktorräume ein:

\subsection{Faktorräume}
\begin{definition}
Sei $V$ ein K-Vektorraum und $U \subseteq V$ ein linearer Unterraum. Für $a \in V$ sei $a + U = \{a + u | u \in U\} \subseteq V$. Der \textbf{Quotienten-} oder \textbf{Faktorraum} von V nach U ist die Menge 
\begin{center}
$\QR{V}{U} = \{a + U | a \in V\}$
\end{center}
\end{definition}

\begin{lemma}
Sei V ein K-Vektorraum und U $\subseteq$ V ein linearer Unterraum. Dann ist der Faktorraum V/U ein K-Vektorraum mittels
\begin{center}
$(a_1 + U)+(a_2 + U)=(a_1 +a_2)+U$ und $\alpha(a+U)=\alpha a+U$.
\end{center} 
Die Abbildung q : V $\to$ V /U, a $\mapsto$ a + U ist ein Epimorphismus.
\end{lemma}

\begin{lemma}
Sei V ein K-Vektorraum und seien U $\subseteq$ W $\subseteq$ V linearere Unterräume. Dann gilt:
\begin{enumerate}
\item Ist $\{w_i + U | i \in I\}$ eine Basis von W/U und $\{v_j + W | j \in J\}$ eine Basis von V/W, so ist $\{w_i +U, v_j +U | i \in I, j \in J\}$ eine Basis von V/U.
\item Ist dimV/U = n < $\infty$, so ist dimV/U = dimV/W + dimW/U.
\item Ist dim V = n < $\infty$, so ist dim V/W = dim V - dim W.
\end{enumerate}
\end{lemma}

\textbf{Homomorphiesatz}: Seien V,W K-Vektorräume und sei f $\in$ Hom$_K$(V, W).
\begin{compactenum}
\item Es gibt einen Monomorphismus \={f}: V/Kern(f) $\to$ W, so dass f = \={f} $\circ$ q und Bild(f) = Bild(\={f}) ist, d.h. das folgende Diagramm kommutiert:
\begin{align*}
\begin{xy}
  \xymatrix{
      V \ar[r]^f \ar[d]_q &  W \\
      V / \ker(f) \ar[ru]_{\overline{f}}& &
  }
\end{xy}
\end{align*}
Hierbei ist q: V $\to$ V/Kern(f) die kanonische Projektion definiert durch q(a) := a + Kern(f).
\item Ist dim$_K$ V = n < $\infty$, so gilt die Formel: \\
dimV = dimKern(f) + dimBild(f).
\end{compactenum}

\begin{lemma}
Seien V, W K-Vektorräume mit dim$_K$V = dim$_K$W = n < $\infty$. Für f $\in$ Hom$_K$(V,W) sind gleichwertig:
\begin{enumerate}
\item f ist ein Isomorphismus,
\item f ist ein Monomorphismus, 
\item f ist ein Epimorphismus.
\end{enumerate}
\end{lemma}


\subsection{Matrizen}
\begin{lemma}
Seien V, W K-Vektorräume.
\begin{compactenum}
\item Für $f, g \in Hom_K(V, W), \alpha \in K$ und $a \in V$ setze (f + g)(a) = f(a) + g(a) und $(\alpha f)(a) = \alpha f(a)$.\\
Mittels dieser Operationen ist Hom$_K$(V, W) ein K-Vektorraum.
\item Seien $\{a_j | j \in J\} \subseteq V$ und $\{b_i | i \in I\} \subseteq W$ Basen. Für $j \in J$ und $i \in I$ definiere $e_{ij} \in Hom_K(V, W)$ durch
\begin{center}
$e_{ij}(a_k) =
\begin{cases}
0 ~~j \neq k\\
b_i~~ j = k\\
\end{cases}$
\end{center}
Dann ist $\{e_{ij} | i \in I, j \in J\}$ eine linear unabhängige Teilmenge von Hom$_K$(V,W). Falls V und W endlich erzeugt sind, so ist $\{e_{ij} | i \in I, j \in J\}$ eine Basis von Hom$_K$(V,W). Insbesondere  gilt dann:\\
dim$_K$Hom(V,W) = dim$_K$V $\cdot$ dim$_K$W.
\end{compactenum}
\end{lemma}

\begin{lemma}
Seien V$_i$ K-Vektorräume, i = 1, 2, 3, 4.
\begin{compactenum}
\item Sind $f \in Hom_K (V_2, V_3)$ und $g \in Hom_K (V_1, V_2)$ so definiert $(fg)(a_1) = f(g(a_1)), a_1 \in V_1$
eine lineare Abbildung $fg \in Hom_K(V_1,V_3)$.
\item Ist $f \in Hom_K (V_2, V_3)$ und sind $g_1, g_2 \in Hom_K (V_1, V_2)$, so gilt\\
$f(g_1 +g_2)=fg_1 +fg_2$.
\item Sind $f_1, f_2 \in Hom_K (V_2, V_3)$ und $g \in Hom_ (V_1, V_2)$, so gilt\\
$(f_1 + f_2)g = f_1g + f_2g$.
\item Für $f \in Hom_K(V_3,V_4)$, $g \in Hom_K(V_2,V_3)$ und $h \in Hom_K(V_1,V_2)$, so gilt: $f(gh) = (fg)h$.
\end{compactenum}
\end{lemma}

\begin{lemma}
Seien $V_i$ K-Vektorräume, i = 1, 2, 3.
\begin{compactenum}
\item Sei $f \in Hom_K(V_1,V_2)$ ein Isomorphismus. Sei $g = f^{-1}$ die inverse Abbildung. Dann ist g linear, d.h. $g \in Hom_K(V_2,V_1)$.
\item Sind $f \in Hom_K(V_1,V_2)$ und $g \in Hom_K(V_2,V_3)$ Isomorphismen, so ist auch $gf \in Hom_K(V_1,V_3)$ ein Isomorphismus; es gilt: $(gf)^{-1} = f^{-1}g^{-1}$.
\end{compactenum}
\end{lemma}

Eine \textbf{K- Algebra} ist ”fast ein Köper”, aber
\begin{enumerate}
\item die Multiplikation ist im Allgemeinen nicht kommutativ
\item nicht jedes Element ungleich Null ist invertierbar bezüglich der Multiplikation\\
\end{enumerate}

\begin{definition}
Das \textbf{Kroneckersymbol} $\delta_{jk}$ ist definiert als $\delta_{jk} = 1$ falls j = k und $\delta_{jk} = 0$ falls j $\neq$ k.
\end{definition}

Ist $\{a_1,... ,a_n\}$ eine Basis von V, so sind bilden die Endomorphismen $e_{ij} \in End_K (V)$ mit $e_{ij}(a_k) = \delta_{jk}a_i$ eine Basis $\{e_{11}, ..., e_{nn}\}$ von $End_K (V)$; es ist $dim_K End_K (V) = n^2$. Für die Basiselemente $\{e_{ij}\}$ von $End_K(V)$ gelten die Formeln
\begin{center}
$e_{ij}e_{kl} = \delta_{jk}e_{il}$ und $\sum\nolimits_{i=1}^{n}e_{ii} = id_V$.
\end{center}
Ist $dim_K V > 1$, so ist die Multiplikation in $End_K (V)$ nicht kommutativ: Die obige Formel liefert $e_{12}e_{22} = \delta_{22}e_{12} = e_{12} \neq 0$ und $e_{22}e_{12} = \delta_{21}e_{22} = 0$, d.h. $e_{12}e_{22} \neq e_{22}e_{12}$.

\begin{definition}
Sei V ein K-Vektorraum. Ist $f \in End_K(V)$ ein Isomorphismus, so nennt man f \textbf{regulär} (auch ‘invertierbar’ bzw. ‘Automorphismus’); ist f nicht regulär, so heißt f singulär. Die regulären Abbildungen aus $End_K (V)$ bilden bzgl. der Multiplikation von Endomorphismen eine Gruppe mit neutralem Element $id_V$; diese Gruppe bezeichnen wir mit GL(V).
\end{definition}

\begin{example}
Sei K ein endlicher Körper mit p Elementen und sei V ein K-Vektorraum der Dimension n. Für zwei (beliebige) endlich-dimensionale K-Vektorräume V,W und $f \in Hom_K(V,W)$ gilt: \\
f ist ein Isomorphismus genau dann, wenn f jede Basis von V auf eine Basis von W abbildet. Also ist die Anzahl der Elemente von GL(V) genau die Anzahl der verschiedenen Basen von V , wobei auch die Reihenfolge der Basiselemente berücksichtigt werden muss. Jede Basis $\{a_1, …, a_n\}$ von V entsteht durch Wahl der $a_i$ wie folgt:\\
$0 \neq a_1 \in V \hspace*{23mm} p^n-1~Möglichkeiten$,\\
$a_1 \in V \textbackslash \langle a_1 \rangle \hspace*{21mm} p^n - p~Möglichkeiten$,\\
$\cdots \hspace*{36mm} \cdots$\\
$a_n \in V \textbackslash \langle a_1, …, a_{n-1} \rangle \hspace*{4mm} p^n - p^{n-1}~Möglichkeiten$.\\
Damit ist $|GL(V )| = (p^n - 1)(p^n - p) \cdots (p^n - p^{n-1})$.
\end{example}

\begin{definition}
Seien V, W K-Vektorräume und sei $f \in Hom_K (V, W)$. Ist $dim_K Bild(f) < \infty$, so ist der \textbf{Rang} r(f) von f definiert als $r(f) = dim_K Bild(f)$.
\begin{enumerate}
\item Wegen Bild(f) $\subseteq$ W ist stets r(f) $\le$ $dim_K W$.
\item Aus dem Homomorphiesatz folgt:\\
$r(f) = dim_K Bild(f) = dim_K V - dim_K Kern(f)$
\end{enumerate}
\end{definition}

Überblick der linearen Abbildungen:\\
\begin{tabular}{lll}
Monomorphismus & f: V $\to$ W& linear, injektiv\\
Epimorphismus & f: V $\to$ W & linear, surjektiv\\
Isomorphismus & f: V $\to$ W & linear, bijektiv\\
Endomorphismus & f: V $\to$ V & linear\\
Automorphismus & f: V $\to$ V & linear, bijektiv\\
\end{tabular}\\

\begin{proposition}
Seien U,V,W K-Vektorräume mit Basen X = $\{u_1,...,u_k\}$,Y = $\{v_1,...,v_n\}$ und Z = $\{w_1,...,w_m\}$. Dann gilt:
\begin{compactenum}
\item Die Abbildung $\kappa: Hom_K(U,V) \to K^{n \times k}, f \mapsto A_{f,X,Y}$ ist ein Isomorphismus.
\item Seien $f \in Hom_K(U,V)$ und $g \in Hom_K(V,W)$. Sind $A_{f,X,Y} = (\alpha_{jl}) \in K^{n \times k}$ und $A_{g,X,Y} = (\beta_{ij}) \in K^{m \times n}$, so ist $A_{gf,X,Z} = (\gamma_{il}) \in K^{m \times k}$ mit
$\gamma_{il} = \sum\nolimits_{j=1}^{n} \beta_{ij} \alpha_{jl}$.
\end{compactenum}
\end{proposition}

\begin{lemma}
Sei V ein K-Vektorraum der Dimension n < $\infty$ und sei $f \in End_K (V)$. Dann sind gleichwertig:
\begin{enumerate}
\item f ist Automorphismus,
\item Für jede Basis X von V ist $A_{f,X}$ invertierbar; weiter gilt $A_{f^{-1},X} = A^{-1}_{f,X}$,
\item Für wenigstens eine Basis X von V ist $A_{f,X}$ invertierbar.
\end{enumerate}
\end{lemma}

\begin{theorem}
Sei f: V $\to$ W linear, V, W endlich erzeugt. Seien X = $\{v_1, …, v_n\}$ und X' = $\{v'_1, …, v'_n\}$ Basen von V und Y = $\{w_1, …, w_m\}$ und Y' = $\{w'_1, …, w'_m\}$ Basen von W. Sei $v'_j = \sum\nolimits_{i=1}^{n} \beta_{ij} v_i$ und $w'_l = \sum\nolimits_{k=1}^{m} \gamma_{kl} w_k$. Sei B = $(\beta_{ij}) \in K^{n \times n}$, C = $(\gamma_{kl}) \in K^{m \times m}$. Dann gilt:
\begin{compactenum}
\item $A_{f, X', Y'} = C^{-1} A_{f, X, Y} B$
\item Sei V = W und f: V $\to$ V, seien X und X' zwei verschiedene Basen von V. Sei $v'_j = \sum\nolimits_{i=1}^{n} \alpha_{ij} v_i$.\\
Dann gilt: $A_{f, X'} = (\alpha_{ij})^{-1} A_{f, X} (\alpha_{ij})$
\end{compactenum}
\end{theorem}


\section{Elementare Umformungen}
Sei $A \in K^{m \times n}$ mit r(A) = r. Dann gibt es invertierbare Matrizen $C \in K^{m \times m}$ und $B \in K^{n \times n}$, so dass gilt: CAB = $\begin{pmatrix} E_r & 0 \\ 0 & 0 \end{pmatrix} \in K^{m \times n}$\\
Es lassen sich Elementarmatrizen dazu verwenden, die obigen Matrizen C und B explizit zu berechnen. Das Rechenverfahren zur Bestimmung von C und B basiert auf der folgenden Beobachtung: Jede Matrix $A \in K^{m \times n}$ lässt sich durch geeignete Zeilenumformungen (d.h. Linksmultiplikation mit Elementarmatrizen) und Spaltenumformungen (d.h. Rechtsmultiplikation mit Elementarmatrizen) in eine Matrix der Form $\begin{pmatrix} E_r & 0 \\ 0 & 0 \end{pmatrix}$ überführen.\\
Also gibt es $T_1, :, T_k \in \mathbb{E}_m$ und $S_1, :, S_l \in \mathbb{E}_n$, so dass $T_k \cdots T_1AS_1 \cdots S_l = \begin{pmatrix} E_r & 0 \\ 0 & 0 \end{pmatrix}$.\\
Also ist $C = T_k \cdots T_1$ und $B = S_1 \cdots S_l$.


\section{Lineare Gleichungen}
(L)  Ax = b\\
Die Matrix A = $(\alpha_{ij})$ definiert eine lineare Abbildung $f = f_A: K^n \to K^m, x \mapsto Ax$. Sei für $b \in K^m$ das Urbild von b unter der Abbildung f mit f$^{-1}$(b) bezeichnet. Damit lässt sich das System (L) wie folgt interpretieren: Ist $b \in K^m$ fest gewählt, so gilt für das Urbild von b
\begin{center}
$f^{-1}(b) = \{x \in K^n | Ax = b\}$,
\end{center}
d.h. die Elemente von $f^{-1}(b)$ sind genau die Lösung von (L).

\begin{lemma}
Sei $A \in K^{m \times n}$ eine Matrix.
\begin{compactenum}
\item Die Lösungen $L_0$ des homogenen Systems Ax = 0 bilden einen linearen Unterraum des $K^n$ der Dimension n - r(A).
\item Ist $x_0$ eine Lösung des inhomogenen Systems Ax = b, so ist $x_0 + L_0 = \{x_0 +y | y \in L_0\}$ die Menge aller Lösungen von Ax = b
\end{compactenum}
\end{lemma}

\begin{proposition}
(\textbf{Existenz}): Sei $A \in K^{m \times n}$ und sei $b \in K^m$. Sei (L) das System Ax = b und sei B = [A, b] die erweiterte Koeffizientenmatrix. Dann ist (L) genau dann lösbar, wenn r(A) = r(B) ist. Insbesondere: Ist b = 0 und n > m, so hat das homogene System Ax = 0 stets eine nicht-triviale Lösung x $\neq$ 0.
\end{proposition}

\begin{remark}
Das System Ax = b ist genau dann für jedes $b \in K^m$ lösbar, wenn r(A) = m ist.
\end{remark}

\begin{proposition}
(\textbf{Eindeutigkeit}): Sei $A \in K^{m \times n},b \in K^m$ und das lineare Gleichungssystem (L) Ax = b habe eine Lösung. Dann hat (L) genau dann eine eindeutige Lösung, wenn Ax = 0 nur die triviale Lösung x = 0 hat; dies gilt genau dann, wenn r(A) = n ist.
\end{proposition}

\begin{remark}
Sei $A \in K^{m \times n}$, so dass Ax = b für alle $b \in K^m$ lösbar ist. Demnach ist dann r(A) = m. Sind diese Lösungen eindeutig, so folgt r(A) = n. Also ist in diesem Fall A vom Typ (n, n) und wegen r(A) = n invertierbar. Ist $A^{-1}$ die inverse Matrix, so sind die eindeutigen Lösungen von Ax = b genau die x = A$^{-1}$b.
\end{remark}

\textbf{Zulässige Umformungen} von (L) sind:
\begin{enumerate}
\item Vertauschen der Zeilen von B (Permutation der Gleichungen),
\item Zeilenübergänge in B der Form $z_i \to z_i+\alpha z_j, i \neq j,\alpha \in K$.
\item Vertauschen der Spalten von A (Permutation der $x_1, …, x_n$).
\end{enumerate}


\section{Determinanten}
\begin{definition}
Sei R ein kommutativer Ring und $A = (\alpha_{ij}) \in R^{n \times n}$ eine quadratische Matrix vom Typ (n,n). Die \textbf{Determinante} von A ist $det(A) = \sum\nolimits_{\tau \in S_n} sgn(\tau) \alpha_{1\tau(1)} \alpha_{1 \tau(2)} \cdots \alpha_{n \tau(n)} \in R$.
\end{definition}

Für $A \in R^{n \times n}$ mit Zeilen $z_1, ..., z_n$ und Spalten $s_1, ..., s_n$ betrachten wir im folgenden det(A) als eine Funktion der Zeilen bzw. Spalten $det(A) = f_{det}(z_1,..., z_n) = g_{det}(s_1, ..., s_n)$.

\begin{lemma}
Sei $A \in R^{n \times n}$. Dann gilt:
\begin{compactenum}
\item det(A) = det(A$^t$).
\item Für $r, r' \in R$ und $z_j,z'_j \in R^n$ gilt die Formel $f_{det}(...,rz_j + r'z'_j,...) = r f_{det}(...,z_j,...) + r' f_{det}(...,z'_j,...)$ (d.h. für R = K ein Körper, und $z_i$ mit i $\neq$ j fest ist die Abbildung $z \mapsto f_{det}(z_1,… , z_{j-1}, z, z_{j+1},… , z_n)$ linear). 
\item Ist $z_i = z_j$ für ein i $\neq$ j, so ist $f_{det}(z_1, ..., z_n) = 0$.
\item Die zu b) und c) analogen Aussagen gelten für $g_{det}(s_1, …, s_n)$.
\end{compactenum}
\end{lemma}

\begin{proposition}
Für eine abstrakte Volumenfunktion V auf R$^n$ gilt:
\begin{compactenum}
\item Für i $\neq$ j und $r \in R$ ist $V(…, z_i + rz_j, …, z_j, …) = V(z_1, …, z_i, …, z_j, …, z_n)$
\item Für $\tau \in S_n$ ist $V(z_{\tau(1)}, …, z_{\tau(n)}) = sgn(\tau)V(z_1, …, z_n)$
\item Ist $z_i = (\alpha_{i1}, …, \alpha_{in})$ und $e_i = (0, …, 0, 1, 0, …, 0)$ das Element von $R^n$ mit 1 an der Stelle i und Nullen sonst, so ist $V(z_1, …, z_n) = det(\alpha_{ij}) V(e_1, …, e_n)$.
\end{compactenum}
\end{proposition}

\begin{remark}
Die letzte der obigen Aussagen besagt, dass jede abstrakte Volumenfunktion auf R$^n$ folgende Form hat:
$V (z_1, …, z_n) = f_{det}(z_1, … z_n) \cdot V (e_1, …, e_n) = f_{det}(z_1, …, z_n) \cdot c$, wobei c = $V (e_1,…, e_n) \in R$ eine Konstante ist.
\end{remark}

\begin{lemma}
(\textbf{Kästchensatz}): Seien $B \in R^{m \times m},C \in R^{n \times n}$ und sei $D \in R^{n \times m}$. Setze k = m + n und betrachte die k $\times$ k-Matrix $A = \begin{pmatrix} B & 0 \\ D & C \end{pmatrix}$. Dann gilt: det(A) = det(B) det(C).
\end{lemma}

\begin{definition}
Sei $A = (\alpha_{ij}) \in R^{n \times n}$, und sei $A_{ij} \in R$ die Determinante der Matrix, die aus A durch Ersetzen der i-ten Zeile durch $e_j = (0,...,0,1,0,...,0)$ mit 1 an der Stelle j entsteht. Die Adjunkte \~{A} von A ist die Matrix $(A_{ij})^t$.\\
Sei A\textbackslash\{ij\} die Matrix, die aus A durch Streichen der i-ten Zeile und der j-ten Spalte entsteht. Aus Lemma 10.7(b) und Lemma 10.9 folgt $A_{ij} =(-1)^{i+j} det(A\textbackslash\{ij\})$.
\end{definition}

Zum Schluss dieses Kapitels betrachten wir ein lineares Gleichungssystem Ax = b mit einer invertierbaren Matrix $A \in K^{n \times n}$ und $b \in K^n$. Das Gleichungssystem hat die eindeutige Lösung y = A$^{-1}$b.

\begin{theorem}
(\textbf{Cramersche Regel}): Es sei wie eben $y = (y_1, …, y_n)^t = A^{-1}b$ die eindeutige Lösung von Ax = b. Dann gilt
\begin{center}
$y_i = \frac{1}{det(A)} det(s_1, …, s_{i-1}, b, s_{i+1}, …, s_n)$.
\end{center}
Hierbei sind $s_1, …, s_n$ die Spalten von A.
\end{theorem}


\section{Polynome und ihre Nullstellen}
\begin{definition}
Seien R, S Ringe.
\begin{enumerate}
\item Eine Abbildung f: R $\to$ S ist ein \textbf{Ringhomomoprhimus}, falls
\begin{enumerate}
\item $f(r_1 +r_2) = f(r_1) + f(r_2), r_1,r_2 \in R$
\item $f(r_1r_2) = f(r_1)f(r_2), r_1,r_2 \in R$
\item $f(1_R) = 1_S$
\end{enumerate}
\item Ein Monomorphismus (bzw. Epimorphismus, Isomorphismus) ist ein injektiver (bzw. surjektiver, bijektiver) Ringhomomorphimus.
\end{enumerate}
\end{definition}

\begin{definition}
Sei R ein Ring. Der Polynomring R[x] über R ist R[x] = \{($a_0,a_1,...) | a_j \in R$, nur endlich viele $a_j \neq 0$\},
mit Addition und Multiplikation definiert durch $(a_j) + (b_j) = (a_j + b_j)$ und $(a_j)(b_j) = (c_j)$ mit $c_k = \sum\nolimits_{j=0}^{k} a_j b_{k-j}$.
\end{definition}

\begin{lemma}
Sei R ein Ring mit Einselement 1.
\begin{compactenum}
\item R[x] ist ein Ring mit Einselement 1 = (1,0,0,...); der Ring R[x] ist genau dann kommutativ, wenn R kommutativ ist.
\item Die Abbildung R $\to$ R[x], a $\mapsto$ (a,0,0,...) ist ein Monomorphismus von Ringen.
\item Ist K ein Körper, so ist K[x] eine kommutative K-Algebra. Ist x = (0,1,0,...), so ist \{$x^j | j = 0,1,2,...$\} eine K-Basis von K[x].
\end{compactenum}
\end{lemma}

\begin{proposition}
(\textbf{Division mit Rest}): Seien f,g $\in$ K[x] mit g $\neq$ 0. Dann gibt es eindeutig bestimmte h, r $\in$ K[x], so dass gilt f = gh + r mit Grad(r) < Grad(g).
\end{proposition}

\begin{definition}
Sei K ein Körper, $\mathcal{A}$ eine K-Algebra und c $\in$ A. Ist f(x) = $\sum\nolimits_{j=0}^{n} a_j x^j \in K[x]$, so setze f(c) = $\sum\nolimits_{j=0}^{n} a_j c^j \in \mathcal{A}$. Die Abbildung
\begin{center}
$\alpha = \alpha_c: K[x] \to A, f \mapsto f(c)$
\end{center}
ist der \textbf{Einsetzungshomomorphismus} (bzgl. c); $\alpha_c$ ist ein Homomorphismus von K-Algebren, d.h. $\alpha_c$ ist K-linear und es gilt $\alpha_c(f)\alpha_c(g) = \alpha_c(fg)$ für alle f,g $\in$ K[x].
\end{definition}

\begin{example}
\begin{compactenum}
\item Seien K $\subseteq$ L Körper. Dann ist L eine K-Algebra und für f $\in$ K[x] und c $\in$ L ist f(c) $\in$ L definiert.
\item Ist K endlich mit |K| = q = p$^n$, so gilt c$^q$ = c für alle c $\in$ K. Ist f(x) = x$^q$ - x $\in$ K[x], so ist f $\neq$ 0, aber f(c) = 0 für alle c $\in$ K, d.h. das Polynom f $\in$ K[x] ist von der durch f induzierten Abbildung K $\to$ K, c $\mapsto$ f(c), zu unterscheiden. Ist $\phi: K^2 \to K^2, (x_1,x_2) \mapsto (0,x_1)$ so folgt $f(\phi) = \phi^q - \phi = -\phi \neq 0$, da $\phi^2$ = 0 ist.
\end{compactenum}
\end{example}

\begin{lemma}
Seien K $\subseteq$ L Körper, f $\in$ K[x] und c $\in$ L.
\begin{compactenum}
\item Ist f(c) = 0, so ist f = (x - c)h für ein geeignetes h $\in$ L[x].
\item Ist f $\neq$ 0 und f(c) = 0, so gibt es ein eindeutiges bestimmtes m $\in$ $\mathbb{N}$ und ein eindeutig bestimmtes Polynom h $\in$ L[x] mit f = (x - c)$^m$h und h(c) $\neq$ 0.
\end{compactenum}
\end{lemma}

\begin{definition}
Seien K $\subseteq$ L Körper, f $\in$ K[x] und c $\in$ L.
\begin{compactenum}
\item Ist f $\in$ K[x] und f(c) = 0, so ist c eine Nullstelle von f.
\item Die Zahl m aus Lemma 11.8 (b) nennt man die Vielfachheit der Nullstelle c von f.
\end{compactenum}
\end{definition}

\begin{lemma}
Seien K $\subseteq$ L Körper und sei 0 $\neq$ f $\in$ K[x]. Seien $c_1, …, c_r$ die paarweise verschiedenen Nullstellen von f in L mit Vielfachheiten $m_1 , …, m_r$ . Dann gibt es ein g $\in$ L[x], so dass gilt:
\begin{center}
$f = \prod\limits_{j=1}^{r} (x - c_j)^{mj} g$ und $g(c_j) \neq 0$ für $j = 1, …, r$.
\end{center}
Weiter ist $r \le \sum\nolimits_{j=1}^{r}m_j \le Grad(f)$.\\
Insbesondere hat f höchstens Grad(f) viele verschiedene Nullstellen.
\end{lemma}

Zur Ableitung: Ist Grad(f) = n, so ist Grad(f') = n - 1, falls char(K) $\nmid$ n, und Grad(f') $\le$ n - 1, falls char(K) $\mid$ n.

\begin{lemma}
Sei f $\in$ K[x] mit Grad(f) $\ge$ 1, und sei c $\in$ K. Ist char(K) = 0 oder char(K) > m, so ist c eine m-fache Nullstelle von f genau dann, wenn $f(c) = f'(c) = \cdots = f^{(m-1)}(c) = 0 \neq f^{(m)}(c)$ ist.
\end{lemma}

\begin{definition}
Seien K $\subseteq$ L Körper und sei f $\in$ K[x]. Dann zerfällt f über L, falls es $a, c_1, …, c_n \in L$ gibt, so dass in L[x] gilt
\begin{center}
$f = a \prod\limits_{j=1}^{n}(x - c_j)$.
\end{center}
Ein Körper K ist \textbf{algebraisch abgeschlossen}, falls jedes f $\in$ K[x] mit Grad(f) $\ge$ 1 in K eine Nullstelle hat (also über K zerfällt).
\end{definition}

\begin{remark}
\begin{compactenum}
\item Der \textbf{Fundamentalsatz der Algebra} besagt, dass jedes f $\in$ $\mathbb{C}$[x] mit Grad(f) $\ge$ 1 in $\mathbb{C}$ einen Nullstelle besitzt. Also ist $\mathbb{C}$ algebraisch abgeschlossen. Insbesondere gilt: Sind K $\subseteq$ $\mathbb{C}$ Körper und ist f $\in$ K[x], so liegen alle Nullstellen von f in $\mathbb{C}$.
\item Da $x^2 + 1 \in \mathbb{R}[x]$ keine reelle Nullstelle hat, ist $\mathbb{R}$ nicht algebraisch abgeschlossen.
\item Sei K ein endlicher Körper mit |K| = q = p$^n$. Dann gilt c$^q$ = c für alle c $\in$ K. Also hat $f = x^q - x+1 \in K[x]$ keine Nullstelle in K und K ist nicht algebraisch abgeschlossen. Endliche Körper sind also nie algebraisch abgeschlossen.
\item Ein Satz der Algebra besagt, dass es zu jedem Körper K einen algebraisch abgeschlossenen Körper L mit K $\subseteq$ L gibt.
\end{compactenum}
\end{remark}


\section{Charakteristisches Polynom und Eigenwerte}
\begin{definition}
Sei V ein K-Vektorraum und f $\in$ End$_K$(V).
Sei $\alpha \in \sigma(f)$\footnote{Das Spektrum $\sigma(f)$ von f ist die Menge aller Eigenwerte}. Dann ist der \textbf{Eigenraum} von f zu $\alpha$ der lineare Unterraum
\begin{center}
$V_f(\alpha) = V(\alpha) = Kern(\alpha id_V - f) = \{v \in V | f(v) = \alpha v\} \subseteq V$.
\end{center}
\end{definition}

\begin{lemma}
Sei V ein K-Vektorraum und f $\in$ End$_K$(V). Sind $v_1, …, v_r \in V$ Eigenvektoren von f zu paarweise verschiedenen Eigenwerten $\alpha_1, ..., \alpha_r$, so sind die $v_1, ..., v_r$ linear unabhängig.
\begin{enumerate}
\item Ist dim$_K$ V = n, so hat f höchstens n verschiedene Eigenwerte.
\item Sei dim$_K$ V = n. Hat f $\in$ End$_K$ (V ) genau n verschiedene Eigenwerte $\alpha_1, …, \alpha_n$, so bilden die entsprechenden Eigenvektoren $v_1, ..., v_n$ eine Basis von V . Die Matrix von f ist bzgl. dieser Basis eine Diagonalmatrix mit Diagonaleinträgen $\alpha_1, …, \alpha_n$.
\end{enumerate}
\end{lemma}

\begin{definition}
Sei V ein n-dimensionaler K-Vektorraum, und f $\in$ End$_K$ (V ) ein Endomorphismus. Sei A = A$_{f,B}$ $\in$ K$^{n \times n}$ die Matrix von f bzgl. einer gewählten Basis B von V . Sei E = E$_n$ $\in$ K$^{n \times n}$ die Einheitsmatrix. Dann ist
\begin{center}
$\chi_f (x) = det(xE - A) \in K[x]$
\end{center}
das \textbf{charakteristische Polynom} von f.
\end{definition}

\begin{lemma}
Sei dim$_K$V < $\infty$ und f $\in$ End$_K$(V). Für $\alpha \in K$ gilt dann:
\begin{center}
$\alpha$ ist \textbf{Eigenwert} von f $\Leftrightarrow$ $\chi_f(\alpha) = 0$.
\end{center}
\end{lemma}


