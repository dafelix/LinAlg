\documentclass[fontsize=10pt]{scrartcl}
 
\usepackage[utf8]{inputenc}
\usepackage[T1]{fontenc}
\usepackage{lmodern}
\usepackage[ngerman]{babel}
\usepackage{amsmath}
\usepackage{amssymb}
\usepackage{amsthm}
\usepackage{amsfonts}
\usepackage[onehalfspacing]{setspace}
\usepackage{stmaryrd}
\usepackage{romannum}
\usepackage{esvect}
\usepackage{paralist}

\usepackage{hyperref}
\usepackage{geometry}
\geometry{a4paper,left=20mm,right=20mm, top=1.5cm, bottom=3cm} 

%\hypersetup{
%  pdftitle    = {Titel},
%  pdfsubject  = {Um was geht es },
%  pdfauthor   = {Autor bzw. Autoren},
%  pdfkeywords = {Stichwort1, Stichwort2 ...} ,
%  pdfcreator  = {Mit welcher Anwendung i.d.R. pdflatex},
%  pdfproducer = {LaTeX with hyperref}
%}

\parindent 0pt

\parindent 0pt

\title{Lineare Algebra}
\author{}
\date{}
\begin{document}
\pagenumbering{arabic}
\addtokomafont{section}{\center}

\maketitle

\section{mehr zu positiv und positiv definit}
\begin{compactitem}
\item Sei A $\in M_n(\mathbb{R})$ symmetrisch. Dann heißt A positiv definit, falls $x^tAx > 0$ $\forall x \neq 0$
\item Sei V K-VR mit ( , ) und dim(V) = n < $\infty$. Dann gilt für $f \in End(V)$:\\
f ist positiv $\Leftrightarrow$ $\exists u \in Gl(V)$ mit $f = u^*u$
\item A positiv $\Leftrightarrow$ \={A} positiv\\
f positiv $\Leftrightarrow$ A positiv
\item Sei B $\in M_n(\mathbb{C})$.\\
B positiv $\Leftrightarrow$ B = B$^*$ und det(B$^{(k)}$) > 0 für alle 1 $\le$ k $\le$ n
\end{compactitem}

\section{Erhaltung innerer Produkte}
\begin{compactitem}
\item Eine lineare Abbildung $f: V \to W$ erhält innere Produkte, falls gilt: $(f(v_1), f(v_2))_W = (v_1, v_2)_V$
\item f erhält innere Produkte $\Rightarrow$ $||f(v)||_W = \sqrt{(f(v), f(v))_W} = \sqrt{(v,v)_V} = ||v||_V$
\item f erhält innere Produkte und ist bijektiv $\Rightarrow$ f$^{-1}$ erhält ebenfalls innere Produkte, denn: $(f^{-1}(w_1), f^{-1}(w_2))_V = (w_1, w_2)_W$
\item Sei dim(V) = n < $\infty$, sei $f: V \to W$ linear. FASÄ:\\
1) f erhält das innere Produkt\\
2) f ist ein Isom. zwischen VR mit innerem Produkt\\
3) f bildet jede ON-Basis von V auf eine ON-Basis von W ab\\
4) f bildet eine ON-Basis von V auf eine ON-Basis von W ab
\item Eine unitäre Abbildung ist ein Isom. $u: V \to V$, der innere Produkte erhält.\\
Die unitären Abbildungen bilden eine Gruppe U(V). Für u $\in$ U(V) gilt:\\
1) Sei dim(V) = n < $\infty$. Dann gilt: u $\in$ U(V) $\Leftrightarrow$ u(ON-Basis) = ON-Basis\\
2) u $\in$ U(V) $\Leftrightarrow$ $\exists u^*$ mit $u^*u=uu^*=id$
\end{compactitem}

\section{unitäre Matrizen}
Eine Matrix A $\in M_n(K)$ heißt unitär, falls AA$^*$ = E\\
Falls K = $\mathbb{R}$ heißt A orthogonal und A$^{-1}$ = A$^t$\\
$\Rightarrow$ Jede unitäre Matrix ist invertierbar und A$^*$ = A$^{-1}$\\
A ist unitär $\Leftrightarrow$ die Spalten von A bilden eine ON-Basis des K$^n$ bez. des Standardskalarprodukts\\
\hspace*{18mm} $\Leftrightarrow$ die Zeilen von A bilden eine ON-Basis des K$^n$ bez. des Standardskalarprodukts\\

\section{Orthogonale Matrizen}
A $\in M_n(K)$ heißt orthogonal, falls A$^t$A = E

\section{zu orthogonal und unitär}
Seien A, B $\in M_n(K)$.\\
A ist unitär ähnlich zu B, falls es eine unitäre Matrix U gibt mit B = U$^*$AU\\
A ist orthogonal ähnlich zu B, falls es eine orthogonale Matrix O gibt mit B = O$^t$AO

\section{Normale Endomorphismen}
$f: V \to V$ heißt normal, falls $ff^*=f^*f$\\
Beispiele:\\
1) Selbstadjungierte, also $f = f^*$\\
2) Unitäre, also $f^{-1} = f^*$\\
3) f normal, $\alpha \in K$ $\Rightarrow$ $\alpha f$ normal\\
Lemma:\\
Sei $f = f^*$. Dann ist jeder Eigenwert von f reell. Die Eigenvektoren zu verschiedenen Eigenwerten sind zueinander orthogonal.

\section{Invariante Unterräume}
Sei f $\in$ End(V) beliebig. Sei W $\subseteq$ V ein f invarianter Unterraum, d.h. f(W) $\subseteq$ W. Dann ist W$^\perp$ $f^*$ invariant, d.h. $f^*(w') \subseteq W^\perp$ $\forall w' \in W^\perp$.

\section{zu selbstadjungiert}
Sei dim(V) = n < $\infty$ und $f = f^*$ ($\Rightarrow$ f normal).\\
Dann gibt es eine ON-Basis von V, die aus Eigenvektoren zu f besteht (d.h. f ist unitär diagonalisierbar).\\
Anders ausgedrückt: Es gibt eine ON-Basis, sodass die darstellende Matrix Diagonalgestalt hat.

\section{Folgerung}
Sei A $\in M_n(\mathbb{C})$ hermitesch, d.h. A$^*$=A. Dann gibt es eine unitäre Matrix U, sodass U$^*$AU Diagonalgestalt hat.\\
Sei A $\in M_n(\mathbb{R})$ symm., d.h. A$^t$=A, dann gibt es eine orthogonale Matrix O, sodass O$^t$AO Diagonalgestalt hat.\\

\section{Zusammenfassung im Fall K=$\mathbb{R}$}
ZU f $\exists$ ON-Basis aus Eigenvektoren $\Leftrightarrow$ f = f$^*$\\
A $\in M_n(\mathbb{R})$ ist orthogonal diagonalisierbar $\Leftrightarrow$ A = A$^t$\\

\section{Eigenwerte von f}
Sei f normal, dann gilt:\\
v ist EV von f  zum EW $\alpha$ $\Leftrightarrow$ v ist EV von f$^*$ zum EW $\overline{\alpha}$

\section{Matrix normal}
Eine Matrix A $\in M_n(\mathbb{C})$ heißt normal, falls AA$^*$ = A$^*$A. Folgerung: Dann gilt für A:\\
$\exists$ unitäre Matrix U mit U$^*$AU hat Diagonalgestalt $\Leftrightarrow$ A ist normal

\section{Projektionen}
$p: V \to V$ heißt Projektion auf U falls $p(V) = U$ und $p|_U = id_U$, bzw. $p^2 = p$
\begin{compactitem}
\item Sei V K-VR mit ( , ), dim(V) < $\infty$ und p: V $\to$ V Projektion. FASÄ:\\
1) p ist normal, d.h. p$^*$p = pp$^*$\\
2) p ist selbstadjungiert, d.h. p = p$^*$\\
3) p ist eine Orthogonalprojektion, d.h. $im(p)^\perp = ker(p)$
\item Sei dim(V) = n < $\infty$ mit ( , ). Seien $W_1, …, W_k \subseteq V$ Unterräume und seien $p_i: V \to V$ Orthogonalprojektionen auf die $W_i$. FASÄ:\\
1) $V = W_1 \oplus … \oplus W_k$ ist eine orthogonale Summe\\
2) $id_V = \sum\limits_{j=1}^k p_j$ und $p_ip_j = 0$ $\forall i \neq j$\\
3) Ist $B_j$ eine ON-Basis von $W_j$, j = 1, …, k, so ist $B := \bigcup\limits_{j=1}^k B_j$ eine ON-Basis von V 
\item Lemma: Sei $f: V \to V$ normal. Dann gilt:\\
1) $f^2(v) = 0$ $\Rightarrow$ $f(v) = 0$ $\forall v \in V$\\
2) $q \in K[x]$ $\Rightarrow$ $q(f)$ ist normal\\
3) Das Minimalpolynom hat keine mehrfachen Nullstellen
\item Theorem: V sei endl.dim. $\mathbb{C}$-VR mit ( , ). Sei $f: V \to V$ normal. Seien $\alpha_1, …, \alpha_k$ die paarweise verschiedenen Eigenwerte. Seien $p_j: V \to V$ die Orthogonalprojektionen auf die $V(\alpha_j)$. Dann git:\\
1) $f = \alpha_1p_1 + … + \alpha_kp_k$\\
2) $id_V = p_1 + … + p_k$\\
3) $p_ip_j = 0$ für $i \neq j$\\
Insbesondere ist $V = V(\alpha_1) \oplus … \oplus V(\alpha_k)$ eine direkte orth. Summe.\\
Bemerkung: Falls es für $f: V \to V$ Orthogonalprojektionen $p_1, …, p_k$ mit 1), 2) und 3) gibt, so ist f normal.
\end{compactitem}

\section{Dualräume}
$V^* = Hom(V, K)$ heißt Dualraum.
\begin{compactitem}
\item Ist $\{ v_i : i \in I \}$ eine Basis von V und sind $\{ c_i | i \in I \}$ beliebige Element von W, so bestimmt die Zuordnung $c_i = f(v_i)$ ein Element $f \in Hom_k(V,W)$.
\item Sei dim(V) = n < $\infty$ und $v_1, …, v_n$ eine Basis. Sei $f_j: V \to K$ definiert durch $f_j(v_i) = \delta{ij}$. Dann ist $f_1, …, f_n \in V^*$ eine Basis von $V^*$, denn $V^* \ni f = f(v_1)f_1 + … + f(v_n)f_n$.\\
Falls $0 = \alpha_1 f_1 + … + \alpha_n f_n$ so fogt $0 = \alpha_i$.\\
Insbesondere: Falls dim(V) < $\infty$, so gilt: dim(V$^*$) = dim(V).\\
\item Lemma: Sei $U \subseteq V$ ein linearer Unterraum und %$v \in \QR{V}{U}$. 
Dann gibt es $f \in V^*$ mit $f(u) = 0$ $\forall u \in u$ und $f(v) \neq 0$
\item Proposition: Sei ${V^*}^* = Hom(Hom(V,K), K)$. Setzte $T: V \to V^*, v \mapsto T(v)$ mit $T(v)(f) := f(v)$ $\forall f \in V^*$. Dann gilt:\\
1) T ist linear und injektiv\\
2) Falls dim(V) < $\infty$, so ist T ein Isomorphismus
\end{compactitem}

\section{Restriktion und Inflation}
Sei $W \subseteq V$ ein lin. UR.\\
1) $R: V^* \to W^*, f \mapsto f|_W$ heißt Restriktion\\
2) %$I: (\QR{V}{W})^* \to V^*, f \mapsto If$
mit $If(v) := f(v + W)$ heißt Inflation von f\\
3) Sei $M \subseteq V$ eine Teilmenge. $M^\perp := \{ f \in V^* | f(m) = 0$ $\forall m \in M \}$. Also: $M^\perp \subseteq V^*$ (lin. UR)\\
4) Sei $S \subseteq V^*$ eine Teilmenge. $S^\top := \{ v \in V | s(v) = 0$ $\forall s \in S \}$. $S^\top$ ist lin. UR von V.\\
Lemma: Sei $W \subseteq V$ ein lin. UR.\\
1) $R: V^* \to W^*$ ist linear und es gilt ker(R) = W$^\perp$\\
2) Falls dim(V) = n < $\infty$, so gilt: dim(W$^\perp$) = dim(V) - dim(W)\\
3) %$I: (\QR{V}{W})^* \to W^\perp$ ist ein Isom.

\section{Dualitätssatz}
1) Ist $W \subseteq V$ ein lin. UR, so ist $W^{\perp \top} = W$\\
2) Sei dim(V) < $\infty$ und $S \subseteq V^*$ ein lin. UR, so gilt  $S^{\top \perp} = S$\\
3) Seien $W_1, W_2 \subseteq V$ lin. UR (V beliebig). Dann gilt:\\
\hspace*{3mm} $(W_1 + W_2)^\perp = W_1^\perp \cap W_2^\perp$\\
\hspace*{3mm} Falls dim(V) < $\infty$, so gilt auch: $(W_1 \cap W_2)^\perp ) W_1^\perp + W_2^\perp$

\section{Affine Hyperebene}
Falls dim(V) = n < $\infty$ und dim(W) = n-1, so heißt H = a + W affine Hyperebene.\\
Erinnerung affiner Unterraum: Ein m-dim. affiner Unterraum von V ist eine Menge M = v + W mit  v $\in$ V und W $\subseteq$ V ein m-dim. linearer Unterraum.\\
Dazu: Sei dim(V) = n < $\infty$. Dann ist jeder m-dim. affiner Unterraum Schnitt von n-m affinen Hyperebenen.

\section{zu Linearformen}
Sei $g: V \to W$ linear und dim(V) = n < $\infty$, dim(W) = m < $\infty$. Seien $v_1, …, v_n$ bzw. $w_1, …, w_m$ Basen von V bzw. W. Sei A = ($\alpha_{ij}$) die Matrix von g bez. dieser Basen, d.h. $g(v_j) ) \sum\limits_{i=1}^m \alpha_{ij} w_i$, j = 1, …, n. Seien $f_1, …, f_n$ bzw. $g_1, …, g_m$ die dualen Basen, d.h. $f_i(v_j) = \delta_{ij}$, $g_k(w_l) = \delta_{kl}$. Dann hat $g^*$ bez. dieser dualen Basen die Matrix $A^t$.\\
Proposition:\\
Sei $g: V \to W$ linear. Dann gilt:\\
1) ker(g$^*$) = im(g)$^\perp$\\
2) ker(g) = im(g$^*$)$^\top$\\
3) g surjektiv $\Leftrightarrow$ g$^*$ injektiv\\
4) g injektiv $\Leftrightarrow$ g$^*$ surjektiv\\
5) g Isomorphismus $\Leftrightarrow$ g$^*$ Isomorphismus

\section{Bilinearform ausgeartet}
Sei $\beta: V \times W \to K$ eine Bilinearform. Dann heißt $\beta$ ausgeartet im 1.Argument, falls es $0 \neq v |in V$ gibt, sodass $\beta(v,w) = 0$ $\forall w \in W$.\\
(ausgeartet im 2.Argument analog)\\
Bemerkungen:
\begin{compactitem}
\item Innere Produkte sind stets nicht-ausgeartet
\item $V = \{ f: \mathbb{R} \to \mathbb{R} | f stetig \}$, $\beta(f,g) = \displaystyle\int_0^1 f(x) g(x) dx$ ist ausgeartet.
\end{compactitem}

\section{Strukturmatrix}
Sei $\beta: V \times W \to K$ eine Bilinearform und dim(V) = n < $\infty$, dim(W) = m < $\infty$. Seien $v_1, …, v_n$ bzw. $w_1, …, w_m$ Basen von V bzw. W.\\
Dann heißt B = ($\beta(v_i, w_j))_{ij}$ $\in K^{n \times m}$ die Strukturmatrix von $\beta$ bez. der gewählten Basen.\\
Beispiele:
\begin{compactitem}
\item Sei $v_1, …, v_n$ Basis von V und $f_1, …, f_n$ die duale Basis. Dann: B = ($f_i(v_j)$) = E
\item Sei $v_1, …, v_n$ ON-Basis von V. Dann: B = (($v_i, v_j))_{ij}$ = E
\item Bez. der Standardbasis ist B = E
\item Bez. der Standardbasen hat $\beta(x,y) := x^tBy$ die Strukturmatrix B
\end{compactitem}
Satz: Seien $v_1, …, v_n$ bzw. $w_1, …, w_m$ Basen von V bzw. W. Sei $\beta: V \times W \to K$ eine Bilinearform und B die Strukturmatrix.\\
Sei $V \ni v = x_1v_1 + … + x_nv_n$, $x_i \in K$ und\\
\hspace*{10mm} $W \ni w = y_1w_1 + … + y_mw_m$, $y_j \in K$.\\
Dann gilt: $\beta(v, w) = x^tBy$\\
Folgerung:\\
Sei dim(V) = dim(W) = n < $\infty$. Sei $\beta: V \times W \to K$ eine Bilinearform mit Strukturmatrix B. FASÄ:\\
1) $\beta$ ist ausgeartet im 1.Argument\\
2) $\beta$ ist ausgeartet im 2.Argument\\
3) det(B) = det(B$^t$) = 0\\
4) rg(B) = rg(B$^t$) < n\\
5) ker(B) = \{0\}\\
6) ker(B$^t$) = \{0\}\\

\end{document}





