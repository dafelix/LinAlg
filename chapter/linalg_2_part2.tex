\section{Lineare Funktionale und adjungierte Abbildungen}
\begin{definition}
$V^* = Hom_K(V,K)$ ist der K-VR der linearen Funktionale. ($K = \{\mathbb{R}, \mathbb{C}\}$).
\end{definition}

\begin{lemma}
Sei dim(V) = n < $\infty$ und $f \in V^*$ $\Rightarrow$ $\exists ! w \in V$ mit $f(v) = (v,w)$ $\forall v \in V$
\end{lemma}

\begin{theorem}
Sei dim(V) = n < $\infty$ mit innerem Produkt und $ f \in End_K(V)$ $\Rightarrow$ $\exists ! f^* \in End_K(V)$ mit $(f(v_1), v_2) = (v_1, f^*(v_2))$
\end{theorem}

\begin{lemma}
Sei dim(V) = n < $\infty$ mit innerem Produkt ( , ).
\begin{enumerate}
\item Sei $f: V \to V$ linear und $a_1, …, a_n$ eine ON-Basis von $V$. Sei $A = (\alpha_{ij})$ die darstellende Matrix von $f$ bez. $a_1, …, a_n$. Dann gilt: $\alpha_{ij} = (f(a_j), a_i)$
\item Sei $B = (\beta_{ij})$ die darstellende Matrix von $f^*$. Dann gilt: $B = A^*$
\end{enumerate}
\end{lemma}

\begin{theorem}
Sei V K-VR mit ( , ), dim(V) = n < $\infty$. Sei $\phi: V \times V \to K$ ein weiteres inneres Produkt. Dann $\exists !$ positiver Endomorphismus $f: V \to V$ mit $\phi(v_1, v_2) = (f(v_1), v_2)$, $\forall v_1, v_2 \in V$
\end{theorem}
\begin{definition}
\leavevmode
\begin{itemize}
\item $f \in End(V)$ heißt \textbf{selbstadjungiert}\index{selbstadjungiert}, falls $f = f^*$\\
$f = f^*$ $\Leftrightarrow$ $A = A^*$ $\Leftrightarrow$ $(f(v), v) \in \mathbb{R}$ $\forall v \in V$
\item $f \in End(V)$ heißt \textbf{positiv}\index{positiv}, falls $f= f^*$ und $(f(v), v) > 0$ $\forall v \in V \backslash \{0\}$
\end{itemize}
\end{definition}
\begin{remark}
\leavevmode
\begin{enumerate}
\item Falls dim(V) < $\infty$, so gibt es stets $f^*$
\item $f^*$ ist eindeutig durch $f$ bestimmt
\item Es gilt also: $f$ ist positiv $\Leftrightarrow$ $p(v_1, v_2) = (f(v_1), v_2)$ ist inners Produkt
\item Falls $K = \mathbb{C}$: $f$ ist positiv $\Leftrightarrow$ $(f(v), v) \in \mathbb{R}$, $\forall v \in V$, und $(f(v), v) > 0$, $\forall v \neq 0$
\end{enumerate}
\end{remark}
\begin{remark}
Sei $V = \mathbb{R}^n$ mit $(x, y) = x^ty$. Zu jedem inneren Produkt $\phi$ gibt es $A \in M_n(\mathbb{R})$ mit $A = A^t$ und $x^tAx > 0$ $\forall x \neq 0$, sodass $\phi (x, y) = (Ax, y) = x^tAy = x^tAy$.\\
Umgekehrt definiert jede solche Matrix ein Skalarprodukt auf $\mathbb{R}^n$.
\end{remark}

\section{mehr zu positiv und positiv definit}
\begin{definition}
Sei $A \in M_n(\mathbb{R})$ symmetrisch. Dann heißt $A$ \textbf{positiv definit}, falls $x^tAx > 0$ $\forall x \neq 0$
\end{definition}
\begin{theorem}
\leavevmode
\begin{compactitem}
\item Sei V K-VR mit ( , ) und dim(V) = n < $\infty$. Dann gilt für $f \in End(V)$:\\
f ist positiv $\Leftrightarrow$ $\exists u \in Gl(V)$ mit $f = u^*u$
\item A positiv $\Leftrightarrow$ \={A} positiv\\
f positiv $\Leftrightarrow$ A positiv
\item Sei $B \in M_n(\mathbb{C})$.\\
$B$ positiv $\Leftrightarrow$ $B = B^*$ und $det(B^{(k)}) > 0$ für alle $1 \le k \le n$
\end{compactitem}
\end{theorem}

\section{Erhaltung innerer Produkte}
\begin{theorem}
\leavevmode
\begin{compactitem}
\item Eine lineare Abbildung $f: V \to W$ erhält innere Produkte, falls gilt: $(f(v_1), f(v_2))_W = (v_1, v_2)_V$
\item f erhält innere Produkte $\Rightarrow$ $||f(v)||_W = \sqrt{(f(v), f(v))_W} = \sqrt{(v,v)_V} = ||v||_V$
\item $f$ erhält innere Produkte und ist bijektiv $\Rightarrow$ $f^{-1}$ erhält ebenfalls innere Produkte, denn: $(f^{-1}(w_1), f^{-1}(w_2))_V = (w_1, w_2)_W$
\item Sei dim(V) = n < $\infty$, sei $f: V \to W$ linear. TFE:
\begin{enumerate}
\item f erhält das innere Produkt
\item f ist ein Isom. zwischen VR mit innerem Produkt
\item f bildet jede ON-Basis von V auf eine ON-Basis von W ab
\item f bildet eine ON-Basis von V auf eine ON-Basis von W ab
\end{enumerate}
Eine unitäre Abbildung ist ein Isom. $u: V \to V$, der innere Produkte erhält.
Die unitären Abbildungen bilden eine Gruppe U(V). Für $u \in U(V)$ gilt:
\begin{enumerate}
\item Sei dim(V) = n < $\infty$. Dann gilt: $u \in U(V)$ $\Leftrightarrow$ $u$(ON-Basis) = ON-Basis
\item $u \in U(V)$ $\Leftrightarrow$ $\exists u^*$ mit $u^*u=uu^*=id$
\end{enumerate}
\end{compactitem}
\end{theorem}

\section{Normale Endomorphismen}
\begin{definition}
$f: V \to V$ heißt normal, falls $ff^*=f^*f$.
\end{definition}

\begin{example}
\leavevmode
\begin{enumerate}
\item Selbstadjungierte, also $f = f^*$
\item Unitäre, also $f^{-1} = f^*$
\item f normal, $\alpha \in K$ $\Rightarrow$ $\alpha f$ normal
\end{enumerate}
\end{example}
\begin{lemma}
Sei $f = f^*$. Dann ist jeder Eigenwert von f reell. Die Eigenvektoren zu verschiedenen Eigenwerten sind zueinander orthogonal.
\end{lemma}

\section{Invariante Unterräume}
\begin{theorem}
Sei $f \in End(V)$ beliebig. Sei $W \subseteq V$ ein $f$ invarianter Unterraum, d.h. $f(W) \subseteq W$. Dann ist $W^\perp$ $f^*$ invariant, d.h. $f^*(w') \subseteq W^\perp$ $\forall w' \in W^\perp$.
\end{theorem}

\section{zu selbstadjungiert}
\begin{lemma}
Sei dim(V) = n < $\infty$ und $f = f^*$ ($\Rightarrow$ f normal).\\
Dann gibt es eine ON-Basis von V, die aus Eigenvektoren zu $f$ besteht (d.h. $f$ ist unitär diagonalisierbar).\\
Anders ausgedrückt: Es gibt eine ON-Basis, sodass die darstellende Matrix Diagonalgestalt hat.

\textbf{Folgerung}: Sei $A \in M_n(\mathbb{C})$ \textbf{hermitesch}, d.h. $A^*=A$. Dann gibt es eine unitäre Matrix $U$, sodass $U^*AU$ Diagonalgestalt hat.\\
Sei $A \in M_n(\mathbb{R})$ symm., d.h. $A^t=A$, dann gibt es eine orthogonale Matrix $O$, sodass $O^tAO$ Diagonalgestalt hat.
\end{lemma}

\section{Zusammenfassung im Fall $K=\mathbb{R}$}
\begin{lemma}
Zu $f$ $\exists$ ON-Basis aus Eigenvektoren $\Leftrightarrow$ $f = f^*$\\
$A \in M_n(\mathbb{R})$ ist orthogonal diagonalisierbar $\Leftrightarrow$ $A = A^t$
\end{lemma}

\section{Eigenwerte von f}
\begin{theorem}
Sei f normal, dann gilt: v ist EV von $f$  zum EW $\alpha$ $\Leftrightarrow$ v ist EV von $f^*$ zum EW $\overline{\alpha}$
\end{theorem}

\section{Matrix normal}
\begin{definition}
Eine Matrix $A \in M_n(\mathbb{C})$ heißt normal, falls $AA^* = A^*A$.
\end{definition}
\begin{lemma}
Dann gilt für $A$: $\exists$ unitäre Matrix $U$ mit $U^*AU$ hat Diagonalgestalt $\Leftrightarrow$ $A$ ist normal
\end{lemma}

\section{Projektionen}
\begin{definition}
$p: V \to V$ heißt Projektion auf U falls $p(V) = U$ und $p|_U = id_U$, bzw. $p^2 = p$
\end{definition}

\begin{theorem}
\leavevmode
\begin{itemize}
\item Sei V K-VR mit ( , ), dim(V) < $\infty$ und $p: V \to V$ Projektion. TFE:
\begin{enumerate}
\item $p$ ist normal, d.h. $p^*p = pp^*$
\item $p$ ist selbstadjungiert, d.h. $p = p^*$
\item $p$ ist eine \textbf{Orthogonalprojektion}, d.h. $im(p)^\perp = ker(p)$
\end{enumerate}
\item Sei dim(V) = n < $\infty$ mit ( , ). Seien $W_1, …, W_k \subseteq V$ Unterräume und seien $p_i: V \to V$ Orthogonalprojektionen auf die $W_i$. TFE:
\begin{enumerate}
\item $V = W_1 \oplus … \oplus W_k$ ist eine orthogonale Summe
\item $id_V = \sum\limits_{j=1}^k p_j$ und $p_ip_j = 0$ $\forall i \neq j$
\item Ist $B_j$ eine ON-Basis von $W_j$, j = 1, …, k, so ist $B := \bigcup\limits_{j=1}^k B_j$ eine ON-Basis von V 
\end{enumerate}
\end{itemize}
\end{theorem}

\begin{lemma}
Sei $f: V \to V$ normal. Dann gilt:
\begin{enumerate}
	\item $f^2(v) = 0$ $\Rightarrow$ $f(v) = 0$ $\forall v \in V$
	\item $q \in K[x]$ $\Rightarrow$ $q(f)$ ist normal
	\item Das Minimalpolynom hat keine mehrfachen Nullstellen
\end{enumerate}
\end{lemma}

\begin{theorem}
V sei endl.dim. $\mathbb{C}$-VR mit ( , ). Sei $f: V \to V$ normal. Seien $\alpha_1, …, \alpha_k$ die paarweise verschiedenen Eigenwerte. Seien $p_j: V \to V$ die Orthogonalprojektionen auf die $V(\alpha_j)$. Dann git:
\begin{enumerate}
\item $f = \alpha_1p_1 + … + \alpha_kp_k$
\item $id_V = p_1 + … + p_k$
\item $p_ip_j = 0$ für $i \neq j$
\end{enumerate}
Insbesondere ist $V = V(\alpha_1) \oplus … \oplus V(\alpha_k)$ eine direkte orth. Summe.
\end{theorem}

\begin{remark}
Falls es für $f: V \to V$ Orthogonalprojektionen $p_1, …, p_k$ mit 1), 2) und 3) gibt, so ist $f$ normal.
\end{remark}

\section{Dualräume}
\begin{definition}
$V^* = Hom(V, K)$ heißt Dualraum.
\end{definition}
\begin{theorem}
\leavevmode
\begin{itemize}
\item Ist $\{ v_i : i \in I \}$ eine Basis von V und sind $\{ c_i | i \in I \}$ beliebige Element von W, so bestimmt die Zuordnung $c_i = f(v_i)$ ein Element $f \in Hom_K(V,W)$.
\item Sei dim(V) = n < $\infty$ und $v_1, …, v_n$ eine Basis. Sei $f_j: V \to K$ definiert durch $f_j(v_i) = \delta_{ij}$. Dann ist $f_1, …, f_n \in V^*$ eine Basis von $V^*$, denn $V^* \ni f = f(v_1)f_1 + … + f(v_n)f_n$.\\
Falls $0 = \alpha_1 f_1 + … + \alpha_n f_n$, so folgt $0 = \alpha_i$.\\
Insbesondere: Falls dim(V) < $\infty$, so gilt: dim(V$^*$) = dim(V).
\end{itemize}
\end{theorem}

\begin{lemma}
Sei $U \subseteq V$ ein linearer Unterraum und $v \in \QR{V}{U}$. 
Dann gibt es $f \in V^*$ mit $f(u) = 0$ $\forall u \in u$ und $f(v) \neq 0$
\end{lemma}

\begin{proposition}
Sei ${V^*}^* = Hom(Hom(V,K), K)$. Setzte $T: V \to V^*, v \mapsto T(v)$ mit $T(v)(f) := f(v)$ $\forall f \in V^*$. Dann gilt:
\begin{enumerate}
\item T ist linear und injektiv
\item Falls dim(V) < $\infty$, so ist T ein Isomorphismus
\end{enumerate}
\end{proposition}

\section{Restriktion und Inflation}
\begin{definition}
Sei $W \subseteq V$ ein lin. UR.
\begin{enumerate}
\item $R: V^* \to W^*, f \mapsto f|_W$ heißt \textbf{Restriktion}
\item $I: \big(\QR{V}{W}\big)^* \to V^*, f \mapsto If$ mit $If(v) := f(v + W)$ heißt \textbf{Inflation} von f
\item Sei $M \subseteq V$ eine Teilmenge. $M^\perp := \{ f \in V^* | f(m) = 0$ $\forall m \in M \}$. Also: $M^\perp \subseteq V^*$ (lin. UR)
\item Sei $S \subseteq V^*$ eine Teilmenge. $S^\top := \{ v \in V | s(v) = 0$ $\forall s \in S \}$. $S^\top$ ist lin. UR von V.
\end{enumerate}
\end{definition}

\begin{lemma}
Sei $W \subseteq V$ ein lin. UR.
\begin{enumerate}
\item $R: V^* \to W^*$ ist linear und es gilt ker(R) = W$^\perp$
\item Falls dim(V) = n < $\infty$, so gilt: dim(W$^\perp$) = dim(V) - dim(W)
\item $I: \big(\QR{V}{W}\big)^* \to W^\perp$ ist ein Isomorphismus
\end{enumerate}
\end{lemma}

\section{Dualitätssatz}
\begin{theorem}
\leavevmode
\begin{enumerate}
\item Ist $W \subseteq V$ ein lin. UR, so ist $W^{\perp \top} = W$
\item Sei dim(V) < $\infty$ und $S \subseteq V^*$ ein lin. UR, so gilt  $S^{\top \perp} = S$
\item Seien $W_1, W_2 \subseteq V$ lin. UR (V beliebig). Dann gilt:\\
$(W_1 + W_2)^\perp = W_1^\perp \cap W_2^\perp$\\
Falls dim(V) < $\infty$, so gilt auch: $(W_1 \cap W_2)^\perp = W_1^\perp + W_2^\perp$
\end{enumerate}
\end{theorem}

\section{Affine Hyperebene}
\begin{definition}
Falls dim(V) = n < $\infty$ und dim(W) = n-1, so heißt H = a + W affine Hyperebene.
\end{definition}

\begin{remark}
Erinnerung affiner Unterraum: Ein m-dim. affiner Unterraum von V ist eine Menge M = v + W mit  $v \in V$ und $W \subseteq V$ ein m-dim. linearer Unterraum.\\
Dazu: Sei dim(V) = n < $\infty$. Dann ist jeder m-dim. affiner Unterraum Schnitt von n-m affinen Hyperebenen.
\end{remark}

\section{zu Linearformen}
\begin{lemma}
Sei $g: V \to W$ linear und dim(V) = n < $\infty$, dim(W) = m < $\infty$. Seien $v_1, …, v_n$ bzw. $w_1, …, w_m$ Basen von V bzw. W. Sei A = ($\alpha_{ij}$) die Matrix von g bez. dieser Basen, d.h. $g(v_j) ) \sum\limits_{i=1}^m \alpha_{ij} w_i$, j = 1, …, n. Seien $f_1, …, f_n$ bzw. $g_1, …, g_m$ die dualen Basen, d.h. $f_i(v_j) = \delta_{ij}$, $g_k(w_l) = \delta_{kl}$. Dann hat $g^*$ bez. dieser dualen Basen die Matrix $A^t$.
\end{lemma}

\begin{proposition}
Sei $g: V \to W$ linear. Dann gilt:
\begin{enumerate}
\item $ker(g^*) = im(g)^\perp$
\item $ker(g) = im(g^*)^\top$
\item $g$ surjektiv $\Leftrightarrow$ $g^*$ injektiv
\item $g$ injektiv $\Leftrightarrow$ $g^*$ surjektiv
\item $g$ Isomorphismus $\Leftrightarrow$ $g^*$ Isomorphismus
\end{enumerate}
\end{proposition}

\section{Bilinearform ausgeartet}
\begin{definition}
Sei $\beta: V \times W \to K$ eine Bilinearform. Dann heißt $\beta$ ausgeartet im 1.Argument, falls es $0 \neq v \in V$ gibt, sodass $\beta(v,w) = 0$ $\forall w \in W$.\\
(ausgeartet im 2.Argument analog)
\end{definition}
\begin{remark}
\leavevmode
\begin{compactitem}
\item Innere Produkte sind stets nicht-ausgeartet
\item $V = \{ f: \mathbb{R} \to \mathbb{R} | f~stetig \}$, $\beta(f,g) = \displaystyle\int_0^1 f(x) g(x) dx$ ist ausgeartet.
\end{compactitem}
\end{remark}
