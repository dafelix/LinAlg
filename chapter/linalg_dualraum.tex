\chapter{Dualräume}
\begin{definition}
$V^* = Hom(V, K)$ heißt \textbf{Dualraum}\index{Dualraum}.
\end{definition}
\begin{theorem}
\leavevmode
\begin{itemize}
\item Ist $\{ v_i : i \in I \}$ eine Basis von V und sind $\{ c_i | i \in I \}$ beliebige Element von W, so bestimmt die Zuordnung $c_i = f(v_i)$ ein Element $f \in Hom_K(V,W)$.
\item Sei dim(V) = n < $\infty$ und $v_1, …, v_n$ eine Basis. Sei $f_j: V \to K$ definiert durch $f_j(v_i) = \delta_{ij}$. Dann ist $f_1, …, f_n \in V^*$ eine Basis von $V^*$, denn $V^* \ni f = f(v_1)f_1 + … + f(v_n)f_n$.\\
Falls $0 = \alpha_1 f_1 + … + \alpha_n f_n$, so folgt $0 = \alpha_i$.\\
Insbesondere: Falls dim(V) < $\infty$, so gilt: dim(V$^*$) = dim(V).
\end{itemize}
\end{theorem}

\begin{lemma}
Sei $U \subseteq V$ ein linearer Unterraum und $v \in \QR{V}{U}$. 
Dann gibt es $f \in V^*$ mit $f(u) = 0$ $\forall u \in u$ und $f(v) \neq 0$
\end{lemma}

\begin{proposition}
Sei ${V^*}^* = Hom(Hom(V,K), K)$. Setzte $T: V \to V^*, v \mapsto T(v)$ mit $T(v)(f) := f(v)$ $\forall f \in V^*$. Dann gilt:
\begin{enumerate}
\item T ist linear und injektiv
\item Falls dim(V) < $\infty$, so ist T ein Isomorphismus
\end{enumerate}
\end{proposition}

\isection{Dualitätssatz}
\begin{theorem}
\leavevmode
\begin{enumerate}
\item Ist $W \subseteq V$ ein lin. UR, so ist $W^{\perp \top} = W$
\item Sei dim(V) < $\infty$ und $S \subseteq V^*$ ein lin. UR, so gilt  $S^{\top \perp} = S$
\item Seien $W_1, W_2 \subseteq V$ lin. UR (V beliebig). Dann gilt:\\
$(W_1 + W_2)^\perp = W_1^\perp \cap W_2^\perp$\\
Falls dim(V) < $\infty$, so gilt auch: $(W_1 \cap W_2)^\perp = W_1^\perp + W_2^\perp$
\end{enumerate}
\end{theorem}

\section{Restriktion und Inflation}
\begin{definition}
Sei $W \subseteq V$ ein lin. UR.
\begin{enumerate}
\item $R: V^* \to W^*, f \mapsto f|_W$ heißt \textbf{Restriktion}\index{Restriktion}
\item $I: \big(\QR{V}{W}\big)^* \to V^*, f \mapsto If$ mit $If(v) := f(v + W)$ heißt \textbf{Inflation} \index{Inflation} von f
\item Sei $M \subseteq V$ eine Teilmenge. $M^\perp := \{ f \in V^* | f(m) = 0$ $\forall m \in M \}$. Also: $M^\perp \subseteq V^*$ (lin. UR)
\item Sei $S \subseteq V^*$ eine Teilmenge. $S^\top := \{ v \in V | s(v) = 0$ $\forall s \in S \}$. $S^\top$ ist lin. UR von V.
\end{enumerate}
\end{definition}

\begin{lemma}
Sei $W \subseteq V$ ein lin. UR.
\begin{enumerate}
\item $R: V^* \to W^*$ ist linear und es gilt ker(R) = W$^\perp$
\item Falls dim(V) = n < $\infty$, so gilt: dim(W$^\perp$) = dim(V) - dim(W)
\item $I: \big(\QR{V}{W}\big)^* \to W^\perp$ ist ein Isomorphismus
\end{enumerate}
\end{lemma}

\isection{Normale Endomorphismen}
\begin{definition}
$f: V \to V$ heißt \textbf{normal}, falls $ff^*=f^*f$.
\end{definition}

\begin{example}
\leavevmode
\begin{enumerate}
\item Selbstadjungierte, also $f = f^*$
\item Unitäre, also $f^{-1} = f^*$
\item f normal, $\alpha \in K$ $\Rightarrow$ $\alpha f$ normal
\end{enumerate}
\end{example}
\begin{lemma}
Sei $f = f^*$. Dann ist jeder Eigenwert von f reell. Die Eigenvektoren zu verschiedenen Eigenwerten sind zueinander orthogonal.
\end{lemma}

\section{zu Linearformen}
\begin{lemma}
Sei $g: V \to W$ linear und dim(V) = n < $\infty$, dim(W) = m < $\infty$. Seien $v_1, …, v_n$ bzw. $w_1, …, w_m$ Basen von V bzw. W. Sei A = ($\alpha_{ij}$) die Matrix von g bez. dieser Basen, d.h. $g(v_j) ) \sum\limits_{i=1}^m \alpha_{ij} w_i$, j = 1, …, n. Seien $f_1, …, f_n$ bzw. $g_1, …, g_m$ die dualen Basen, d.h. $f_i(v_j) = \delta_{ij}$, $g_k(w_l) = \delta_{kl}$. Dann hat $g^*$ bez. dieser dualen Basen die Matrix $A^t$.
\end{lemma}

\begin{proposition}
Sei $g: V \to W$ linear. Dann gilt:
\begin{enumerate}
\item $ker(g^*) = im(g)^\perp$
\item $ker(g) = im(g^*)^\top$
\item $g$ surjektiv $\Leftrightarrow$ $g^*$ injektiv
\item $g$ injektiv $\Leftrightarrow$ $g^*$ surjektiv
\item $g$ Isomorphismus $\Leftrightarrow$ $g^*$ Isomorphismus
\end{enumerate}
\end{proposition}

